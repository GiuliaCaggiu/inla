\documentclass[a4paper,11pt]{article}
\usepackage[scale={0.8,0.9},centering,includeheadfoot]{geometry}
\usepackage{amstext}
\usepackage{listings}
\usepackage{verbatim}
\begin{document}

\section*{Autoregressive model of order $p$ (AR$(p)$)}

\subsection*{Parametrization}

The autoregressive model of order $p$ (AR1$(p)$) for the Gaussian vector
$\mathbf{x}=(x_1,\dots,x_n)$ is defined as (in obvious notation)
\begin{displaymath}
    x_{t} = \phi_{1}x_{t-1} + \phi_{2}x_{t-2} + \cdots + \phi_{p}
    x_{t-p} + \epsilon_{t}
\end{displaymath}
for $t = p, \ldots, n$, and where the innovation process
$\{\epsilon_{t}\}$ has fixed precision.

The AR$(p)$ process has an awkward parameterisation, as there are
severe non-linear constraints on the $\phi_{\cdot}$-parameters for it
to define a stationary model. Therefore we re-parameterized using the
partial autocorrelation autocorrelation function, $\{\psi_{k}, k=1,
\ldots, p\}$, where $|\psi_{k}|< 1$ for all $k$ \footnote{See for
    example
    \texttt{https://en.wikipedia.org/wiki/Partial\_autocorrelation\_function}.
    For $p=1$, then $\psi_{1} = \phi_{1}$, and for $p=2$, then
    $\psi_{1} = \phi_{1}/(1-\phi_{2})$ and $\psi_{2} = \phi_{2}$.}
and its \emph{marginal (NOT conditional) precision}
$\tau$. Furthermore, the joint distribution for $\{x_{t}, t=1, \ldots,
p\}$, is set to the stationary distribution for the process, hence
there are no boundary issues.

\subsection*{Hyperparameters}

The marginal precision parameter $\tau$ is represented as
\begin{displaymath}
    \theta_1 =\log(\tau) 
\end{displaymath}
and the prior for the marginal precision is defined on
$\theta_{1}$. The partial autocorrelation function $\{\psi_{k}\}$ is
represented
\begin{displaymath}
    \psi_{k} = 2\frac{\exp(\theta_{k+1})}{1+\exp(\theta_{k+1})} -1
\end{displaymath}
for $k = 1, \ldots, p$. The prior for $\{\theta_{k+1}, k=1, \ldots,
p\}$ is \emph{defined} to be multivariate normal with mean $\mu$ and
precision matrix $Q$.

\subsection*{Specification}

The AR$(p)$ model is specified inside the {\tt f()} function as
\begin{verbatim}
 f(<whatever>, model="ar", order=<p>, hyper = <hyper>)
\end{verbatim}
The option \texttt{order} ($>0$) is required. The multivariate normal
prior for $\{\theta_{k+1}, k=1, \ldots, p\}$, \texttt{is specified} as
the parameters to the prior for $\theta_{2}$ (the first
pacf-parameter), and the parameters to the multivariate normal prior
(\texttt{mvnorm}), is \texttt{c($\mu, Q$)}; see the example below.

\subsubsection*{Hyperparameter spesification and default values}
\begin{description}
	\item[hyper]\ 
	 \begin{description}
	 	\item[theta1]\ 
	 	 \begin{description}
	 	 	 \item[ name ] log precision 
	 	 	 \item[ short.name ] prec 
	 	 	 \item[ initial ] 4 
	 	 	 \item[ fixed ] FALSE 
	 	 	 \item[ prior ] loggamma 
	 	 	 \item[ param ] 1 5e-05 
	 	 	 \item[ to.theta ] \verb|function(x) log(x)| 
	 	 	 \item[ from.theta ] \verb|function(x) exp(x)| 
	 	 \end{description}
	 	\item[theta2]\ 
	 	 \begin{description}
	 	 	 \item[ name ] pacf1 
	 	 	 \item[ short.name ] pacf1 
	 	 	 \item[ initial ] 2 
	 	 	 \item[ fixed ] FALSE 
	 	 	 \item[ prior ] mvnorm 
	 	 	 \item[ param ] 0 0.15 
	 	 	 \item[ to.theta ] \verb|function(x) log((1+x)/(1-x))| 
	 	 	 \item[ from.theta ] \verb|function(x) 2*exp(x)/(1+exp(x))-1| 
	 	 \end{description}
	 	\item[theta3]\ 
	 	 \begin{description}
	 	 	 \item[ name ] pacf2 
	 	 	 \item[ short.name ] pacf2 
	 	 	 \item[ initial ] 0 
	 	 	 \item[ fixed ] TRUE 
	 	 	 \item[ prior ] none 
	 	 	 \item[ param ]  
	 	 	 \item[ to.theta ] \verb|function(x) log((1+x)/(1-x))| 
	 	 	 \item[ from.theta ] \verb|function(x) 2*exp(x)/(1+exp(x))-1| 
	 	 \end{description}
	 	\item[theta4]\ 
	 	 \begin{description}
	 	 	 \item[ name ] pacf3 
	 	 	 \item[ short.name ] pacf3 
	 	 	 \item[ initial ] 0 
	 	 	 \item[ fixed ] TRUE 
	 	 	 \item[ prior ] none 
	 	 	 \item[ param ]  
	 	 	 \item[ to.theta ] \verb|function(x) log((1+x)/(1-x))| 
	 	 	 \item[ from.theta ] \verb|function(x) 2*exp(x)/(1+exp(x))-1| 
	 	 \end{description}
	 	\item[theta5]\ 
	 	 \begin{description}
	 	 	 \item[ name ] pacf4 
	 	 	 \item[ short.name ] pacf4 
	 	 	 \item[ initial ] 0 
	 	 	 \item[ fixed ] TRUE 
	 	 	 \item[ prior ] none 
	 	 	 \item[ param ]  
	 	 	 \item[ to.theta ] \verb|function(x) log((1+x)/(1-x))| 
	 	 	 \item[ from.theta ] \verb|function(x) 2*exp(x)/(1+exp(x))-1| 
	 	 \end{description}
	 	\item[theta6]\ 
	 	 \begin{description}
	 	 	 \item[ name ] pacf5 
	 	 	 \item[ short.name ] pacf5 
	 	 	 \item[ initial ] 0 
	 	 	 \item[ fixed ] TRUE 
	 	 	 \item[ prior ] none 
	 	 	 \item[ param ]  
	 	 	 \item[ to.theta ] \verb|function(x) log((1+x)/(1-x))| 
	 	 	 \item[ from.theta ] \verb|function(x) 2*exp(x)/(1+exp(x))-1| 
	 	 \end{description}
	 	\item[theta7]\ 
	 	 \begin{description}
	 	 	 \item[ name ] pacf6 
	 	 	 \item[ short.name ] pacf6 
	 	 	 \item[ initial ] 0 
	 	 	 \item[ fixed ] TRUE 
	 	 	 \item[ prior ] none 
	 	 	 \item[ param ]  
	 	 	 \item[ to.theta ] \verb|function(x) log((1+x)/(1-x))| 
	 	 	 \item[ from.theta ] \verb|function(x) 2*exp(x)/(1+exp(x))-1| 
	 	 \end{description}
	 	\item[theta8]\ 
	 	 \begin{description}
	 	 	 \item[ name ] pacf7 
	 	 	 \item[ short.name ] pacf7 
	 	 	 \item[ initial ] 0 
	 	 	 \item[ fixed ] TRUE 
	 	 	 \item[ prior ] none 
	 	 	 \item[ param ]  
	 	 	 \item[ to.theta ] \verb|function(x) log((1+x)/(1-x))| 
	 	 	 \item[ from.theta ] \verb|function(x) 2*exp(x)/(1+exp(x))-1| 
	 	 \end{description}
	 	\item[theta9]\ 
	 	 \begin{description}
	 	 	 \item[ name ] pacf8 
	 	 	 \item[ short.name ] pacf8 
	 	 	 \item[ initial ] 0 
	 	 	 \item[ fixed ] TRUE 
	 	 	 \item[ prior ] none 
	 	 	 \item[ param ]  
	 	 	 \item[ to.theta ] \verb|function(x) log((1+x)/(1-x))| 
	 	 	 \item[ from.theta ] \verb|function(x) 2*exp(x)/(1+exp(x))-1| 
	 	 \end{description}
	 	\item[theta10]\ 
	 	 \begin{description}
	 	 	 \item[ name ] pacf9 
	 	 	 \item[ short.name ] pacf9 
	 	 	 \item[ initial ] 0 
	 	 	 \item[ fixed ] TRUE 
	 	 	 \item[ prior ] none 
	 	 	 \item[ param ]  
	 	 	 \item[ to.theta ] \verb|function(x) log((1+x)/(1-x))| 
	 	 	 \item[ from.theta ] \verb|function(x) 2*exp(x)/(1+exp(x))-1| 
	 	 \end{description}
	 	\item[theta11]\ 
	 	 \begin{description}
	 	 	 \item[ name ] pacf10 
	 	 	 \item[ short.name ] pacf10 
	 	 	 \item[ initial ] 0 
	 	 	 \item[ fixed ] TRUE 
	 	 	 \item[ prior ] none 
	 	 	 \item[ param ]  
	 	 	 \item[ to.theta ] \verb|function(x) log((1+x)/(1-x))| 
	 	 	 \item[ from.theta ] \verb|function(x) 2*exp(x)/(1+exp(x))-1| 
	 	 \end{description}
	 \end{description}
	 \item[ constr ] FALSE 
	 \item[ nrow.ncol ] FALSE 
	 \item[ augmented ] FALSE 
	 \item[ aug.factor ] 1 
	 \item[ aug.constr ]  
	 \item[ n.div.by ]  
	 \item[ n.required ] FALSE 
	 \item[ set.default.values ] FALSE 
	 \item[ pdf ] ar 
\end{description}


\subsection*{Example}

\verbatiminput{example-ar.R}

\subsection*{Notes}

\begin{itemize}
\item The functions \texttt{inla.ar.pacf2phi} and
    \texttt{inla.ar.phi2pacf} converts from the the $\phi$-parameters
    to the $\psi$-parameters, using the Durbin-Levinson
    recursions. These can also be used to compute, the marginal
    posteriors of the $\phi$-parameters from an approximation of the
    joint of the $\phi$-parameters; see the example for a simulation
    based approach.
\item Currently, the order $p$ is limited to $10$. If this creates a
    problem, let us know.
\item If some of the $\psi_{k}$-parameters are fixed, and $k < p$,
    then the marginal (log-)likelihood is wrong; The joint normal
    prior for all the $p$ $\psi$-parameters is used and not the
    conditional normal prior condition on the fixed
    $\psi_{k}$-parameters. If this creates a problem, let us know.
\item The prior spesification for the multivariate normal is a bit
    awkward. Hopefully, we will come up with a better way to do this
    in the future.
\item This model is currently marked as experimental.
\end{itemize}

\end{document}

% LocalWords:  Autoregressive Parametrization autoregressive parameterisation

%%% Local Variables: 
%%% TeX-master: t
%%% End: 
% LocalWords:  parameterized autocorrelation Hyperparameters mvnorm ar
% LocalWords:  Hyperparameter spesification Durbin Levinson
