\documentclass[a4paper,11pt]{article}
\usepackage[scale={0.8,0.9},centering,includeheadfoot]{geometry}
\usepackage{amstext}
\usepackage{listings}
\begin{document}

\section*{Autoregressive model of order $1$ (AR1)}

\subsection*{Parametrization}

The autoregressive model of order $1$ (AR1) for the Gaussian vector $\mathbf{x}=(x_1,\dots,x_n)$ is defined as:
\begin{eqnarray}\nonumber
x_1&\sim&\mathcal{N}(0,(\tau(1-\phi^2))^{-1}) \\\nonumber
x_i&=&\phi\ x_{i-1}+\epsilon_i; \qquad \epsilon_i\sim\mathcal{N}(0,\tau^{-1}) \qquad  i=2,\dots,n
\end{eqnarray}
where
\[
|\phi|<1
\]

\subsection*{Hyperparameters}

The precision parameter $\kappa$ is represented as
\begin{displaymath}
    \theta_1 =\log(\kappa) 
\end{displaymath}
where $\kappa$ is the \emph{marginal} precision,
\begin{displaymath}
    \kappa = \tau (1-\phi^{2}).
\end{displaymath}
The parameter $\phi$ is represented as
\[
\theta_2 = \log\left(\frac{1+\phi}{1-\phi}\right)
\]
and the prior is defined on $\mathbf{\theta}=(\theta_1,\theta_2)$. 

\subsection*{Specification}

The AR1 model is specified inside the {\tt f()} function as
\begin{verbatim}
 f(<whatever>,model="ar1",values=<values>,prior=c(<prior.model.theta1>,<prior.model.theta2>),
              param=c(<param.prior.theta1>,<param.prior.theta1>,
                      <param.prior.theta2>,<param.prior.theta2>))
\end{verbatim}
The (optional) argument {\tt values } is a numeric or factor vector giving the values assumed by the covariate for
 which we want the effect to be estimated. See the example for RW1 for an application. 

\subsection*{Example}

In this exaple we implement a ar1 model where $\theta_1$ has a log-Gamma prior with parameters 1 and 0.001 and $\theta_2$ has a Gaussian prior with parameters 0 and 0.001
\begin{verbatim}

#simulate data
n = 100
phi = 0.8
prec = 10
## note that the marginal precision would be
marg.prec = prec * (1-phi^2)

E=sample(c(5,4,10,12),size=n,replace=T)
eta = as.vector(arima.sim(list(order = c(1,0,0), ar = phi), n = n,sd=sqrt(1/prec)))
y=rpois(n,E*exp(eta))
data = list(y=y,z=1:n)

## fit the model
formula = y~f(z,model="ar1",prior=c("loggamma","gaussian"),param=c(1,0.001,0,0.001))
result = inla(formula,family="poisson", data = data)
\end{verbatim}


\subsection*{Notes}

None

\end{document}


% LocalWords: 

%%% Local Variables: 
%%% TeX-master: t
%%% End: 
