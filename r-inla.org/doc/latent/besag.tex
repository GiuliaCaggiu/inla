\documentclass[a4paper,11pt]{article}
\usepackage[scale={0.8,0.9},centering,includeheadfoot]{geometry}
\usepackage{amstext}
\usepackage{listings}
\begin{document}

\section*{Besag model for spatial effects}

\subsection*{Parametrization}

The besag model for random vector $\mathbf{x}=(x_1,\dots,x_n)$ is defined as
\begin{equation}\label{eq.besag}
    x_i|x_j,i\neq j,\tau\sim\mathcal{N}(\frac{1}{n_i}\sum_{i\sim j}x_j,\frac{1}{n_i\tau})
\end{equation}

where $n_i$ is the number of neighbours of node $i$, $i\sim j$
indicates that the two nodes $i$ and $j$ are neighbours.  


\subsection*{Hyperparameters}

The precision parameter $\tau$ is represented as
\begin{displaymath}
    \theta_{1} =\log \tau
\end{displaymath}
and the prior is defined on $\theta_{1}$. 

\subsection*{Specification}

The besag model is specified inside the {\tt f()} function as
\begin{verbatim}
 f(<whatever>,model="besag",graph.file=<graph file name>, hyper=<hyper>)
\end{verbatim}

The neighbourhood structure of $\mathbf{x}$ is passed to the program
through the {\tt graph.file} argument.  The structure of this file is
described below.

\subsubsection*{Hyperparameter spesification and default values}
\begin{description}
	\item[hyper]\ 
	 \begin{description}
	 	\item[theta]\ 
	 	 \begin{description}
	 	 	\item[name] log precision
	 	 	\item[short.name] prec
	 	 	\item[prior] loggamma
	 	 	\item[param] 1 5e-05
	 	 	\item[initial] 0
	 	 	\item[fixed] TRUE
	 	 	\item[to.theta] \verb|function(x) log(x)|
	 	 	\item[from.theta] \verb|function(x) exp(x)|
	 	 \end{description}
	 \end{description}
\end{description}



\subsubsection*{Structure of the graph file}

We describe the required format for the graph file using a small
example. Let the file {\tt gra.dat}, relative to a small graph of only
5 elements, be
\begin{lstlisting}[basicstyle=\footnotesize]
    5
    1 1 2
    2 2 1 3
    3 3 2 4 5 
    4 1 3
    5 1 3
\end{lstlisting}
Line 1 declares the total number of nodes in the graph (5), then, in
lines 2-6 each node is described. For example, line 4 states that node
3 has 4 neighbours and these are nodes 2, 4 and 5.

The graph file can either have nodes indexed from 1 to $n$, or from 0
to $n-1$. Note that in the latter case, node $i$ seen from R
corresponds to node $i-1$ in the 0-indexed graph.



\subsection*{Example}

For examples of application of this model see the {\tt Bym}, {\tt Munich}, {\tt Zambia} or {\tt Scotland} examples in Volume I.

\subsection*{Notes}

The besag model intrinsic with rankdef 1.

The model is modified accordingly is the graph has more than one
connected components.

\end{document}


% LocalWords: 

%%% Local Variables: 
%%% TeX-master: t
%%% End: 
