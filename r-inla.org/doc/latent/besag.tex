\documentclass[a4paper,11pt]{article}
\usepackage[scale={0.8,0.9},centering,includeheadfoot]{geometry}
\usepackage{amstext}
\usepackage{listings}
\begin{document}

\section*{Besag model for spatial effects}

\subsection*{Parametrization}

The besag model for random vector $\mathbf{x}=(x_1,\dots,x_n)$ is defined as
\begin{equation}\label{eq.besag}
    x_i|x_j,i\neq j,\tau\sim\mathcal{N}(\frac{1}{n_i}\sum_{i\sim j}x_j,\frac{1}{n_i\tau})
\end{equation}

where $n_i$ is the number of neighbours of node $i$, $i\sim j$
indicates that the two nodes $i$ and $j$ are neighbours.  


\subsection*{Hyperparameters}

The precision parameter $\tau$ is represented as
\begin{displaymath}
    \theta_{1} =\log \tau
\end{displaymath}
and the prior is defined on $\theta_{1}$. 

\subsection*{Specification}

The besag model is specified inside the {\tt f()} function as
\begin{verbatim}
 f(<whatever>,model="besag",graph=<graph>,
   hyper=<hyper>, adjust.for.con.comp = TRUE,
   scale.model = FALSE)
\end{verbatim}

The neighbourhood structure of $\mathbf{x}$ is passed to the program
through the {\tt graph} argument.

If the option \verb|adjust.for.con.comp=TRUE| then the model is
adjusted if the graph has more than one connected compoment. This
adjustment can be disabled setting this option to \texttt{FALSE}. If
\verb|adjust.for.con.comp=TRUE| then \texttt{constr=TRUE} is
interpreted as a sum-to-zero constraint on \emph{each} connected
component in the graph and the \texttt{rankdef} parameter is set to
the number of connected components.

The logical option \verb|scale.model| determine if the model should be
scaled to have an average variance (the diagonal of the generalized
inverse) equal to 1. This makes prior spesification much
easier. Default is \verb|FALSE| so that the model is not scaled.


\subsubsection*{Hyperparameter spesification and default values}
\begin{description}
	\item[hyper]\ 
	 \begin{description}
	 	\item[theta]\ 
	 	 \begin{description}
	 	 	\item[name] log precision
	 	 	\item[short.name] prec
	 	 	\item[prior] loggamma
	 	 	\item[param] 1 5e-05
	 	 	\item[initial] 0
	 	 	\item[fixed] TRUE
	 	 	\item[to.theta] \verb|function(x) log(x)|
	 	 	\item[from.theta] \verb|function(x) exp(x)|
	 	 \end{description}
	 \end{description}
\end{description}


\subsection*{Example}

For examples of application of this model see the {\tt Bym}, {\tt Munich}, {\tt Zambia} or {\tt Scotland} examples in Volume I.

\subsection*{Notes}

The besag model intrinsic with rankdef 1.

\end{document}


% LocalWords: 

%%% Local Variables: 
%%% TeX-master: t
%%% End: 
