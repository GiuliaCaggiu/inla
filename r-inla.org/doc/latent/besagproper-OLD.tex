\documentclass[a4paper,11pt]{article}
\usepackage[scale={0.8,0.9},centering,includeheadfoot]{geometry}
\usepackage{amstext}
\usepackage{listings}
\begin{document}

\section*{Proper Besag model for spatial effects}

\subsection*{Parametrization}

The proper version of the Besag model for random vector
$\mathbf{x}=(x_1,\dots,x_n)$ is defined as
\begin{equation}\label{eq.besag}
    x_i|x_{-i},\tau,\phi\quad\sim\quad \mathcal{N}\left(\frac{\phi}{1+\phi
          n_i}\sum_{i\sim j}x_j,\frac{1}{(1 + \phi n_i)\tau}\right)
\end{equation}
where $n_i$ is the number of neighbours of node $i$, $i\sim j$
indicates that the two nodes $i$ and $j$ are neighbours, $\phi > 0$ is
as weight parameter and $\tau > 0$ is a ``precision-like'' (or
scaling) parameter.


\subsection*{Hyperparameters}

The precision parameter $\tau$ is represented as
\begin{displaymath}
    \theta_{1} =\log \tau
\end{displaymath}
and the prior is defined on $\theta_{1}$. 

The weight parameter $\phi$ is represented as
\begin{displaymath}
    \theta_{2} =\log \phi
\end{displaymath}
and the prior is defined on $\theta_{2}$. 

\subsection*{Specification}

The besag model is specified inside the {\tt f()} function as
\begin{verbatim}
 f(<whatever>,model="besagproper",graph.file=<graph file name>,
   hyper=<hyper>)
\end{verbatim}

The neighbourhood structure of $\mathbf{x}$ is passed to the program
through the {\tt graph.file} argument.  The structure of this file is
described below.

\subsubsection*{Hyperparameter spesification and default values}
\documentclass[a4paper,11pt]{article}
\usepackage[scale={0.8,0.9},centering,includeheadfoot]{geometry}
\usepackage{amstext}
\usepackage{listings}
\usepackage{verbatim}
\begin{document}

\section*{Proper/Non-intrinsic Besag model for spatial effects}

\subsection*{Parametrization}

The proper version of the Besag model for random vector
$\mathbf{x}=(x_1,\dots,x_n)$ is defined as
\begin{equation}\label{eq.besag}
    x_i|x_{-i},\tau,d\quad\sim\quad \mathcal{N}\left(\frac{1}{d+
          n_i}\sum_{i\sim j}x_j,\frac{1}{\tau(d + n_i)}\right)
\end{equation}
where $n_i$ is the number of neighbours of node $i$, $i\sim j$
indicates that the two nodes $i$ and $j$ are neighbours, $d > 0$ is an
extra term added on the diagonal controlling the ``properness'' and
$\tau > 0$ is a ``precision-like'' (or scaling) parameter.

This parameterisation corresponds to this precision matrix ${Q} =
(Q_{ij})$, where for $j\not=i$
\begin{displaymath}
    Q_{ii} = \tau(n_{i} + d) \qquad\text{and}\qquad Q_{ij} = -\tau.
\end{displaymath}


\subsection*{Hyperparameters}

The precision parameter $\tau$ is represented as
\begin{displaymath}
    \theta_{1} =\log \tau
\end{displaymath}
and the prior is defined on $\theta_{1}$. The diagonal parameter $d$ is
represented as
\begin{displaymath}
    \theta_{2} =\log d
\end{displaymath}
and the prior is defined on $\theta_{2}$. 

\subsection*{Specification}

The besag model is specified inside the {\tt f()} function as
\begin{verbatim}
 f(<whatever>, model="besagproper", graph=<graph>,
   hyper=<hyper>)
\end{verbatim}

The neighbourhood structure of $\mathbf{x}$ is passed to the program
through the {\tt graph} argument.  The structure of this file is
described below.

\subsubsection*{Hyperparameter spesification and default values}
\documentclass[a4paper,11pt]{article}
\usepackage[scale={0.8,0.9},centering,includeheadfoot]{geometry}
\usepackage{amstext}
\usepackage{listings}
\usepackage{verbatim}
\begin{document}

\section*{Proper/Non-intrinsic Besag model for spatial effects}

\subsection*{Parametrization}

The proper version of the Besag model for random vector
$\mathbf{x}=(x_1,\dots,x_n)$ is defined as
\begin{equation}\label{eq.besag}
    x_i|x_{-i},\tau,d\quad\sim\quad \mathcal{N}\left(\frac{1}{d+
          n_i}\sum_{i\sim j}x_j,\frac{1}{\tau(d + n_i)}\right)
\end{equation}
where $n_i$ is the number of neighbours of node $i$, $i\sim j$
indicates that the two nodes $i$ and $j$ are neighbours, $d > 0$ is an
extra term added on the diagonal controlling the ``properness'' and
$\tau > 0$ is a ``precision-like'' (or scaling) parameter.

This parameterisation corresponds to this precision matrix ${Q} =
(Q_{ij})$, where for $j\not=i$
\begin{displaymath}
    Q_{ii} = \tau(n_{i} + d) \qquad\text{and}\qquad Q_{ij} = -\tau.
\end{displaymath}


\subsection*{Hyperparameters}

The precision parameter $\tau$ is represented as
\begin{displaymath}
    \theta_{1} =\log \tau
\end{displaymath}
and the prior is defined on $\theta_{1}$. The diagonal parameter $d$ is
represented as
\begin{displaymath}
    \theta_{2} =\log d
\end{displaymath}
and the prior is defined on $\theta_{2}$. 

\subsection*{Specification}

The besag model is specified inside the {\tt f()} function as
\begin{verbatim}
 f(<whatever>, model="besagproper", graph=<graph>,
   hyper=<hyper>)
\end{verbatim}

The neighbourhood structure of $\mathbf{x}$ is passed to the program
through the {\tt graph} argument.  The structure of this file is
described below.

\subsubsection*{Hyperparameter spesification and default values}
\documentclass[a4paper,11pt]{article}
\usepackage[scale={0.8,0.9},centering,includeheadfoot]{geometry}
\usepackage{amstext}
\usepackage{listings}
\usepackage{verbatim}
\begin{document}

\section*{Proper/Non-intrinsic Besag model for spatial effects}

\subsection*{Parametrization}

The proper version of the Besag model for random vector
$\mathbf{x}=(x_1,\dots,x_n)$ is defined as
\begin{equation}\label{eq.besag}
    x_i|x_{-i},\tau,d\quad\sim\quad \mathcal{N}\left(\frac{1}{d+
          n_i}\sum_{i\sim j}x_j,\frac{1}{\tau(d + n_i)}\right)
\end{equation}
where $n_i$ is the number of neighbours of node $i$, $i\sim j$
indicates that the two nodes $i$ and $j$ are neighbours, $d > 0$ is an
extra term added on the diagonal controlling the ``properness'' and
$\tau > 0$ is a ``precision-like'' (or scaling) parameter.

This parameterisation corresponds to this precision matrix ${Q} =
(Q_{ij})$, where for $j\not=i$
\begin{displaymath}
    Q_{ii} = \tau(n_{i} + d) \qquad\text{and}\qquad Q_{ij} = -\tau.
\end{displaymath}


\subsection*{Hyperparameters}

The precision parameter $\tau$ is represented as
\begin{displaymath}
    \theta_{1} =\log \tau
\end{displaymath}
and the prior is defined on $\theta_{1}$. The diagonal parameter $d$ is
represented as
\begin{displaymath}
    \theta_{2} =\log d
\end{displaymath}
and the prior is defined on $\theta_{2}$. 

\subsection*{Specification}

The besag model is specified inside the {\tt f()} function as
\begin{verbatim}
 f(<whatever>, model="besagproper", graph=<graph>,
   hyper=<hyper>)
\end{verbatim}

The neighbourhood structure of $\mathbf{x}$ is passed to the program
through the {\tt graph} argument.  The structure of this file is
described below.

\subsubsection*{Hyperparameter spesification and default values}
\input{../hyper/latent/besagproper.tex}


\subsection*{Example}

\verbatiminput{besagproper-example.R}

\subsection*{Notes}

If $d=0$ and the parameter \texttt{rankdef=1} is set, then this model
corresponds to the \texttt{besag} model. \texttt{constr=FALSE} is
default for this model.

\end{document}


% LocalWords: 

%%% Local Variables: 
%%% TeX-master: t
%%% End: 



\subsection*{Example}

\verbatiminput{besagproper-example.R}

\subsection*{Notes}

If $d=0$ and the parameter \texttt{rankdef=1} is set, then this model
corresponds to the \texttt{besag} model. \texttt{constr=FALSE} is
default for this model.

\end{document}


% LocalWords: 

%%% Local Variables: 
%%% TeX-master: t
%%% End: 



\subsection*{Example}

\verbatiminput{besagproper-example.R}

\subsection*{Notes}

If $d=0$ and the parameter \texttt{rankdef=1} is set, then this model
corresponds to the \texttt{besag} model. \texttt{constr=FALSE} is
default for this model.

\end{document}


% LocalWords: 

%%% Local Variables: 
%%% TeX-master: t
%%% End: 



\subsubsection*{Structure of the graph file}

We describe the required format for the graph file using a small
example. Let the file {\tt gra.dat}, relative to a small graph of only
5 elements, be
\begin{lstlisting}[basicstyle=\footnotesize]
    5
    1 1 2
    2 2 1 3
    3 3 2 4 5 
    4 1 3
    5 1 3
\end{lstlisting}
Line 1 declares the total number of nodes in the graph (5), then, in
lines 2-6 each node is described. For example, line 4 states that node
3 has 4 neighbours and these are nodes 2, 4 and 5.

The graph file can either have nodes indexed from 1 to $n$, or from 0
to $n-1$. Note that in the latter case, node $i$ seen from R
corresponds to node $i-1$ in the 0-indexed graph.



\subsection*{Example}

To be added

\subsection*{Notes}

Add notes later ? 

\end{document}


% LocalWords: 

%%% Local Variables: 
%%% TeX-master: t
%%% End: 
