\documentclass[a4paper,11pt]{article}
\usepackage[scale={0.8,0.9},centering,includeheadfoot]{geometry}
\usepackage{amstext}
\usepackage{listings}
\begin{document}

\section*{Bym model for spatial effects}

\subsection*{Parametrization}

This model is simply the sum of a \lstinline$besag$ model and a
\lstinline$iid$ model.
 
The benefite is that this allows to get the posterior marginals of the
sum of the spatial and iid model; otherwise it offers no advantages.

\subsection*{Hyperparameters}
The hyperparameters are the precision $\tau_1$ of the \lstinline$iid$
model and the precision $\tau_2$ of the \lstinline$besag$ model.  The
precision parameters are represented as
\begin{displaymath}
    \theta=(\theta_1,\theta_2) =(\log \tau_1,\log \tau_2)
\end{displaymath}
and the prior is defined on $\mathbf{\theta}$.

\subsection*{Specification}

The bym model is specified inside the {\tt f()} function as
\begin{verbatim}
 f(<whatever>,model="bym",graph.file=<graph file name>
              prior=c(<prior.model.theta1>,<prior.model.theta2>),
              param=c(<param.prior.theta1>,<param.prior.theta1>,<param.prior.theta2>,<param.prior.theta2>,))
\end{verbatim}

The neighbourhood structure of $\mathbf{x}$ is passed to the program
through the {\tt graph.file} argument.  The structure of this file is
described below.

\subsubsection*{Structure of the graph file}

We describe the required format for the graph file using a small
example. Let the file {\tt gra.dat}, relative to a small graph of only
5 elements, be
\begin{lstlisting}[basicstyle=\footnotesize]
    5
    1 1 2
    2 2 1 3
    3 3 2 4 5 
    4 1 3
    5 1 3
\end{lstlisting}
Line 1 declares the total number of nodes in the graph (5), then, in
lines 2-6 each node is described. For example, line 4 states that node
3 has 4 neighbours and these are nodes 2, 4 and 5.

The graph file can either have nodes indexed from 1 to $n$, or from 0
to $n-1$.  Note that in the latter case, node $i$ seen from R
corresponds to node $i-1$ in the 0-indexed graph.

\subsection*{Example}

For examples of application of this model see the {\tt Bym} example in
Volume I.

The model is modified accordingly is the graph has more than one
connected components.

\subsection*{Notes}

None

\end{document}


% LocalWords: 

%%% Local Variables: 
%%% TeX-master: t
%%% End: 
