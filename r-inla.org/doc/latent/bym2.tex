\documentclass[a4paper,11pt]{article}
\usepackage[scale={0.8,0.9},centering,includeheadfoot]{geometry}
\usepackage{amstext}
\usepackage{amsmath}
\usepackage{listings}
\begin{document}

\section*{Bym2 model for spatial effects}

\subsection*{Parametrization}

This model is a reparameterisation of the BYM-model, which is a union
of the \lstinline$besag$ model $u^{*}$ and a \lstinline$iid$ model $v^{*}$, so
that
\begin{displaymath}
    x =
    \begin{pmatrix}
        v^{*} + u^{*}\\
        u^{*}
    \end{pmatrix}
\end{displaymath}
where both $u^{*}$ and $v^{*}$ has a precision (hyper-)parameter.  The
length of $x$ is $2n$ if the length of $u^{*}$ (and $v^{*}$) is
$n$. The BYM2 model uses a different parameterisation of the
hyperparameters where
\begin{displaymath}
    x =
    \begin{pmatrix}
        \frac{1}{\sqrt{\tau}}\left(\sqrt{1-\rho} \;v + \sqrt{\rho} \;u\right)\\
        u
    \end{pmatrix}
\end{displaymath}
where both $u$ and $v$ are \emph{standardised} to have (generalised)
variance equal to one.  The \emph{marginal} precision is then $\tau$
and the proportion of the marginal variance explained by the spatial
effect ($u$) is $\rho$.

\subsection*{Hyperparameters}
The hyperparameters are the margainal precision $\tau$ and the mixing
parameter $\rho$.  The marginal precision $\tau$ is represented as
\begin{displaymath}
    \theta_{1} = \log(\tau)
\end{displaymath}
and the mixing parameter as
\begin{displaymath}
    \theta_{2} = \log\left(\frac{\rho}{1-\rho}\right)
\end{displaymath}
and the prior is defined on $\mathbf{\theta} = (\theta_{1}, \theta_{2})$.

\subsection*{Specification}

The bym2 model is specified inside the {\tt f()} function as
\begin{verbatim}
 f(<whatever>, model="bym2", graph=<graph>,
   hyper=<hyper>, adjust.for.con.comp = TRUE)
\end{verbatim}
The neighbourhood structure of $\mathbf{x}$ is passed to the program
through the {\tt graph} argument.

The option \verb|adjust.for.con.comp| adjust the model if the graph
has more than one connected compoment, and this adjustment can be
disabled setting this option to \texttt{FALSE}. This means that
\texttt{constr=TRUE} is interpreted as a sum-to-zero constraint on
\emph{each} connected component and the \texttt{rankdef} parameter is
set accordingly. 

\subsubsection*{Hyperparameter spesification and default values}
\documentclass[a4paper,11pt]{article}
\usepackage[scale={0.8,0.9},centering,includeheadfoot]{geometry}
\usepackage{amstext}
\usepackage{amsmath}
\usepackage{listings}
\begin{document}

\section*{Bym2 model for spatial effects}

\subsection*{Parametrization}

This model is a reparameterisation of the BYM-model, which is a union
of the \lstinline$besag$ model $u^{*}$ and a \lstinline$iid$ model $v^{*}$, so
that
\begin{displaymath}
    x =
    \begin{pmatrix}
        v^{*} + u^{*}\\
        u^{*}
    \end{pmatrix}
\end{displaymath}
where both $u^{*}$ and $v^{*}$ has a precision (hyper-)parameter.  The
length of $x$ is $2n$ if the length of $u^{*}$ (and $v^{*}$) is
$n$. The BYM2 model uses a different parameterisation of the
hyperparameters where
\begin{displaymath}
    x =
    \begin{pmatrix}
        \frac{1}{\sqrt{\tau}}\left(\sqrt{1-\rho} \;v + \sqrt{\rho} \;u\right)\\
        u
    \end{pmatrix}
\end{displaymath}
where both $u$ and $v$ are \emph{standardised} to have (generalised)
variance equal to one.  The \emph{marginal} precision is then $\tau$
and the proportion of the marginal variance explained by the spatial
effect ($u$) is $\rho$.

\subsection*{Hyperparameters}
The hyperparameters are the margainal precision $\tau$ and the mixing
parameter $\rho$.  The marginal precision $\tau$ is represented as
\begin{displaymath}
    \theta_{1} = \log(\tau)
\end{displaymath}
and the mixing parameter as
\begin{displaymath}
    \theta_{2} = \log\left(\frac{\rho}{1-\rho}\right)
\end{displaymath}
and the prior is defined on $\mathbf{\theta} = (\theta_{1}, \theta_{2})$.

\subsection*{Specification}

The bym2 model is specified inside the {\tt f()} function as
\begin{verbatim}
 f(<whatever>, model="bym2", graph=<graph>,
   hyper=<hyper>, adjust.for.con.comp = TRUE)
\end{verbatim}
The neighbourhood structure of $\mathbf{x}$ is passed to the program
through the {\tt graph} argument.

The option \verb|adjust.for.con.comp| adjust the model if the graph
has more than one connected compoment, and this adjustment can be
disabled setting this option to \texttt{FALSE}. This means that
\texttt{constr=TRUE} is interpreted as a sum-to-zero constraint on
\emph{each} connected component and the \texttt{rankdef} parameter is
set accordingly. 

\subsubsection*{Hyperparameter spesification and default values}
\documentclass[a4paper,11pt]{article}
\usepackage[scale={0.8,0.9},centering,includeheadfoot]{geometry}
\usepackage{amstext}
\usepackage{amsmath}
\usepackage{listings}
\begin{document}

\section*{Bym2 model for spatial effects}

\subsection*{Parametrization}

This model is a reparameterisation of the BYM-model, which is a union
of the \lstinline$besag$ model $u^{*}$ and a \lstinline$iid$ model $v^{*}$, so
that
\begin{displaymath}
    x =
    \begin{pmatrix}
        v^{*} + u^{*}\\
        u^{*}
    \end{pmatrix}
\end{displaymath}
where both $u^{*}$ and $v^{*}$ has a precision (hyper-)parameter.  The
length of $x$ is $2n$ if the length of $u^{*}$ (and $v^{*}$) is
$n$. The BYM2 model uses a different parameterisation of the
hyperparameters where
\begin{displaymath}
    x =
    \begin{pmatrix}
        \frac{1}{\sqrt{\tau}}\left(\sqrt{1-\rho} \;v + \sqrt{\rho} \;u\right)\\
        u
    \end{pmatrix}
\end{displaymath}
where both $u$ and $v$ are \emph{standardised} to have (generalised)
variance equal to one.  The \emph{marginal} precision is then $\tau$
and the proportion of the marginal variance explained by the spatial
effect ($u$) is $\rho$.

\subsection*{Hyperparameters}
The hyperparameters are the margainal precision $\tau$ and the mixing
parameter $\rho$.  The marginal precision $\tau$ is represented as
\begin{displaymath}
    \theta_{1} = \log(\tau)
\end{displaymath}
and the mixing parameter as
\begin{displaymath}
    \theta_{2} = \log\left(\frac{\rho}{1-\rho}\right)
\end{displaymath}
and the prior is defined on $\mathbf{\theta} = (\theta_{1}, \theta_{2})$.

\subsection*{Specification}

The bym2 model is specified inside the {\tt f()} function as
\begin{verbatim}
 f(<whatever>, model="bym2", graph=<graph>,
   hyper=<hyper>, adjust.for.con.comp = TRUE)
\end{verbatim}
The neighbourhood structure of $\mathbf{x}$ is passed to the program
through the {\tt graph} argument.

The option \verb|adjust.for.con.comp| adjust the model if the graph
has more than one connected compoment, and this adjustment can be
disabled setting this option to \texttt{FALSE}. This means that
\texttt{constr=TRUE} is interpreted as a sum-to-zero constraint on
\emph{each} connected component and the \texttt{rankdef} parameter is
set accordingly. 

\subsubsection*{Hyperparameter spesification and default values}
\documentclass[a4paper,11pt]{article}
\usepackage[scale={0.8,0.9},centering,includeheadfoot]{geometry}
\usepackage{amstext}
\usepackage{amsmath}
\usepackage{listings}
\begin{document}

\section*{Bym2 model for spatial effects}

\subsection*{Parametrization}

This model is a reparameterisation of the BYM-model, which is a union
of the \lstinline$besag$ model $u^{*}$ and a \lstinline$iid$ model $v^{*}$, so
that
\begin{displaymath}
    x =
    \begin{pmatrix}
        v^{*} + u^{*}\\
        u^{*}
    \end{pmatrix}
\end{displaymath}
where both $u^{*}$ and $v^{*}$ has a precision (hyper-)parameter.  The
length of $x$ is $2n$ if the length of $u^{*}$ (and $v^{*}$) is
$n$. The BYM2 model uses a different parameterisation of the
hyperparameters where
\begin{displaymath}
    x =
    \begin{pmatrix}
        \frac{1}{\sqrt{\tau}}\left(\sqrt{1-\rho} \;v + \sqrt{\rho} \;u\right)\\
        u
    \end{pmatrix}
\end{displaymath}
where both $u$ and $v$ are \emph{standardised} to have (generalised)
variance equal to one.  The \emph{marginal} precision is then $\tau$
and the proportion of the marginal variance explained by the spatial
effect ($u$) is $\rho$.

\subsection*{Hyperparameters}
The hyperparameters are the margainal precision $\tau$ and the mixing
parameter $\rho$.  The marginal precision $\tau$ is represented as
\begin{displaymath}
    \theta_{1} = \log(\tau)
\end{displaymath}
and the mixing parameter as
\begin{displaymath}
    \theta_{2} = \log\left(\frac{\rho}{1-\rho}\right)
\end{displaymath}
and the prior is defined on $\mathbf{\theta} = (\theta_{1}, \theta_{2})$.

\subsection*{Specification}

The bym2 model is specified inside the {\tt f()} function as
\begin{verbatim}
 f(<whatever>, model="bym2", graph=<graph>,
   hyper=<hyper>, adjust.for.con.comp = TRUE)
\end{verbatim}
The neighbourhood structure of $\mathbf{x}$ is passed to the program
through the {\tt graph} argument.

The option \verb|adjust.for.con.comp| adjust the model if the graph
has more than one connected compoment, and this adjustment can be
disabled setting this option to \texttt{FALSE}. This means that
\texttt{constr=TRUE} is interpreted as a sum-to-zero constraint on
\emph{each} connected component and the \texttt{rankdef} parameter is
set accordingly. 

\subsubsection*{Hyperparameter spesification and default values}
\input{../hyper/latent/bym2.tex}


\subsection*{Example}


\subsection*{Notes}

The term $\frac{1}{2}\log(|R|^{*})$ of the normalisation constant is
not computed, hence you need to add this part to the log marginal
likelihood estimate, if you need it. Here $R$ is the precision matrix
for the standardised Besag part of the model.

\end{document}


% LocalWords: 

%%% Local Variables: 
%%% TeX-master: t
%%% End: 



\subsection*{Example}


\subsection*{Notes}

The term $\frac{1}{2}\log(|R|^{*})$ of the normalisation constant is
not computed, hence you need to add this part to the log marginal
likelihood estimate, if you need it. Here $R$ is the precision matrix
for the standardised Besag part of the model.

\end{document}


% LocalWords: 

%%% Local Variables: 
%%% TeX-master: t
%%% End: 



\subsection*{Example}


\subsection*{Notes}

The term $\frac{1}{2}\log(|R|^{*})$ of the normalisation constant is
not computed, hence you need to add this part to the log marginal
likelihood estimate, if you need it. Here $R$ is the precision matrix
for the standardised Besag part of the model.

\end{document}


% LocalWords: 

%%% Local Variables: 
%%% TeX-master: t
%%% End: 



\subsection*{Example}


\subsection*{Notes}

The term $\frac{1}{2}\log(|R|^{*})$ of the normalisation constant is
not computed, hence you need to add this part to the log marginal
likelihood estimate, if you need it. Here $R$ is the precision matrix
for the standardised Besag part of the model.

\end{document}


% LocalWords: 

%%% Local Variables: 
%%% TeX-master: t
%%% End: 
