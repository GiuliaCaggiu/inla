\documentclass[a4paper,11pt]{article}
\usepackage[scale={0.8,0.9},centering,includeheadfoot]{geometry}
\usepackage{amstext}
\usepackage{listings}
\begin{document}

\section*{Constrained Linear}

\subsection*{Parametrization}

This model is like a ``fixed'' effect where you can constrained the
coefficient of a covariate to be in an interval:
\begin{displaymath}
    \eta_{i} = \beta x_{i}
\end{displaymath}
where $\beta$ is in the interval $[$\texttt{low}, \texttt{high}$]$ and $x$
are the covariates.

\subsection*{Hyperparameters}

The $\beta$ parameter, since its is constrained in general, is a
hyperparamter. The internal transformation depends on the values of
\texttt{low} and \texttt{high}. If \texttt{low} is \texttt{-Inf} and
\texttt{high} is \texttt{Inf}, then
\begin{displaymath}
    \beta = \theta
\end{displaymath}
and the prior is put on $\theta$.  If \texttt{low} is finite and
\texttt{high} is \texttt{Inf}, then
\begin{displaymath}
    \beta = \text{low} + \exp(\theta)
\end{displaymath}
and the prior is put on $\theta$. If \texttt{low} is finite and
\texttt{high} is finite, then
\begin{displaymath}
    \beta = \text{low} + (\text{high}-\text{low})\frac{\exp(\theta)}{1+\exp(\theta)}
\end{displaymath}
and the prior is put on $\theta$. 


\subsection*{Specification}

\begin{verbatim}
    f(x, model="clinear", range = c(low, high), precision = <precision>)
\end{verbatim}
where \texttt{precision} is the precision for the tiny noise used to
implement this as a latent model. 

\subsubsection*{Hyperparameter spesification and default values}
%% DO NOT EDIT!
%% This file is generated automatically from models.R
\begin{description}
	\item[hyper]\ 
	 \begin{description}
	 	\item[theta]\ 
	 	 \begin{description}
	 	 	\item[name] beta
	 	 	\item[short.name] b
	 	 	\item[initial] 1
	 	 	\item[fixed] FALSE
	 	 	\item[prior] normal
	 	 	\item[param] 1 10
	 	 	\item[to.theta] \verb!function(x, low = -Inf, high = Inf) {! \verb!                                         if (all(is.infinite(c(low, high))) || low == high) {! \verb!                                             stopifnot(low < high)! \verb!                                             return (x)! \verb!                                         } else if (all(is.finite(c(low, high)))) {! \verb!                                             stopifnot(low < high)! \verb!                                             return (log( - (low - x)/(high -x)))! \verb!                                         } else if (is.finite(low) && is.infinite(high) && high > low) {! \verb!                                             return (log(x-low))! \verb!                                         } else {! \verb!                                             stop("Condition not yet implemented")! \verb!                                         }! \verb!                                     }!
	 	 	\item[from.theta] \verb!function(x, low = -Inf, high = Inf) {! \verb!                                         if (all(is.infinite(c(low, high))) || low == high) {! \verb!                                             stopifnot(low < high)! \verb!                                             return (x)! \verb!                                         } else if (all(is.finite(c(low, high)))) {! \verb!                                             stopifnot(low < high)! \verb!                                             return (low + exp(x)/(1+exp(x)) * (high - low))! \verb!                                         } else if (is.finite(low) && is.infinite(high) && high > low) {! \verb!                                             return (low + exp(x))! \verb!                                         } else {! \verb!                                             stop("Condition not yet implemented")! \verb!                                         }! \verb!                                     }!
	 	 \end{description}
	 \end{description}
	\item[constr] FALSE
	\item[nrow.ncol] FALSE
	\item[augmented] FALSE
	\item[aug.factor] 1
	\item[aug.constr] 
	\item[n.div.by] 
	\item[n.required] FALSE
	\item[set.default.values] FALSE
	\item[pdf] clinear
\end{description}


\subsection*{Example}

\begin{verbatim}
n = 100
x = runif(n)
y = 1 + x + rnorm(n)
r = inla(y ~ f(x, model = "clinear", range = c(0, Inf)),
         data = data.frame(y,x))
summary(r)
\end{verbatim}

\subsection*{Notes}
None

\end{document}

% LocalWords: 

%%% Local Variables: 
%%% TeX-master: t
%%% End: 
