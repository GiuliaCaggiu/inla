\documentclass[a4paper,11pt]{article}
\usepackage[compat2]{geometry}
\usepackage{amstext}
\usepackage{listings}
\begin{document}

\section*{Random walk model of order $1$ (RW1)}

\subsection*{Parametrization}

The random walk model of order $1$ (RW1) for the Gaussian vector $\mathbf{x}=(x_1,\dots,x_n)$ is constructed assuming independent increments:
\[
\Delta x_i = x_i-x_{i+1}\sim\mathcal{N}(0,\tau^{-1})
\]
The density for $\mathbf{x}$ is derived from its $n-1$ increments as
\begin{eqnarray}
\pi(\mathbf{x}|\tau) &\propto& \tau^{(n-1)/2} \exp\left\{-\frac{\tau}{2}\sum (\Delta x_i)^2\right\}\\
& = &\tau^{(n-1)/2}\exp\left\{-\frac{1}{2}\mathbf{x}^T\mathbf{Q}\mathbf{x} \right\}
\end{eqnarray}
where $\mathbf{Q}=\tau\mathbf{R}$ and $\mathbf{R}$ is the structure matrix reflecting the neighbourhood structure of the model.

It is also possible to define a {\it cyclic} version of the RW1 model, in this case the graph is modified so that last node $x_n$ 
is neighbour of $x_{n-1}$ and $x_1$. 
\subsection*{Hyperparameters}

The precision parameter $\tau$ is represented as
\begin{displaymath}
    \theta =\log \tau
\end{displaymath}
and the prior is defined on $\mathbf{\theta}$. 

\subsection*{Specification}

The RW1 model is specified inside the {\tt f()} function as
\begin{verbatim}
 f(<whatever>,model="rw1",values=<values>,cyclic=<TRUE,FALSE>,
              prior=c(<prior.model.theta>),
              param=c(<param.prior.theta1>))
\end{verbatim}
The (optional) argument {\tt values } is a numeric or factor vector giving the values assumed by the covariate for
 which we want the effect to be estimated. See next example for an application. 
 
\subsection*{Example}

\begin{verbatim}

n=100
z=seq(0,6,length.out=n)
y=sin(z)+rnorm(n,mean=0,sd=0.5)
data=data.frame(y=y,z=z)

formula=y~f(z,model="rw1",prior="loggamma",param=c(1,0.01))
result=inla(formula,data=data,family="gaussian")

#here we estimate the effect only for some of the values in z
formula1=y~f(z,model="rw1",prior="loggamma",param=c(1,0.01),values=z[seq(1,length(z),2)])
result1=inla(formula1,data=data,family="gaussian")

\end{verbatim}


\subsection*{Notes}

The RW1 is a intrinsic random field with rank deficiency of 1.

There exist also support to define irregular RW1 models. 
\end{document}


% LocalWords: 

%%% Local Variables: 
%%% TeX-master: t
%%% End: 
