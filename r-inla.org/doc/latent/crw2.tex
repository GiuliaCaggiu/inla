\documentclass[a4paper,11pt]{article}
\usepackage[scale={0.8,0.9},centering,includeheadfoot]{geometry}
\usepackage{amstext}
\usepackage{listings}
\begin{document}

\section*{Continous random walk model of order $2$ (CRW2)}

\subsection*{Parametrization}

The continous random walk model of order $2$ (CRW2) for the Gaussian
vector $\mathbf{x}=(x_1,\dots,x_n)$ is descibed in the GMRF-book
chapter 3. It is an exact representation of the continous CRW2 model
augmented with its derivaties. The use its the same as for RW2. 

\subsection*{Hyperparameters}

The precision parameter $\tau$ is represented as
\begin{displaymath}
    \theta =\log \tau
\end{displaymath}
and the prior is defined on $\mathbf{\theta}$. Note that $\tau$ is the
precision for the first order increments.

\subsection*{Specification}

The CRW2 model is specified inside the {\tt f()} function as
\begin{verbatim}
 f(<whatever>, model="crw2", values=<values>, hyper = <hyper>)
\end{verbatim}
The (optional) argument {\tt values } is a numeric or factor vector
giving the values assumed by the covariate for which we want the
effect to be estimated. See next example for an application.
 
\subsubsection*{Hyperparameter spesification and default values}
\begin{description}
	\item[hyper]\ 
	 \begin{description}
	 	\item[theta]\ 
	 	 \begin{description}
	 	 	 \item[ name ] log precision 
	 	 	 \item[ short.name ] prec 
	 	 	 \item[ prior ] loggamma 
	 	 	 \item[ param ] 1 5e-05 
	 	 	 \item[ initial ] 4 
	 	 	 \item[ fixed ] FALSE 
	 	 	 \item[ to.theta ] \verb|function(x) log(x)| 
	 	 	 \item[ from.theta ] \verb|function(x) exp(x)| 
	 	 \end{description}
	 \end{description}
	 \item[ constr ] TRUE 
	 \item[ nrow.ncol ] FALSE 
	 \item[ augmented ] FALSE 
	 \item[ aug.factor ] 2 
	 \item[ aug.constr ] 1 
	 \item[ n.div.by ]  
	 \item[ n.required ] FALSE 
	 \item[ set.default.values ] FALSE 
	 \item[ pdf ] crw2 
\end{description}



\subsection*{Example}

\begin{verbatim}
n=100
z=seq(0,6,length.out=n)
y=sin(z)+rnorm(n,mean=0,sd=0.5)
data=data.frame(y=y,z=z)

formula=y~f(z,model="crw2")
result=inla(formula,data=data,family="gaussian")
\end{verbatim}


\subsection*{Notes}

The CRW2 model is intrinsic with rank deficiency of 2.

The model RW2 is an good (enough) approximation to CRW2 and do not
augment with the derivaties.
\end{document}


% LocalWords: 

%%% Local Variables: 
%%% TeX-master: t
%%% End: 
