\documentclass[a4paper,11pt]{article}
\usepackage[scale={0.8,0.9},centering,includeheadfoot]{geometry}
\usepackage{amstext}
\usepackage{listings}
\begin{document}

\section*{Generic0 model}

\subsection*{Parametrization}

The Type 0 generic model implements the following precision matrix
\begin{displaymath}
    \mathbf{Q}=\tau\mathbf{C}
\end{displaymath}
where $\mathbf{C}$ is the structure matrix.

\subsection*{Hyperparameters}

The precision parameters of the generic0 model is represented as
\begin{displaymath}
    \theta = \log(\tau)
\end{displaymath}
and prior is assigned to $\theta$

\subsection*{Specification}

The generic0 models is specified inside the {\tt f()} function as
\begin{verbatim}
 f(<whatever>,model="generic0",Cmatrix = <Cmat>, hyper = <hyper>)
\end{verbatim}


where {\tt <Cmat>} can be given in two different ways:
\begin{itemize}
\item a dense matrix or a sparse-matrix defined be
    \texttt{Matrix::sparseMatrix()}.
\item the name of a file giving the structure matrix. The file should
    have the following format
    \[
    i\quad j\quad \mathbf{C}_{ij}
    \]
    where $i$ and $j$ are the row and column index and
    $\mathbf{C}_{ij}$ is the corresponding element of the precision
    matrix. Only the non-zero elements of the precision matrix need to
    be stored in the file.  
\end{itemize}
See the following example for an application

\subsubsection*{Hyperparameter spesification and default values}
\documentclass[a4paper,11pt]{article}
\usepackage[scale={0.8,0.9},centering,includeheadfoot]{geometry}
\usepackage{amstext}
\usepackage{listings}
\begin{document}

\section*{Generic 0 model}

\subsection*{Parametrization}

The Type 0 generic model implements the following precision matrix
\begin{displaymath}
    \mathbf{Q}=\tau\mathbf{C}
\end{displaymath}
where $\mathbf{C}$ is the structure matrix.

\subsection*{Hyperparameters}

The precision parameters of the generic0 model is represented as
\begin{displaymath}
    \theta = \log(\tau)
\end{displaymath}
and prior is assigned to $\theta$

\subsection*{Specification}

The generic0 models is specified inside the {\tt f()} function as
\begin{verbatim}
 f(<whatever>,model="generic0",Cmatrix = <Cmat>, hyper = <hyper>)
\end{verbatim}


where {\tt <Cmat>} can be given in two different ways:
\begin{itemize}
\item a dense matrix or a sparse-matrix defined be
    \texttt{Matrix::sparseMatrix()}.
\item the name of a file giving the structure matrix. The file should
    have the following format
    \[
    i\quad j\quad \mathbf{C}_{ij}
    \]
    where $i$ and $j$ are the row and column index and
    $\mathbf{C}_{ij}$ is the corresponding element of the precision
    matrix. Only the non-zero elements of the precision matrix need to
    be stored in the file.  
\end{itemize}
See the following example for an application

\subsubsection*{Hyperparameter spesification and defaults}
\documentclass[a4paper,11pt]{article}
\usepackage[scale={0.8,0.9},centering,includeheadfoot]{geometry}
\usepackage{amstext}
\usepackage{listings}
\begin{document}

\section*{Generic 0 model}

\subsection*{Parametrization}

The Type 0 generic model implements the following precision matrix
\begin{displaymath}
    \mathbf{Q}=\tau\mathbf{C}
\end{displaymath}
where $\mathbf{C}$ is the structure matrix.

\subsection*{Hyperparameters}

The precision parameters of the generic0 model is represented as
\begin{displaymath}
    \theta = \log(\tau)
\end{displaymath}
and prior is assigned to $\theta$

\subsection*{Specification}

The generic0 models is specified inside the {\tt f()} function as
\begin{verbatim}
 f(<whatever>,model="generic0",Cmatrix = <Cmat>, hyper = <hyper>)
\end{verbatim}


where {\tt <Cmat>} can be given in two different ways:
\begin{itemize}
\item a dense matrix or a sparse-matrix defined be
    \texttt{Matrix::sparseMatrix()}.
\item the name of a file giving the structure matrix. The file should
    have the following format
    \[
    i\quad j\quad \mathbf{C}_{ij}
    \]
    where $i$ and $j$ are the row and column index and
    $\mathbf{C}_{ij}$ is the corresponding element of the precision
    matrix. Only the non-zero elements of the precision matrix need to
    be stored in the file.  
\end{itemize}
See the following example for an application

\subsubsection*{Hyperparameter spesification and defaults}
\documentclass[a4paper,11pt]{article}
\usepackage[scale={0.8,0.9},centering,includeheadfoot]{geometry}
\usepackage{amstext}
\usepackage{listings}
\begin{document}

\section*{Generic 0 model}

\subsection*{Parametrization}

The Type 0 generic model implements the following precision matrix
\begin{displaymath}
    \mathbf{Q}=\tau\mathbf{C}
\end{displaymath}
where $\mathbf{C}$ is the structure matrix.

\subsection*{Hyperparameters}

The precision parameters of the generic0 model is represented as
\begin{displaymath}
    \theta = \log(\tau)
\end{displaymath}
and prior is assigned to $\theta$

\subsection*{Specification}

The generic0 models is specified inside the {\tt f()} function as
\begin{verbatim}
 f(<whatever>,model="generic0",Cmatrix = <Cmat>, hyper = <hyper>)
\end{verbatim}


where {\tt <Cmat>} can be given in two different ways:
\begin{itemize}
\item a dense matrix or a sparse-matrix defined be
    \texttt{Matrix::sparseMatrix()}.
\item the name of a file giving the structure matrix. The file should
    have the following format
    \[
    i\quad j\quad \mathbf{C}_{ij}
    \]
    where $i$ and $j$ are the row and column index and
    $\mathbf{C}_{ij}$ is the corresponding element of the precision
    matrix. Only the non-zero elements of the precision matrix need to
    be stored in the file.  
\end{itemize}
See the following example for an application

\subsubsection*{Hyperparameter spesification and defaults}
\input{../hyper/latent/generic0.tex}



\subsection*{Example}
In the example below we define a RW1 model first using the {\tt
    generic0} model and this using the {\tt rw1} model.
\begin{verbatim}
## Simulate data
n=100
z=1:n
y=sin(z/n*2*pi)+rnorm(n,mean=0,sd=0.5)
data=data.frame(y=y,z=z)

Q = toeplitz(c(2,-1, rep(0,n-3),-1))
Q[1,1] = Q[n,n] = 1
Q[n,1] = Q[1,n] = 0

## Q as dense
formula1 = y ~ f(z, model="generic0", Cmatrix = Q,
                 rankdef=1, constr=TRUE, diagonal=1e-05)
result1 = inla(formula1, data=data, family="gaussian")

## Q as sparse
Q.sparse = as(Q, "dgTMatrix")
formula2 = y ~ f(z, model="generic0", Cmatrix = Q.sparse,
                 rankdef=1, constr=TRUE, diagonal=1e-05)
result2 = inla(formula2, data=data, family="gaussian")

## This is the same model defined using the rw1 model
formula3 = y ~ f(z,model="rw1")
result3 = inla(formula3, data=data, family="gaussian")

\end{verbatim}

\subsection*{Notes}
None

\end{document}


% LocalWords: 

%%% Local Variables: 
%%% TeX-master: t
%%% End: 




\subsection*{Example}
In the example below we define a RW1 model first using the {\tt
    generic0} model and this using the {\tt rw1} model.
\begin{verbatim}
## Simulate data
n=100
z=1:n
y=sin(z/n*2*pi)+rnorm(n,mean=0,sd=0.5)
data=data.frame(y=y,z=z)

Q = toeplitz(c(2,-1, rep(0,n-3),-1))
Q[1,1] = Q[n,n] = 1
Q[n,1] = Q[1,n] = 0

## Q as dense
formula1 = y ~ f(z, model="generic0", Cmatrix = Q,
                 rankdef=1, constr=TRUE, diagonal=1e-05)
result1 = inla(formula1, data=data, family="gaussian")

## Q as sparse
Q.sparse = as(Q, "dgTMatrix")
formula2 = y ~ f(z, model="generic0", Cmatrix = Q.sparse,
                 rankdef=1, constr=TRUE, diagonal=1e-05)
result2 = inla(formula2, data=data, family="gaussian")

## This is the same model defined using the rw1 model
formula3 = y ~ f(z,model="rw1")
result3 = inla(formula3, data=data, family="gaussian")

\end{verbatim}

\subsection*{Notes}
None

\end{document}


% LocalWords: 

%%% Local Variables: 
%%% TeX-master: t
%%% End: 




\subsection*{Example}
In the example below we define a RW1 model first using the {\tt
    generic0} model and this using the {\tt rw1} model.
\begin{verbatim}
## Simulate data
n=100
z=1:n
y=sin(z/n*2*pi)+rnorm(n,mean=0,sd=0.5)
data=data.frame(y=y,z=z)

Q = toeplitz(c(2,-1, rep(0,n-3),-1))
Q[1,1] = Q[n,n] = 1
Q[n,1] = Q[1,n] = 0

## Q as dense
formula1 = y ~ f(z, model="generic0", Cmatrix = Q,
                 rankdef=1, constr=TRUE, diagonal=1e-05)
result1 = inla(formula1, data=data, family="gaussian")

## Q as sparse
Q.sparse = as(Q, "dgTMatrix")
formula2 = y ~ f(z, model="generic0", Cmatrix = Q.sparse,
                 rankdef=1, constr=TRUE, diagonal=1e-05)
result2 = inla(formula2, data=data, family="gaussian")

## This is the same model defined using the rw1 model
formula3 = y ~ f(z,model="rw1")
result3 = inla(formula3, data=data, family="gaussian")

\end{verbatim}

\subsection*{Notes}
None

\end{document}


% LocalWords: 

%%% Local Variables: 
%%% TeX-master: t
%%% End: 




\subsection*{Example}
In the example below we define a RW1 model first using the {\tt
    generic0} model and this using the {\tt rw1} model.
\begin{verbatim}
## Simulate data
n=100
z=1:n
y=sin(z/n*2*pi)+rnorm(n,mean=0,sd=0.5)
data=data.frame(y=y,z=z)

Q = toeplitz(c(2,-1, rep(0,n-3),-1))
Q[1,1] = Q[n,n] = 1
Q[n,1] = Q[1,n] = 0

## Q as dense
formula1 = y ~ f(z, model="generic0", Cmatrix = Q,
                 rankdef=1, constr=TRUE, diagonal=1e-05)
result1 = inla(formula1, data=data, family="gaussian")

## Q as sparse
Q.sparse = as(Q, "dgTMatrix")
formula2 = y ~ f(z, model="generic0", Cmatrix = Q.sparse,
                 rankdef=1, constr=TRUE, diagonal=1e-05)
result2 = inla(formula2, data=data, family="gaussian")

## This is the same model defined using the rw1 model
formula3 = y ~ f(z,model="rw1")
result3 = inla(formula3, data=data, family="gaussian")
\end{verbatim}

\subsection*{Notes}
None

\end{document}


% LocalWords: 

%%% Local Variables: 
%%% TeX-master: t
%%% End: 
