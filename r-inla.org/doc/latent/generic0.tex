\documentclass[a4paper,11pt]{article}
\usepackage[compat2]{geometry}
\usepackage{amstext}
\usepackage{listings}
\begin{document}

\section*{Generic 0 model}

\subsection*{Parametrization}

The Type 0 generic model implements the following precision matrix
\begin{displaymath}
    \mathbf{Q}=\tau\mathbf{C}
\end{displaymath}
where $\mathbf{C}$ is the structure matrix.

\subsection*{Hyperparameters}

The precision parameters of the generic0 model is represented as
\begin{displaymath}
    \theta = \log(\tau)
\end{displaymath}
and prior is assigned to $\theta$

\subsection*{Specification}

The generic0 models is specified inside the {\tt f()} function as
\begin{verbatim}
 f(<whatever>,model="generic0",Cmatrix = <Cmat>,
              prior=c(<prior.model.theta>),
              param=c(<param.prior.theta1>))
\end{verbatim}


where {\tt <Cmat>} can be given in two different ways:
\begin{itemize}
\item a list of type {\tt Cmatrix = list(i = c(), j = c(), Cij =
        c())}, where {\tt i}, {\tt j} and {\tt Cij} are vectors of the
    non-zero elements of $\mathbf{C}$. Note that {\tt i} and {\tt j}
    start from 1, and only the upper or lower part of $\mathbf{C}$ has
    to be given.
\item the name of a file giving the structure matrix. The file should
    have the following format
    \[
    i\quad j\quad \mathbf{C}_{ij}
    \]
    where $i$ and $j$ are the row and column index and
    $\mathbf{C}_{ij}$ is the corresponding element of the precision
    matrix. Only the non-zero elements of the precision matrix need to
    be stored in the file.  {\bf NOTE:} the indexes for $i$ and $j$
    must start from 0, as this matrix is passed directly into the
    \texttt{inla}-program.
\end{itemize}
See the following example for an application

\subsection*{Example}
In the example below we define a RW1 model first using the {\tt
    generic0} model and this using the {\tt rw1} model.
\begin{verbatim}

#simulate data
n=50
z=seq(0,6,length.out=n)
y=sin(z)+rnorm(n,mean=0,sd=0.5)
data=data.frame(y=y,z=z)

#specify Cmatrix
i=c(1:n,1:(n-1))
j=c(1:n,2:n)
Cij=c(1,rep(2,n-2),1,rep(-1,n-1))

#pass the C matrix as a list

#note that for the generic0 model constraints and diagonal have to be defined by the user
formula= y~f(z,model="generic0",Cmatrix=list(i = i, j = j, Cij = Cij),
             rankdef=1,constr=TRUE,diagonal=1e-05)
result=inla(formula,data=data,family="gaussian")

#pass the C matrix as a file
file.name="Cmatrix.dat"
write.table(cbind(i-1,j-1,Cij),file=file.name,col.name=F,row.name=F)
formula1= y~f(z,model="generic0",Cmatrix=file.name,rankdef=1,rankdef=1,constr=TRUE,diagonal=1e-05)
result1=inla(formula1,data=data,family="gaussian")

#this is the same model defined using the rw1 model
formula2=y~f(z,model="rw1")
result2=inla(formula2,data=data,family="gaussian")

\end{verbatim}

\subsection*{Notes}
None

\end{document}


% LocalWords: 

%%% Local Variables: 
%%% TeX-master: t
%%% End: 
