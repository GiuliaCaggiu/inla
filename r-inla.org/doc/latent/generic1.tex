\documentclass[a4paper,11pt]{article}
\usepackage[scale={0.8,0.9},centering,includeheadfoot]{geometry}
\usepackage{amstext}
\usepackage{listings}
\begin{document}

\section*{Generic 1 model}

\subsection*{Parametrization}

The Type 1 generic model implements the following precision matrix
\[
\mathbf{Q}=\tau(\mathbf{I}-\frac{\beta}{\lambda_{max}}\mathbf{C})
\]
where $\mathbf{C}$ is the structure matrix. The parameter
$\lambda_{max}$ is the maximum eigenvalue of $\mathbf{C}$, which
allows $\beta$ to be in the range $\beta\in[0,1)$
\subsection*{Hyperparameters}

The two parameters of the generic1 model are represented as
\begin{eqnarray*}
    \theta_1 &= & \log(\tau)\\
    \theta_2 &=&\text{logit}(\beta)
\end{eqnarray*}
and priors are assigned to $(\theta_1,\theta_2)$

\subsection*{Specification}

The generic1 model is specified inside the {\tt f()} function as
\begin{verbatim}
 f(<whatever>,model="generic1",Cmatrix = <Cmat>, hyper = <hyper>)
\end{verbatim}
where {\tt <Cmat>} can be given in two different ways:
\begin{itemize}
\item a dense matrix or a sparse-matrix defined be
    \texttt{Matrix::sparseMatrix()}.
\item the name of a file giving the structure matrix. The file should
    have the following format
    \[
    i\quad j\quad \mathbf{C}_{ij}
    \]
    where $i$ and $j$ are the row and column index and
    $\mathbf{C}_{ij}$ is the corresponding element of the precision
    matrix. Only the non-zero elements of the precision matrix need to
    be stored in the file.
\end{itemize}

\subsubsection*{Hyperparameter spesification and default values}
\documentclass[a4paper,11pt]{article}
\usepackage[compat2]{geometry}
\usepackage{amstext}
\usepackage{listings}
\begin{document}

\section*{Generic 1 model}

\subsection*{Parametrization}

The Type 1 generic model implements the following precision matrix
\[
\mathbf{Q}=\tau(\mathbf{I}-\frac{\beta}{\lambda_{max}}\mathbf{C})
\]
where $\mathbf{C}$ is the structure matrix. The parameter
$\lambda_{max}$ is the maximum eigenvalue of $\mathbf{C}$, which
allows $\beta$ to be in the range $\beta\in[0,1)$
\subsection*{Hyperparameters}

The two parameters of the generic1 model are represented as
\begin{eqnarray*}
    \theta_1 &= & \log(\tau)\\
    \theta_2 &=&\text{logit}(\beta)
\end{eqnarray*}
and priors are assigned to $(\theta_1,\theta_2)$

\subsection*{Specification}

The generic1 model is specified inside the {\tt f()} function as
\begin{verbatim}
 f(<whatever>,model="generic1",Cmatrix = <Cmat>,
              prior=c(<prior.model.theta1>,<prior.model.theta2>),
              param=c(<param.prior.theta1>,<param.prior.theta1>,
                      <param.prior.theta2>,<param.prior.theta2>))
\end{verbatim}
where {\tt <Cmat>} can be given in two different ways:
\begin{itemize}
\item a list of type {\tt Cmatrix = list(i = c(), j = c(), Cij =
        c())}, where {\tt i}, {\tt j} and {\tt Cij} are vectors of the
    non-zero elements of $\mathbf{C}$. Note that {\tt i} and {\tt j}
    start from 1, and only the upper or lower part of $\mathbf{C}$ has
    to be given.
\item the name of a file giving the structure matrix. The file should
    have the following format
    \[
    i\quad j\quad \mathbf{C}_{ij}
    \]
    where $i$ and $j$ are the row and column index and
    $\mathbf{C}_{ij}$ is the corresponding element of the precision
    matrix. Only the non-zero elements of the precision matrix need to
    be stored in the file. {\bf NOTE:} the indexes for $i$ and $j$
    start from 0, as this matrix is passed directly into the
    \texttt{inla}-program.
\end{itemize}

\subsection*{Example}
\begin{verbatim}

n = 10

#build a structure matrix
Cm = matrix(runif(n^2,min=-1,max=1),n,n)
diag(Cm) = 0
Cm = 0.5*(Cm + t(Cm))
lambda.max = max(eigen(Cm)$values)

#define the precision matrix
beta = 0.5
Q = diag(rep(1,n)) - beta/lambda.max * Cm
Sigma = solve(Q)

#simulate data
require(mvtnorm)
sd = 0.001
z = rnorm(n)
eta = rmvnorm(n=1,sigma = Sigma)
y = c(eta) + sd*rnorm(n) + z

#print the file containing the C matrix
file = "Cmatrix.dat"
cat("",file=file, append = FALSE)
for(i in 1:n)
{
    j = i
    cat(i-1,j-1,Cm[i,j], "\n", sep = " ", file=file, append=TRUE)
    if (i < n)
    {
        for(j in (i+1):n)
        {
            cat(i-1,j-1,Cm[i,j], "\n", sep = " ", file=file, append=TRUE)
        }
    }
}


idx = 1:n
formula = y ~ f(idx, model = "generic1", Cmatrix = file, initial=c(0,0),fixed=c(F,F) ) + z

d = list(y=y,idx=idx,z=z)
result = inla(formula, data=d,family="gaussian",
              control.data = list(initial = log(1/sd^2), fixed=TRUE))



\end{verbatim}

\subsection*{Notes}
None

\end{document}


% LocalWords: 

%%% Local Variables: 
%%% TeX-master: t
%%% End: 



\subsection*{Example}
\begin{verbatim}
n = 100
## build a structure matrix
Cm = matrix(runif(n^2,min=-1,max=1),n,n)
diag(Cm) = 0
Cm = 0.5*(Cm + t(Cm))
lambda.max = max(eigen(Cm)$values)

## define the precision matrix
beta = 0.9
Q = diag(rep(1,n)) - beta/lambda.max * Cm
Sigma = solve(Q)

#simulate data
require(mvtnorm)
sd = 0.001
z = rnorm(n)
eta = rmvnorm(n=1,sigma = Sigma)
y = c(eta) + sd*rnorm(n) + z
idx = 1:n
d = list(y=y,idx=idx,z=z)

## Alternative 1
## print the file containing the C matrix
file = "Cmatrix.dat"
cat("",file=file, append = FALSE)
for(i in 1:n)
{
    j = i
    cat(i, j, Cm[i,j], "\n", sep = " ", file=file, append=TRUE)
    if (i < n)
        for(j in (i+1):n)
            cat(i, j, Cm[i,j], "\n", sep = " ", file=file, append=TRUE)
}
formula = y ~ f(idx, model = "generic1", Cmatrix = file) + z

## Alternative 2
## formula = y ~ f(idx, model = "generic1", Cmatrix = Cm) + z

## Alternative 3
## formula = y ~ f(idx, model = "generic1", Cmatrix = as(Cm, "dgTMatrix"))+z

###############################################################

result = inla(formula, data=d,family="gaussian",
              control.data = list(initial = log(1/sd^2), fixed=TRUE),
              verbose=T, keep=T)
\end{verbatim}

\subsection*{Notes}
None

\end{document}


% LocalWords: 

%%% Local Variables: 
%%% TeX-master: t
%%% End: 
