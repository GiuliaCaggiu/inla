\documentclass[a4paper,11pt]{article}
\usepackage[scale={0.8,0.9},centering,includeheadfoot]{geometry}
\usepackage{amstext}
\usepackage{listings}
\usepackage{verbatim}
\begin{document}

\section*{The Berkson Measurement Error (MEB) model}

\subsection*{Parametrization}

This is an implementation of the Berksom measurement error model for a
fixed effect. The observed covariate is $w$ but it is $x$
that goes into the linear predictor
\begin{displaymath}
\eta = \ldots + \beta x + \ldots \ ,
\end{displaymath}
where $x = w + u$. The error term $u$ is Gaussian with prior $\mathcal{N}(0, \tau_u\mathbf{D})$\footnote{Note:
The second argument in ${\mathcal N}(,)$ is the precision not the
variance.}, where $\tau_u$ is the observational precision of the error $\text{Prec}(u|x)$) with possible heteroscedasticy, encoded in the entries $d_i$ of the diagonal matrix $\mathbf{D}$. The vector $s$ contains the fixed scalings $s=(d_1,\ldots,d_n)$ (with $n$ the number of data points).



\subsection*{Hyperparameters}

This model has 2 hyperparameters, $\theta = (\theta_{1}, \theta_{2})$.
The hyperparameter specification is as follows:
\begin{displaymath}
\theta_{1} = \beta
\end{displaymath}
and the prior is defined on $\theta_{1}$,
\begin{displaymath}
\theta_{2} = \log(\tau_u)
\end{displaymath}
and the prior is defined on $\theta_{2}$.

\subsection*{Specification}

The MEB is specified inside the {\tt f()}
function as
\begin{verbatim}
f(w, [<weights>,] model="meb", scale = <s>, values= <w>, hyper = <hyper>)
\end{verbatim}
Here, \texttt{w} are the observed covariates, and the fixed scaling of
the observational precision is given in argument \texttt{scale}. If
the argument \texttt{scale} is not given, then $s$ is set to $1$.

Note that only the unique values of \texttt{w} are used, so if two or
more elements of \texttt{w} are \emph{identical}, then they refer to
the \emph{same} element in the covariate $x$. If data points with identical $w$ values belong to different $x$ values (e.g., different individuals), please add a \emph{tiny} random value to $w$ to make this difference obvious to the model.


\subsubsection*{Hyperparameter specification and default values}
\documentclass[a4paper,11pt]{article}
\usepackage[scale={0.8,0.9},centering,includeheadfoot]{geometry}
\usepackage{amstext}
\usepackage{listings}
\usepackage{verbatim}
\begin{document}

\section*{The Berkson Measurement Error (MEB) model}

\subsection*{Parametrization}

This is an implementation of the Berksom measurement error model for a
fixed effect. The observed covariate is $w$ but it is $x$
that goes into the linear predictor
\begin{displaymath}
    \eta = \ldots + \beta x + \ldots \ ,
\end{displaymath}
where $x = w + u$, $u$ is Gaussian with precision $\tau_u\times s$,
 and $s$ is a vector of fixed scalings. 


\subsection*{Hyperparameters}

This model has 2 hyperparameters, $\theta = (\theta_{1}, \theta_{2})$.
The hyperparameter specification is as follows:
\begin{displaymath}
    \theta_{1} = \beta
\end{displaymath}
and the prior is defined on $\theta_{1}$,
\begin{displaymath}
    \theta_{2} = \log(\tau_u)
\end{displaymath}
and the prior is defined on $\theta_{2}$.

\subsection*{Specification}

The MEB is specified inside the {\tt f()}
function as
\begin{verbatim}
 f(w, [<weights>,] model="meb", hyper = <hyper>, scale = <s>)
\end{verbatim}
Here, \texttt{w} are the observed covariates, and the fixed scaling of
the observational precision is given in argument \texttt{scale}. If
the argument \texttt{scale} is not given, then $s$ is set to $1$.

Note that only the unique values of \texttt{w} are used, so if two or
more elements of \texttt{w} are \emph{identical}, then they refer to
the \emph{same} element in the covariate $x$.


\subsubsection*{Hyperparameter specification and default values}
\documentclass[a4paper,11pt]{article}
\usepackage[scale={0.8,0.9},centering,includeheadfoot]{geometry}
\usepackage{amstext}
\usepackage{listings}
\usepackage{verbatim}
\begin{document}

\section*{The Berkson Measurement Error (MEB) model}

\subsection*{Parametrization}

This is an implementation of the Berksom measurement error model for a
fixed effect. The observed covariate is $w$ but it is $x$
that goes into the linear predictor
\begin{displaymath}
    \eta = \ldots + \beta x + \ldots \ ,
\end{displaymath}
where $x = w + u$, $u$ is Gaussian with precision $\tau_u\times s$,
 and $s$ is a vector of fixed scalings. 


\subsection*{Hyperparameters}

This model has 2 hyperparameters, $\theta = (\theta_{1}, \theta_{2})$.
The hyperparameter specification is as follows:
\begin{displaymath}
    \theta_{1} = \beta
\end{displaymath}
and the prior is defined on $\theta_{1}$,
\begin{displaymath}
    \theta_{2} = \log(\tau_u)
\end{displaymath}
and the prior is defined on $\theta_{2}$.

\subsection*{Specification}

The MEB is specified inside the {\tt f()}
function as
\begin{verbatim}
 f(w, [<weights>,] model="meb", hyper = <hyper>, scale = <s>)
\end{verbatim}
Here, \texttt{w} are the observed covariates, and the fixed scaling of
the observational precision is given in argument \texttt{scale}. If
the argument \texttt{scale} is not given, then $s$ is set to $1$.

Note that only the unique values of \texttt{w} are used, so if two or
more elements of \texttt{w} are \emph{identical}, then they refer to
the \emph{same} element in the covariate $x$.


\subsubsection*{Hyperparameter specification and default values}
\documentclass[a4paper,11pt]{article}
\usepackage[scale={0.8,0.9},centering,includeheadfoot]{geometry}
\usepackage{amstext}
\usepackage{listings}
\usepackage{verbatim}
\begin{document}

\section*{The Berkson Measurement Error (MEB) model}

\subsection*{Parametrization}

This is an implementation of the Berksom measurement error model for a
fixed effect. The observed covariate is $w$ but it is $x$
that goes into the linear predictor
\begin{displaymath}
    \eta = \ldots + \beta x + \ldots \ ,
\end{displaymath}
where $x = w + u$, $u$ is Gaussian with precision $\tau_u\times s$,
 and $s$ is a vector of fixed scalings. 


\subsection*{Hyperparameters}

This model has 2 hyperparameters, $\theta = (\theta_{1}, \theta_{2})$.
The hyperparameter specification is as follows:
\begin{displaymath}
    \theta_{1} = \beta
\end{displaymath}
and the prior is defined on $\theta_{1}$,
\begin{displaymath}
    \theta_{2} = \log(\tau_u)
\end{displaymath}
and the prior is defined on $\theta_{2}$.

\subsection*{Specification}

The MEB is specified inside the {\tt f()}
function as
\begin{verbatim}
 f(w, [<weights>,] model="meb", hyper = <hyper>, scale = <s>)
\end{verbatim}
Here, \texttt{w} are the observed covariates, and the fixed scaling of
the observational precision is given in argument \texttt{scale}. If
the argument \texttt{scale} is not given, then $s$ is set to $1$.

Note that only the unique values of \texttt{w} are used, so if two or
more elements of \texttt{w} are \emph{identical}, then they refer to
the \emph{same} element in the covariate $x$.


\subsubsection*{Hyperparameter specification and default values}
\input{../hyper/latent/meb.tex}

\subsection*{Example}

\verbatiminput{meb-example.R}

\subsection*{Notes}

\begin{itemize}
\item \texttt{INLA} provide the posterior of $\nu=\beta x$ and NOT
    $x$.  The results comes (default) in the order given by the sorted
    (from low to high) values of \texttt{x} and the field \texttt{ID}
    gives the mapping.
\item The option \verb|scale| defines the scaling in the same order as
    argument \verb|values|.  It is therefore adviced to also give
    argument \verb|values| when \verb|scale| is used to be sure that
    they are consistent.
\end{itemize}


\end{document}



%%% Local Variables: 
%%% TeX-master: t
%%% End: 



\subsection*{Example}

\verbatiminput{meb-example.R}

\subsection*{Notes}

\begin{itemize}
\item \texttt{INLA} provide the posterior of $\nu=\beta x$ and NOT
    $x$.  The results comes (default) in the order given by the sorted
    (from low to high) values of \texttt{x} and the field \texttt{ID}
    gives the mapping.
\item The option \verb|scale| defines the scaling in the same order as
    argument \verb|values|.  It is therefore adviced to also give
    argument \verb|values| when \verb|scale| is used to be sure that
    they are consistent.
\end{itemize}


\end{document}



%%% Local Variables: 
%%% TeX-master: t
%%% End: 



\subsection*{Example}

\verbatiminput{meb-example.R}

\subsection*{Notes}

\begin{itemize}
\item \texttt{INLA} provide the posterior of $\nu=\beta x$ and NOT
    $x$.  The results comes (default) in the order given by the sorted
    (from low to high) values of \texttt{x} and the field \texttt{ID}
    gives the mapping.
\item The option \verb|scale| defines the scaling in the same order as
    argument \verb|values|.  It is therefore adviced to also give
    argument \verb|values| when \verb|scale| is used to be sure that
    they are consistent.
\end{itemize}


\end{document}



%%% Local Variables: 
%%% TeX-master: t
%%% End: 



\subsection*{Example}

\verbatiminput{meb-example.R}

\subsection*{Notes}

\begin{itemize}
\item \texttt{INLA} provides the posteriors of $\nu_i=\beta x_i$ and NOT $x_i$.
\item The posteriors of $\nu_i$ come (default) in the order given by the sorted
(from low to high) values of \texttt{w}. The entry \verb|$ID|
gives the mapping.
\item The option \verb|scale| defines the scaling in the same order as
argument \verb|values|. It is therefore adviced to also give
argument \verb|values| when \verb|scale| is used to be sure that
they are consistent.
\end{itemize}


\end{document}



%%% Local Variables:
%%% TeX-master: t
%%% End:
