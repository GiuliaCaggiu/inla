\documentclass[a4paper,11pt]{article}
\usepackage[scale={0.8,0.9},centering,includeheadfoot]{geometry}
\usepackage{amstext}
\usepackage{listings}
\usepackage{verbatim}
\begin{document}

\section*{The Ornstein-Uhlenbeck process}

\subsection*{Parametrization}

The Ornstein-Uhlenbeck process is defined with (mean zero), as the SDE
\begin{displaymath}
    dx_{t} = -\phi x_{t} + \sigma dW_{t}
\end{displaymath}
where $\phi>0$ and $W_{t}$ is the Wiener process. This is the
continuous time analogue to the discrete time AR$(1)$ model.

The process has a Markov property. Let ${x} = (x_{1}, x_{2}, \ldots,
x_{n})$ be value of the process at increasing time-points ${t} =
(t_{1}, t_{2}, \ldots, t_{n})$, then the conditional distribution
\begin{displaymath}
    x_{i} \mid x_{1},  \ldots, x_{i-1}, \qquad i=2, \ldots, n,
\end{displaymath}
is Gaussian with mean
\begin{displaymath}
    x_{i-1} \exp(-\phi \delta_{i})
\end{displaymath}
and precision
\begin{displaymath}
    \tau\left(1-\exp(-2\phi \delta_{i})
    \right)^{-1}
\end{displaymath}
where
\begin{displaymath}
    \delta_{i} = t_{i} - t_{i-1},\qquad i=2,\ldots, n
\end{displaymath}
and
\begin{displaymath}
    \tau = 2\phi/\sigma^{2}.
\end{displaymath}
The marginal distribution for $x_{1}$ is taken to be the stationary
distribution, which is a zero mean Gaussian with precision $\tau$.

\subsection*{Hyperparameters}

The precision parameter $\tau$ is represented as
\begin{displaymath}
    \theta_1 =\log(\tau) 
\end{displaymath}
where $\tau$ is the \emph{marginal} precision for the
Ornstein-Uhlenbeck process given above.

The parameter $\phi$ is represented as
\begin{displaymath}
    \theta_{2} = \log( \phi )
\end{displaymath}
and the prior is defined on $\mathbf{\theta}=(\theta_1,\theta_2)$.

\subsection*{Specification}

The Ornstein-Uhlenbeck model is specified inside the {\tt f()}
function as
\begin{verbatim}
 f(<whatever>, model="ou", values=<values>, hyper = <hyper>)
\end{verbatim}
The optional argument {\tt values } gives the time-points where the process is
defined/observed on (default is \verb|unique(sort(<whatever>))|).

\subsubsection*{Hyperparameter specification and default values}
%% DO NOT EDIT!
%% This file is generated automatically from models.R
\begin{description}
	\item[hyper]\ 
	 \begin{description}
	 	\item[theta1]\ 
	 	 \begin{description}
	 	 	\item[name] log precision
	 	 	\item[short.name] prec
	 	 	\item[prior] loggamma
	 	 	\item[param] 1 5e-05
	 	 	\item[initial] 4
	 	 	\item[fixed] FALSE
	 	 	\item[to.theta] \verb|function(x) log(x)|
	 	 	\item[from.theta] \verb|function(x) exp(x)|
	 	 \end{description}
	 	\item[theta2]\ 
	 	 \begin{description}
	 	 	\item[name] log phi
	 	 	\item[short.name] phi
	 	 	\item[prior] normal
	 	 	\item[param] 0 0.2
	 	 	\item[initial] -1
	 	 	\item[fixed] FALSE
	 	 	\item[to.theta] \verb|function(x) log(x)|
	 	 	\item[from.theta] \verb|function(x) exp(x)|
	 	 \end{description}
	 \end{description}
	\item[constr] FALSE
	\item[nrow.ncol] FALSE
	\item[augmented] FALSE
	\item[aug.factor] 1
	\item[aug.constr] 
	\item[n.div.by] 
	\item[n.required] FALSE
	\item[set.default.values] FALSE
	\item[pdf] ou
\end{description}


\subsection*{Example}

\verbatiminput{ou-example.R}

\subsection*{Notes}

The Ornstein-Uhlenbeck process is the continuous-time analogue to the
discrete AR$(1)$ model (for positive lag-one correlation only), but
they are parameterised slightly different.

\end{document}


% LocalWords:  Ornstein Uhlenbeck Parametrization SDE dx dW Hyperparameters ou

%%% Local Variables: 
%%% TeX-master: t
%%% End: 
% LocalWords:  Hyperparameter parameterised
