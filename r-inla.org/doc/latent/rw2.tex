\documentclass[a4paper,11pt]{article}
\usepackage[compat2]{geometry}
\usepackage{amstext}
\usepackage{listings}
\begin{document}

\section*{Random walk model of order $2$ (RW2)}

\subsection*{Parametrization}
The random walk model of order $2$ (RW2) for the Gaussian vector $\mathbf{x}=(x_1,\dots,x_n)$ is constructed assuming independent second-orderincrements:
\[
\Delta^2 x_i = x_i-2\ x_{i+1}+x_{i+2}\sim\mathcal{N}(0,\tau^{-1})
\]
The density for $\mathbf{x}$ is derived from its $n-2$ second-order increments as
\begin{eqnarray}
\pi(\mathbf{x}|\tau) &\propto& \tau^{(n-2)/2} \exp\left\{-\frac{\tau}{2} \sum (\Delta^2 x_i)^2\right\}\\
& = &\tau^{(n-2)/2}\exp\left\{-\frac{1}{2}\mathbf{x}^T\mathbf{Q}\mathbf{x} \right\}
\end{eqnarray}
where $\mathbf{Q}=\tau\mathbf{R}$ and $\mathbf{R}$ is the structure matrix reflecting the neighbourhood structure of the model.

It is also possible to define a {\it cyclic} version of the RW2 model.

\subsection*{Hyperparameters}

The precision parameter $\tau$ is represented as
\begin{displaymath}
    \theta =\log \tau
\end{displaymath}
and the prior is defined on $\theta$. 

\subsection*{Specification}

The RW2 model is specified inside the {\tt f()} function as
\begin{verbatim}
 f(<whatever>,model="rw2",values=<values>,cyclic=<TRUE,FALSE>,
              prior=c(<prior.model.theta>),
              param=c(<param.prior.theta1>))
\end{verbatim}
The (optional) argument {\tt values } is a numeric or factor vector giving the values assumed by the covariate for
 which we want the effect to be estimated. See the example for RW1 for an application. 

\subsection*{Example}

\begin{verbatim}

n=100
z=seq(0,6,length.out=n)
y=sin(z)+rnorm(n,mean=0,sd=0.5)
data=data.frame(y=y,z=z)

formula=y~f(z,model="rw2",prior="loggamma",param=c(1,0.01))
result=inla(formula,data=data,family="gaussian")
\end{verbatim}


\subsection*{Notes}

The RW2 is a intrinsic random field with rank deficiency of 2.

There exist also support to define irregular RW2 models. 


\end{document}


% LocalWords: 

%%% Local Variables: 
%%% TeX-master: t
%%% End: 
