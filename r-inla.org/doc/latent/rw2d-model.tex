\documentclass[a4paper,11pt]{article}
\usepackage[scale={0.8,0.9},centering,includeheadfoot]{geometry}
\usepackage{amstext}
\usepackage{verbatim}
\usepackage{amsmath,amssymb}

\begin{document}
\bibliographystyle{apalike}


\section*{The RW2d-model}
\subsection*{Parametrization}

The 2-dimensional random walk model is defined on a regular grid.  The
full conditional distributions for the nodes in the interior of the
grid are given by:
\begin{eqnarray}\label{eq48}
    \text{E}(x_{i} \mid \mathbf{x}_{-i}, \tau)
    &=& \frac{1}{20}\Biggl( 8 \;
    \begin{smallmatrix}
        \circ & \circ & \circ & \circ & \circ
        \\
        \circ & \circ & \bullet & \circ & \circ
        \\
        \circ & \bullet & \circ & \bullet & \circ
        \\
        \circ & \circ & \bullet & \circ & \circ
        \\
        \circ & \circ & \circ & \circ & \circ
        \\
    \end{smallmatrix}
    \; -2 \;
    \begin{smallmatrix}
        \circ & \circ & \circ & \circ & \circ
        \\
        \circ & \bullet & \circ & \bullet & \circ
        \\
        \circ & \circ & \circ & \circ & \circ
        \\
        \circ & \bullet & \circ & \bullet & \circ
        \\
        \circ & \circ & \circ & \circ & \circ
        \\
    \end{smallmatrix}
    \; -1 \;
    \begin{smallmatrix}
        \circ & \circ & \bullet & \circ & \circ
        \\
        \circ & \circ & \circ & \circ & \circ
        \\
        \bullet & \circ & \circ & \circ & \bullet
        \\
        \circ & \circ & \circ & \circ & \circ
        \\
        \circ & \circ & \bullet & \circ & \circ
        \\
    \end{smallmatrix}
    \Biggr), \\ \label{eq49}
    \text{Prec}(x_{i}\mid \mathbf{x}_{-i},
    \tau)
    &=& 20\tau.
\end{eqnarray}
Necessary corrections to to equations (\ref{eq48}) and (\ref{eq49})
near the boundary can be found using the stencils in~\cite{art398}.
For a detailed description of this model
see~\cite[Sec.~3.4.2]{book80}.

\subsection*{Hyperparameters}

The precision parameter $\tau$ is the only hyperparameter, $\theta =
\tau$. It is represented internally as $\log\tau$ and the prior is
assigned to $\log\tau$.

\subsection*{Specification}

The {\tt rw2d} model is specified insiede the {\tt f()} function as:
\begin{verbatim}
f(<whatever>,model="rw2d",
  nrow=<n.of rows>,ncol=<n.of columns>,
  hyper= <hyper>, scale.model = FALSE)
\end{verbatim}

The logical option \verb|scale.model| determine if the model should be
scaled to have an average variance (the diagonal of the generalized
inverse) equal to 1. This makes prior spesification much
easier. Default is \verb|FALSE| so that the model is not scaled.



\subsubsection*{Hyperparameter spesification and default values}
%% DO NOT EDIT!
%% This file is generated automatically from models.R
\begin{description}
	\item[hyper]\ 
	 \begin{description}
	 	\item[theta]\ 
	 	 \begin{description}
	 	 	\item[name] log precision
	 	 	\item[short.name] prec
	 	 	\item[initial] 4
	 	 	\item[fixed] FALSE
	 	 	\item[prior] loggamma
	 	 	\item[param] 1 5e-05
	 	 	\item[to.theta] \verb|function(x) log(x)|
	 	 	\item[from.theta] \verb|function(x) exp(x)|
	 	 \end{description}
	 \end{description}
	\item[constr] TRUE
	\item[nrow.ncol] TRUE
	\item[augmented] FALSE
	\item[aug.factor] 1
	\item[aug.constr] 
	\item[n.div.by] 
	\item[n.required] FALSE
	\item[set.default.values] TRUE
	\item[pdf] rw2d
\end{description}


\subsection*{Example}

\verbatiminput{rw2d-example.R}

\subsection*{Notes}
All indexes in the R-INLA library are one-dimensional so an
appropriate mapping is required to get it into the ordering defined
internally in \verb|inla|; see \verb|?inla.matrix2vector|,
\verb|?inla.vector2matrix|, \verb|?inla.node2lattice| and
\verb|?inla.lattice2node|.


The $\frac{n-r}{2}\log(|R|^{*})$-part (with $r=3$) of the
normalisation constant is not computed, hence you need to add this
part to the log marginal likelihood estimate, if you need it. Here,
$Q=\tau R$. 


{\small\bibliography{../mybib}}


\end{document}
