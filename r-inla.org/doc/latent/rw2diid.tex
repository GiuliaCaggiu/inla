\documentclass[a4paper,11pt]{article}
\usepackage[scale={0.8,0.9},centering,includeheadfoot]{geometry}
\usepackage{amstext}
\usepackage{amsmath}
\usepackage{listings}
\begin{document}

\section*{RW2DIID model for spatial effects}

\subsection*{Parametrization}

This model is a union of the \lstinline$RW2D$ model $u^{*}$ and a
\lstinline$iid$ model $v^{*}$, so that
\begin{displaymath}
    x =
    \begin{pmatrix}
        v^{*} + u^{*}\\
        u^{*}
    \end{pmatrix}
\end{displaymath}
where both $u^{*}$ and $v^{*}$ has a precision (hyper-)parameter.  The
length of $x$ is $2n$ if the length of $u^{*}$ (and $v^{*}$) is
$n$. The RW2DIID model uses a different parameterisation of the
hyperparameters where
\begin{displaymath}
    x =
    \begin{pmatrix}
        \frac{1}{\sqrt{\tau}}\left(\sqrt{1-\phi} \;v +
          \sqrt{\phi} \;u\right)\\
        u
    \end{pmatrix}
\end{displaymath}
where both $u$ and $v$ are \emph{standardised} to have (generalised)
variance equal to one.  The \emph{marginal} precision is then $\tau$
and the proportion of the marginal variance explained by the spatial
effect ($u$) is $\phi$.

\subsection*{Hyperparameters}
The hyperparameters are the margainal precision $\tau$ and the mixing
parameter $\phi$.  The marginal precision $\tau$ is represented as
\begin{displaymath}
    \theta_{1} = \log(\tau)
\end{displaymath}
and the mixing parameter as
\begin{displaymath}
    \theta_{2} = \log\left(\frac{\phi}{1-\phi}\right)
\end{displaymath}
and the prior is defined on $\mathbf{\theta} = (\theta_{1}, \theta_{2})$.

\subsection*{Specification}

The rw2diid model is specified inside the {\tt f()} function as
\begin{verbatim}
 f(<whatever>, model="rw2didd", nrow=<nrow>, ncol=<ncol>,
   hyper=<hyper>)
\end{verbatim}

\subsubsection*{Hyperparameter spesification and default values}
\documentclass[a4paper,11pt]{article}
\usepackage[scale={0.8,0.9},centering,includeheadfoot]{geometry}
\usepackage{amstext}
\usepackage{amsmath}
\usepackage{listings}
\begin{document}

\section*{RW2DIID model for spatial effects}

\subsection*{Parametrization}

This model is a union of the \lstinline$RW2D$ model $u^{*}$ and a
\lstinline$iid$ model $v^{*}$, so that
\begin{displaymath}
    x =
    \begin{pmatrix}
        v^{*} + u^{*}\\
        u^{*}
    \end{pmatrix}
\end{displaymath}
where both $u^{*}$ and $v^{*}$ has a precision (hyper-)parameter.  The
length of $x$ is $2n$ if the length of $u^{*}$ (and $v^{*}$) is
$n$. The RW2DIID model uses a different parameterisation of the
hyperparameters where
\begin{displaymath}
    x =
    \begin{pmatrix}
        \frac{1}{\sqrt{\tau}}\left(\sqrt{1-\phi} \;v +
          \sqrt{\phi} \;u\right)\\
        u
    \end{pmatrix}
\end{displaymath}
where both $u$ and $v$ are \emph{standardised} to have (generalised)
variance equal to one.  The \emph{marginal} precision is then $\tau$
and the proportion of the marginal variance explained by the spatial
effect ($u$) is $\phi$.

\subsection*{Hyperparameters}
The hyperparameters are the margainal precision $\tau$ and the mixing
parameter $\phi$.  The marginal precision $\tau$ is represented as
\begin{displaymath}
    \theta_{1} = \log(\tau)
\end{displaymath}
and the mixing parameter as
\begin{displaymath}
    \theta_{2} = \log\left(\frac{\phi}{1-\phi}\right)
\end{displaymath}
and the prior is defined on $\mathbf{\theta} = (\theta_{1}, \theta_{2})$.

\subsection*{Specification}

The rw2diid model is specified inside the {\tt f()} function as
\begin{verbatim}
 f(<whatever>, model="rw2didd", nrow=<nrow>, ncol=<ncol>,
   hyper=<hyper>)
\end{verbatim}

\subsubsection*{Hyperparameter spesification and default values}
\documentclass[a4paper,11pt]{article}
\usepackage[scale={0.8,0.9},centering,includeheadfoot]{geometry}
\usepackage{amstext}
\usepackage{amsmath}
\usepackage{listings}
\begin{document}

\section*{RW2DIID model for spatial effects}

\subsection*{Parametrization}

This model is a union of the \lstinline$RW2D$ model $u^{*}$ and a
\lstinline$iid$ model $v^{*}$, so that
\begin{displaymath}
    x =
    \begin{pmatrix}
        v^{*} + u^{*}\\
        u^{*}
    \end{pmatrix}
\end{displaymath}
where both $u^{*}$ and $v^{*}$ has a precision (hyper-)parameter.  The
length of $x$ is $2n$ if the length of $u^{*}$ (and $v^{*}$) is
$n$. The RW2DIID model uses a different parameterisation of the
hyperparameters where
\begin{displaymath}
    x =
    \begin{pmatrix}
        \frac{1}{\sqrt{\tau}}\left(\sqrt{1-\phi} \;v +
          \sqrt{\phi} \;u\right)\\
        u
    \end{pmatrix}
\end{displaymath}
where both $u$ and $v$ are \emph{standardised} to have (generalised)
variance equal to one.  The \emph{marginal} precision is then $\tau$
and the proportion of the marginal variance explained by the spatial
effect ($u$) is $\phi$.

\subsection*{Hyperparameters}
The hyperparameters are the margainal precision $\tau$ and the mixing
parameter $\phi$.  The marginal precision $\tau$ is represented as
\begin{displaymath}
    \theta_{1} = \log(\tau)
\end{displaymath}
and the mixing parameter as
\begin{displaymath}
    \theta_{2} = \log\left(\frac{\phi}{1-\phi}\right)
\end{displaymath}
and the prior is defined on $\mathbf{\theta} = (\theta_{1}, \theta_{2})$.

\subsection*{Specification}

The rw2diid model is specified inside the {\tt f()} function as
\begin{verbatim}
 f(<whatever>, model="rw2didd", nrow=<nrow>, ncol=<ncol>,
   hyper=<hyper>)
\end{verbatim}

\subsubsection*{Hyperparameter spesification and default values}
\documentclass[a4paper,11pt]{article}
\usepackage[scale={0.8,0.9},centering,includeheadfoot]{geometry}
\usepackage{amstext}
\usepackage{amsmath}
\usepackage{listings}
\begin{document}

\section*{RW2DIID model for spatial effects}

\subsection*{Parametrization}

This model is a union of the \lstinline$RW2D$ model $u^{*}$ and a
\lstinline$iid$ model $v^{*}$, so that
\begin{displaymath}
    x =
    \begin{pmatrix}
        v^{*} + u^{*}\\
        u^{*}
    \end{pmatrix}
\end{displaymath}
where both $u^{*}$ and $v^{*}$ has a precision (hyper-)parameter.  The
length of $x$ is $2n$ if the length of $u^{*}$ (and $v^{*}$) is
$n$. The RW2DIID model uses a different parameterisation of the
hyperparameters where
\begin{displaymath}
    x =
    \begin{pmatrix}
        \frac{1}{\sqrt{\tau}}\left(\sqrt{1-\phi} \;v +
          \sqrt{\phi} \;u\right)\\
        u
    \end{pmatrix}
\end{displaymath}
where both $u$ and $v$ are \emph{standardised} to have (generalised)
variance equal to one.  The \emph{marginal} precision is then $\tau$
and the proportion of the marginal variance explained by the spatial
effect ($u$) is $\phi$.

\subsection*{Hyperparameters}
The hyperparameters are the margainal precision $\tau$ and the mixing
parameter $\phi$.  The marginal precision $\tau$ is represented as
\begin{displaymath}
    \theta_{1} = \log(\tau)
\end{displaymath}
and the mixing parameter as
\begin{displaymath}
    \theta_{2} = \log\left(\frac{\phi}{1-\phi}\right)
\end{displaymath}
and the prior is defined on $\mathbf{\theta} = (\theta_{1}, \theta_{2})$.

\subsection*{Specification}

The rw2diid model is specified inside the {\tt f()} function as
\begin{verbatim}
 f(<whatever>, model="rw2didd", nrow=<nrow>, ncol=<ncol>,
   hyper=<hyper>)
\end{verbatim}

\subsubsection*{Hyperparameter spesification and default values}
\input{../hyper/latent/rw2diid.tex}


\subsection*{Example}


\subsection*{Notes}

The term $\frac{1}{2}\log(|R|^{*})$ of the normalisation constant is
not computed, hence you need to add this part to the log marginal
likelihood estimate, if you need it. Here $R$ is the precision matrix
for the standardised RW2D part of the model.

The generic PC-prior for $\phi$ is available as \texttt{prior="pc"}
and parameters \texttt{param="c(u, alpha)"}, where $\text{Prob}(\phi
\le u) = \alpha$. If $\alpha < 0$ or $\alpha>1$, then it is set to a value
close to the minimum value of $\alpha$ allowed.

\end{document}


% LocalWords: 

%%% Local Variables: 
%%% TeX-master: t
%%% End: 



\subsection*{Example}


\subsection*{Notes}

The term $\frac{1}{2}\log(|R|^{*})$ of the normalisation constant is
not computed, hence you need to add this part to the log marginal
likelihood estimate, if you need it. Here $R$ is the precision matrix
for the standardised RW2D part of the model.

The generic PC-prior for $\phi$ is available as \texttt{prior="pc"}
and parameters \texttt{param="c(u, alpha)"}, where $\text{Prob}(\phi
\le u) = \alpha$. If $\alpha < 0$ or $\alpha>1$, then it is set to a value
close to the minimum value of $\alpha$ allowed.

\end{document}


% LocalWords: 

%%% Local Variables: 
%%% TeX-master: t
%%% End: 



\subsection*{Example}


\subsection*{Notes}

The term $\frac{1}{2}\log(|R|^{*})$ of the normalisation constant is
not computed, hence you need to add this part to the log marginal
likelihood estimate, if you need it. Here $R$ is the precision matrix
for the standardised RW2D part of the model.

The generic PC-prior for $\phi$ is available as \texttt{prior="pc"}
and parameters \texttt{param="c(u, alpha)"}, where $\text{Prob}(\phi
\le u) = \alpha$. If $\alpha < 0$ or $\alpha>1$, then it is set to a value
close to the minimum value of $\alpha$ allowed.

\end{document}


% LocalWords: 

%%% Local Variables: 
%%% TeX-master: t
%%% End: 



\subsection*{Example}


\subsection*{Notes}

The term $\frac{1}{2}\log(|R|^{*})$ of the normalisation constant is
not computed, hence you need to add this part to the log marginal
likelihood estimate, if you need it. Here $R$ is the precision matrix
for the standardised RW2D part of the model.

The generic PC-prior for $\phi$ is available as \texttt{prior="pc"}
and parameters \texttt{param="c(u, alpha)"}, where $\text{Prob}(\phi
\le u) = \alpha$. If $\alpha < 0$ or $\alpha>1$, then it is set to a value
close to the minimum value of $\alpha$ allowed.

\end{document}


% LocalWords: 

%%% Local Variables: 
%%% TeX-master: t
%%% End: 
