\documentclass[a4paper,11pt]{article}
\usepackage[scale={0.8,0.9},centering,includeheadfoot]{geometry}
\usepackage{amstext}
\usepackage{listings}
\usepackage{verbatim}
\begin{document}

\section*{Spatial lag model for spatial effects}

\subsection*{Parametrization}

The \texttt{slm} model is defined as
\begin{displaymath}
    \mathbf{x} = (I_n-\rho W)^{-1}(X\beta +\varepsilon)
\end{displaymath}
where $I_n$ is the identity matrix of dimension $n\times n$, $W$ is a
spatial weights matrix, $X$ is a matrix of covariates, $\rho$ is a
spatial autocorrelation parameter, $\beta$ are coefficients of the
covariates and $\varepsilon$ is zero mean Gaussian noise with precision
$\tau$.

\subsection*{Hyperparameters}

This model has two hyperparameters $\mathbf{\theta} = (\theta_{1},
\theta_{2})$. The precision parameter $\tau$ is represented as
\begin{displaymath}
    \theta_{1} =\log \tau
\end{displaymath}
and the prior is defined on $\theta_{1}$. The spatial autocorrelation
parameter $\rho$ is represented as
\begin{displaymath}
    \rho^{*} = \frac{\rho - \rho_{\text{min}}}{
        \rho_{\text{max}} - \rho_{\text{min}}}
\end{displaymath}
and then
\begin{displaymath}
    \theta_{2} = \log(\rho^{*}/(1-\rho^{*}))
\end{displaymath}
and the prior is defined on $\theta_{2}$. Here, $\rho_{\text{min}}$
and $\rho_{\text{max}}$ are lower and upper limits of the legal range
for $\rho$.

\subsection*{Specification}

The slm model is specified inside the {\tt f()} function as
\begin{verbatim}
 f(<whatever>, model="slm",
        args.slm = list(rho.min = NULL,
                        rho.max = NULL,
                        X = NULL,
                        W = NULL,
                        Q.beta = NULL))
\end{verbatim}
$\verb|args.slm|$ is used to define the slm-spesific parameters in the
model.
\begin{description}
\item[rho.min and rho.max] define the range in which $\rho$ can take
    values. Note that, $\rho^{*}$ is in the interval $(0,1)$ and that
    it is re-scaled to the interval (\verb|rho.min|, \verb|rho.max|)
    when computing $I_n-\rho W$.  Initial values on $\rho$ need to be
    re-scaled to the (0,1) interval.
\item[X] defines the matrix of covariates.
\item[W] defines the adjacency matrix.
\item[Q.beta] defines the precision of the vector of coefficients
    $\beta$ in the model.
\end{description}

\subsubsection*{Hyperparameter spesification and default values}
%% DO NOT EDIT!
%% This file is generated automatically from models.R
\begin{description}
	\item[hyper]\ 
	 \begin{description}
	 	\item[theta1]\ 
	 	 \begin{description}
	 	 	\item[name] log precision
	 	 	\item[short.name] prec
	 	 	\item[initial] 4
	 	 	\item[fixed] FALSE
	 	 	\item[prior] loggamma
	 	 	\item[param] 1 5e-05
	 	 	\item[to.theta] \verb!function(x) log(x)!
	 	 	\item[from.theta] \verb!function(x) exp(x)!
	 	 \end{description}
	 	\item[theta2]\ 
	 	 \begin{description}
	 	 	\item[name] rho
	 	 	\item[short.name] rho
	 	 	\item[initial] 0
	 	 	\item[fixed] FALSE
	 	 	\item[prior] normal
	 	 	\item[param] 0 10
	 	 	\item[to.theta] \verb!function(x) log(x/(1-x))!
	 	 	\item[from.theta] \verb!function(x) 1/(1+exp(-x))!
	 	 \end{description}
	 \end{description}
	\item[constr] FALSE
	\item[nrow.ncol] FALSE
	\item[augmented] FALSE
	\item[aug.factor] 1
	\item[aug.constr] 
	\item[n.div.by] 
	\item[n.required] TRUE
	\item[set.default.values] TRUE
	\item[pdf] slm
	\item[status] experimental
\end{description}



\subsection*{Example}

{\small\verbatiminput{slm-example.R}}



\subsection*{Notes}

The estimates of $\beta$ are included at the end of the vector of random
effects. See the example for details on how to extract them.

\end{document}


% LocalWords: 

%%% Local Variables: 
%%% TeX-master: t
%%% End: 
