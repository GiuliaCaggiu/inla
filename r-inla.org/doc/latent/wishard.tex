\documentclass[a4paper,11pt]{article}
\usepackage[compat2]{geometry}
\usepackage{amstext}
\usepackage{listings}
\usepackage{block}
\begin{document}

\section*{The Wishart model for correlated effects}

This model is availabe for dimensions $p=2,3$, we describe in detail
the case for $p=2$ and the the case for $p=3$.

\subsection*{Parametrization}

The 2-dimesional Wishard model is used if one want to define the model
for the linear predictor $\eta$ as:
\[
\eta=a+b
\]
where $a$ and $b$ are correlated
\[
\left(
  \begin{array}{c}
      a\\
      b
  \end{array}\right)
\sim \mathcal{N}\left(\mathbf{0}, \mathbf{W}^{-1}\right)
\]
with covariance matrix $\mathbf{W}^{-1}$
\begin{equation}
    \label{precision}
    \mathbf{W}^{-1} = \left(\begin{array}{cc}
          1/\tau_a & \rho \sqrt{\tau_a\tau_b}\\
          \rho\sqrt{\tau_a\tau_b}&  1/\tau_b
      \end{array}\right)
\end{equation}
and $\tau_{a}$, $\tau_{b}$ and $\rho$ are the hyperparameters. In this
case the following model is implemented for the precision matrix
$\mathbf{W}$
\begin{displaymath}
    \mathbf{W}
    \;\sim\;\text{Wishart}_{p}(r, \mathbf{R}^{-1}), \quad p=2
\end{displaymath}
where the Wishart distribution has density
\begin{displaymath}
    \pi(\mathbf{W}) = c^{-1} |\mathbf{W}|^{(r-(p+1))/2} \exp\left\{
      -\frac{1}{2}\text{Trace}(\mathbf{W}\mathbf{R})\right\}, \quad r > p+1
\end{displaymath}
and
\begin{displaymath}
    c = 2^{(rp)/2} |\mathbf{R}|^{-r/2} \pi^{(p(p-1))/4}\prod_{j=1}^{p} \Gamma((r+1-j)/2).
\end{displaymath}
Then,
\begin{displaymath}
    \text{E}(\mathbf{W}) = r\mathbf{R}^{-1}, \quad\text{and}\quad
    \text{E}(\mathbf{W}^{-1}) = \mathbf{R}/(r-(p+1)).
\end{displaymath}
\subsection*{Hyperparameters}

The hyperparameters are
\begin{displaymath}
    \theta = (\log\tau_{a}, \log\tau_{b}, \tilde\rho)
\end{displaymath}
where
\begin{displaymath}
    \rho =
    2\frac{\exp(\tilde\rho)}{\exp(\tilde{\rho})+1} -1
\end{displaymath}

The prior-parameters are
\[
(r,R_{11},R_{22},R_{12})
\]
where
\[
\mathbf{R}= \left(\begin{array}{cc}
      R_{11} &R_{12}\\
      R_{21} & R_{22}
  \end{array}\right)
\]
and $r_{12}=R_{21}$ due to symmetry.

The {\tt inla} function reports posterior distribution for the
hyperparameters $\tau_a,\tau_b,\rho$ in equation (\ref{precision}).

The prior for $\theta$ is {\bf fixed} to be {\tt wishart}


\subsection*{Specification}

The model 2d wishart for
\[
\eta=a+b
\]
is specified as
\begin{verbatim}
y~f(a,model="2diidwishartp1",param=<param.vector(4 elements)>) 
  +f(b,model="2diidwishartp2")
\end{verbatim}
The parameters for the Wishart distribution are specified {\it only}
for {\tt 2diidwishartp1}



\subsection*{Example}
In this example we implement the model
\[
y|\eta\sim\mbox{Pois}(\exp(\eta))
\]
where
\[
\eta=a+b+c
\]
and $b$ and $c$ are correlated as described above.

\begin{verbatim}

n=100
#set hyperparameters
r=4
R11=1
R22=2
R12=0.1
R=matrix(c(R11,R12,R12,R22),2,2)
S=solve(R)
#these are needed to simulate from a wishart prior
# and sample from a multivariate normal
library(MCMCpack)
library(mvtnorm)
W=rwish(r, S)
cc=rmvnorm(n, mean=c(0,0), sigma=solve(W))
a=cc[,1]
b=cc[,2]
#simulate data
x1=1:n
x2=1:n
eta=0.1+a+b
y=rpois(n,exp(eta))
data=data.frame(y=y,x1=1:n,x2=1:n)
#fit the model
formula=y~f(x1,model="2diidwishartp1",param=c(4,1,2,0.1))+f(x2,model="2diidwishartp2")
result=inla(formula,family="poisson",data=data)

\end{verbatim}

\subsection*{Notes}
If more than one pair of {\tt 2diidwishartp1/2} is defined, the
following rule is used to determine the match between {\tt p(art)1}
and {\tt p(art)2}.
\begin{quote}
    The first occurrence of {\tt 2diidwishartp1} belongs with the
    first occurrence of {\tt 2diidwishartp2}.  The second occurrence
    of {\tt 2diidwishartp1} belongs with the second occurrence of {\tt
        2diidwishartp2} and so on.
\end{quote}


\subsection*{Three dimensional case}


The previous formulation is also available for 3D. In this case the
hyperparameters are
\begin{displaymath}
    \theta = (\log \tau_{1}, \log \tau_{2}, \log \tau_{3},
    \tilde\rho_{12},
    \tilde\rho_{13},
    \tilde\rho_{23})
\end{displaymath}
In this case the
name the prior is \textbf{fixed} to be \texttt{Wishart3d}. The
parameters in the prior are
\begin{quote}
    \emph{parameters = } $r\;R_{11}\;R_{22}\;R_{33}\; R_{12}\;
    R_{13}\; R_{23}$
\end{quote}
where
\begin{displaymath}
    \mathbf{R} =
    \left[\begin{array}{ccc}
        R_{11} &R_{12} & R_{13}\\
        R_{12} & R_{22} & R_{23}\\
        R_{13} & R_{23} & R_{33}
    \end{array}\right]
\end{displaymath}
The reported hyperparameters are the marginal precisions $\tau_{1}$,
$\tau_{2}$ and $\tau_{3}$ and the correlations $\rho_{12}$,
$\rho_{13}$ and $\rho_{23}$. The model names are as given in the
following example.
\begin{verbatim}
formula2 <- Y ~ f(diid.part0,model="3diidwishartp1",
                  param=c(7,1,2,3,0.1,0.2,0.3))
                f(diid.part1,model="3diidwishartp2") +
                f(diid.part2,model="3diidwishartp3") +
\end{verbatim}

If more than one pair of {\tt 3diidwishartp1/2/3} is defined, the
following rule is used to determine the match between \emph{p(art)1,
    p(art)2} and \emph{p(art)3}.
\begin{quote}
    The first occurrence of {\tt 3diidwishartp1} belongs with the
    first occurrence of {\tt 3diidwishartp2} and {\tt 3diidwishartp3}.
    The second occurrence of {\tt 3diidwishartp1} belongs with the
    second occurrence of {\tt 3diidwishartp2} and the second
    occurrence of {\tt 3diidwishartp3} , and so on.
\end{quote}

\end{document}


% LocalWords: 

%%% Local Variables: 
%%% TeX-master: t
%%% End: 
