\documentclass[a4paper,11pt]{article}
\usepackage[scale={0.8,0.9},centering,includeheadfoot]{geometry}
\usepackage{amstext}
\usepackage{amsmath}
\usepackage{listings}
\usepackage{verbatim}
\begin{document}

\section*{The z-model}

\subsection*{Parametrization}

The z-model is an implementation of the ``classical'' way to define
the ``random effect'' part of a mixed model, through
\begin{displaymath}
    \eta = \ldots + Z z 
\end{displaymath}
where $Z$ is a $n\times m$ matrix and $z$ a vector of length $m$
representing zero-mean ``random effects''. The z-model is defined as
the augmented model
\begin{displaymath}
    \widetilde{z} =
    \begin{pmatrix}
        v \\
        z 
    \end{pmatrix}
\end{displaymath}
where $v \sim {\mathcal N}_{n}(Zz, \kappa I)$, where $\kappa$ is a high
fixed precision, and where the precision matrix for $z$ is $\tau C$
where $C > 0$ is a $m\times m$ (fixed) matrix and $\tau$ is the
precision parameter. 

\subsection*{Hyperparameters}

The precision parameter of the z-model is represented as
\begin{displaymath}
    \theta = \log(\tau)
\end{displaymath}
and prior is assigned to $\theta$. The parameter $\kappa$ is kept
fixed at all times.

\subsection*{Specification}

The z-model is specified inside the {\tt f()} function as
\begin{verbatim}
 f(<whatever>, model="z", Z = <Z>, Cmatrix = <Cmat>, hyper = <hyper>,
   precision = <precision>)
\end{verbatim}
where the required \texttt{Z}-matrix argument defines the $Z$
matrix. The (optional) \texttt{Cmatrix} defines the $C$ matrix and is
by default taken to the the diagonal matrix with dimension $m$.  The
\texttt{precision} parameter defines the value of $\kappa$, and
\texttt{hyper} the hyperparameter spesification for $\tau$.

If $Z$ is a $n\times m$ matrix then the $C$ matrix must be $m\times m$
matrix, and $\widetilde z$ has length $n+m$. The $n$ first terms of
$\widetilde z$ is $v$ and the last $m$ terms of $\widetilde z$ is $z$.

If \texttt{constr=TRUE} is given, then this is defined as
$\sum_{i=1}^{m}z_{i} = 0$. If \texttt{extraconstr} is given, then it
is applied to $\widetilde z$, hence \texttt{extraconstr\$A} must be a
$k\times (n+m)$ matrix where $k$ is the number of linear constraints.

\subsubsection*{Hyperparameter spesification and default values}
%% DO NOT EDIT!
%% This file is generated automatically from models.R
\begin{description}
	\item[hyper]\ 
	 \begin{description}
	 	\item[theta]\ 
	 	 \begin{description}
	 	 	\item[name] log precision
	 	 	\item[short.name] prec
	 	 	\item[initial] 4
	 	 	\item[fixed] FALSE
	 	 	\item[prior] loggamma
	 	 	\item[param] 1 5e-05
	 	 	\item[to.theta] \verb|function(x) log(x)|
	 	 	\item[from.theta] \verb|function(x) exp(x)|
	 	 \end{description}
	 \end{description}
	\item[constr] FALSE
	\item[nrow.ncol] FALSE
	\item[augmented] FALSE
	\item[aug.factor] 1
	\item[aug.constr] 
	\item[n.div.by] 
	\item[n.required] TRUE
	\item[set.default.values] TRUE
	\item[pdf] z.pdf
	\item[status] experimental
\end{description}



\subsection*{Example 1}

{\small\verbatiminput{example3-z.R}}

\subsection*{Example 2}

{\small\verbatiminput{example-z.R}}

\subsection*{Example 3}

{\small\verbatiminput{example2-z.R}}

\subsection*{Notes}

None.

\end{document}


% LocalWords: 

%%% Local Variables: 
%%% TeX-master: t
%%% End: 

