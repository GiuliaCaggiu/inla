\documentclass[a4paper,11pt]{article}
\usepackage[scale={0.8,0.9},centering,includeheadfoot]{geometry}
\usepackage{ifpdf}
\usepackage{amstext}
\usepackage{amsmath}
\usepackage{verbatim}
\newcommand{\vect}[1]{\boldsymbol{#1}}
\begin{document}
\section*{Exponential}

\subsection*{Parametrisation}

The Exponential distribution is
\begin{displaymath}
    \text{Prob}(y) = \lambda \exp(-\lambda y)\qquad \lambda > 0
\end{displaymath}
for responses $y>0$.\\
In survival analysis, models are generally specified through the
hazard function. For exponential model, the baseline hazard is
constant over time and the hazard function is:
\begin{displaymath}
    h(y)  = \lambda
\end{displaymath}

\subsection*{Link-function}
The parameter $\lambda$ is linked to the linear predictor as:
\[
\lambda = \exp(\eta)
\]

\subsection*{Hyperparameters}

None.

\subsection*{Specification}

\begin{itemize}
\item $\text{family}=\texttt{Exponential}$
\item Required arguments: $y$ (to be given in a format by using
    $\texttt{inla.surv()}$ function )
\end{itemize}

\subsubsection*{Hyperparameter spesification and default values}
%% DO NOT EDIT!
%% This file is generated automatically from models.R
\begin{description}
	\item[hyper]\ 
	\item[survival] TRUE
	\item[discrete] FALSE
	\item[link] default log
	\item[pdf] exponential
\end{description}



\subsection*{Example}

In the following example we estimate the parameters in a simulated
case \verbatiminput{example-exponential.R}

\subsection*{Notes}
\begin{itemize}
\item Exponential model can be used for right censored, left censored and interval censored data.\\
\item A general frame work to represent time is given by
    $\texttt{inla.surv()}$
\end{itemize}


\end{document}


% LocalWords:  np Hyperparameters Ntrials

%%% Local Variables: 
%%% TeX-master: t
%%% End: 
