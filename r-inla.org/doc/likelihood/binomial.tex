\documentclass[a4paper,11pt]{article}
\usepackage[scale={0.8,0.9},centering,includeheadfoot]{geometry}
\usepackage{amstext}
\usepackage{amsmath}
\usepackage{verbatim}

\begin{document}
\section*{Binomial}

\subsection*{Parametrisation}

The Binomial distribution is
\begin{displaymath}
    \text{Prob}(y) = {n \choose y} \ p^n (1-p)^{n-y}
\end{displaymath}
for responses $y=0, 1, 2, \ldots,n$, where
\begin{description}
\item[$n$:] number of trials.
\item[$p$:] probability of success in each trial.
\end{description}

\subsection*{Link-function}

The mean and variance of $y$ are given as
\begin{displaymath}
    \mu = np \qquad\text{and}\qquad \sigma^{2} = np(1-p)
\end{displaymath}
and the probability $p$ is linked to the linear predictor by
\begin{displaymath}
    p(\eta) = \frac{\exp(\eta)}{1+\exp(\eta)}
\end{displaymath}

\subsubsection*{Hyperparameters}
None.

\subsubsection*{Hyperparameter spesification and default values}
\documentclass[a4paper,11pt]{article}
\usepackage[compat2]{geometry}
\usepackage{amstext}
\usepackage{amsmath}
\usepackage{verbatim}

\begin{document}
\section*{Binomial}

\subsection*{Parametrisation}

The Binomial distribution is
\begin{displaymath}
    \text{Prob}(y) = {n \choose y} \ p^n (1-p)^{n-y}
\end{displaymath}
for responses $y=0, 1, 2, \ldots,n$, where
\begin{description}
\item[$n$:] number of trials.
\item[$p$:] probability of success in each trial.
\end{description}

\subsection*{Link-function}

The mean and variance of $y$ are given as
\begin{displaymath}
    \mu = np \qquad\text{and}\qquad \sigma^{2} = np(1-p)
\end{displaymath}
and the probability $p$ is linked to the linear predictor by
\begin{displaymath}
    p(\eta) = \frac{\exp(\eta)}{1+\exp(\eta)}
\end{displaymath}

\subsection*{Hyperparameters}

None.

\subsection*{Specification}

\begin{itemize}
\item $\text{family}=\texttt{binomial}$
\item Required arguments: $y$ and $n$ (keyword \texttt{Ntrials})
\end{itemize}

\subsection*{Example}

In the following example we estimate the parameters in a simulated
example with binomial responses.
\verbatiminput{example-binomial.R}

\subsection*{Notes}

None.


\end{document}


% LocalWords:  np Hyperparameters Ntrials

%%% Local Variables: 
%%% TeX-master: t
%%% End: 


\subsection*{Specification}

\begin{itemize}
\item $\text{family}=\texttt{binomial}$
\item Required arguments: $y$ and $n$ (keyword \texttt{Ntrials})
\end{itemize}

\subsection*{Example}

In the following example we estimate the parameters in a simulated
example with binomial responses.
\verbatiminput{example-binomial.R}

\subsection*{Notes}

If the response is a \verb|factor| it must be converted to $\{0,1\}$
before calling \verb|inla()|, as this conversion is not done automatic
(as for example in \verb|glm()|).


\end{document}


% LocalWords:  np Hyperparameters Ntrials

%%% Local Variables: 
%%% TeX-master: t
%%% End: 
