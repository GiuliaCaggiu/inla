\documentclass[a4paper,11pt]{article}
\usepackage[scale={0.8,0.9},centering,includeheadfoot]{geometry}
\usepackage{amstext}
\usepackage{amsmath}
\usepackage{verbatim}
\newcommand{\vect}[1]{\boldsymbol{#1}}
\begin{document}
\section*{Cox Proportional Hazards Model}

\subsection*{Parametrisation}

In the Cox proportional hazards model, defines the hazard rate as:
\begin{displaymath}
    h(t) = h_{0}(t)\exp(\eta)
\end{displaymath}
where
\begin{description}
\item[$h_{0}(\cdot)$:] baseline hazard
\item[$\eta$:] linear predictor
\end{description}

We start from a finite partition of the time axis
$0=s_{0}<s_{1}<\dots,s_{K}$ and assume the baseline hazard to be
constant in each time interval
\[
h_{0}(t) = \exp(b_{k})\mbox{ for }t\in(s_{k-1},s_{k}],\qquad
k=1,\dots,K
\]
and assign $\mathbf{b} = (b_{1},\dots,b_{K})$ a Gaussian prior (RW1 or
RW2) with unknown precision $\tau_{b}$
\subsection*{Link-function}

The parameter $\eta$ is the linear predictor

\subsection*{Hyperparameters}

The log precision $\log\tau_{b}$ for the piecewise constant hazard

\subsection*{Specification}

\begin{itemize}
\item $\text{family}=\texttt{coxph}$
\item Required arguments:
    \begin{itemize}
    \item $y$ (to be given in a format by using $\texttt{inla.surv()}$
        function )
    \item $\texttt{control.hazard = list()}$ to control the prior for
        the piecewise constant hazar, see $\texttt{?control.hazard}$
        for more information.
    \end{itemize}
\end{itemize}


\subsubsection*{Hyperparameter spesification and default values}
\paragraph{The ``RW1'' model for the hazard}
%% DO NOT EDIT!
%% This file is generated automatically from models.R
\begin{description}
	\item[hyper]\ 
	 \begin{description}
	 	\item[theta]\ 
	 	 \begin{description}
	 	 	\item[name] log precision
	 	 	\item[short.name] prec
	 	 	\item[prior] loggamma
	 	 	\item[param] 1 5e-05
	 	 	\item[initial] 4
	 	 	\item[fixed] FALSE
	 	 	\item[to.theta] \verb|function(x) log(x)|
	 	 	\item[from.theta] \verb|function(x) exp(x)|
	 	 \end{description}
	 \end{description}
	\item[constr] TRUE
	\item[nrow.ncol] FALSE
	\item[augmented] FALSE
	\item[aug.factor] 1
	\item[aug.constr] 
	\item[n.div.by] 
	\item[n.required] FALSE
	\item[set.default.values] FALSE
	\item[pdf] rw1
\end{description}

\paragraph{The ``RW2'' model for the hazard}
\begin{description}
	\item[hyper]\ 
	 \begin{description}
	 	\item[theta]\ 
	 	 \begin{description}
	 	 	 \item[ name ] log precision 
	 	 	 \item[ short.name ] prec 
	 	 	 \item[ prior ] loggamma 
	 	 	 \item[ param ] 1 5e-05 
	 	 	 \item[ initial ] 4 
	 	 	 \item[ fixed ] FALSE 
	 	 	 \item[ to.theta ] \verb|| 
	 	 	 \item[ from.theta ] \verb|| 
	 	 \end{description}
	 \end{description}
	 \item[ constr ] TRUE 
	 \item[ nrow.ncol ] FALSE 
	 \item[ augmented ] FALSE 
	 \item[ aug.factor ] 1 
	 \item[ aug.constr ]  
	 \item[ n.div.by ]  
	 \item[ n.required ] FALSE 
	 \item[ set.default.values ] FALSE 
	 \item[ pdf ] rw2 
\end{description}



\subsection*{Example}

In the following example we estimate the baseline hazard in a
simulated case \verbatiminput{example-cox.R}

\subsection*{Notes}

\begin{itemize}
\item The Cox model can be used only for uncensored or right censored
    data.
\item The model for the piecewise constant baseline hazard is
    specified through $\texttt{control.hazard}$
\item A general frame work to represent time is given by
    $\texttt{inla.surv}$
\item If the observed times $y$ are large/huge, then this can cause
    numerical overflow in the likelihood routines giving error
    messages like
\begin{verbatim}
        file: smtp-taucs.c  hgid: 891deb69ae0c  date: Tue Nov 09 22:34:28 2010 +0100
        Function: GMRFLib_build_sparse_matrix_TAUCS(), Line: 611, Thread: 0
        Variable evaluates to NAN/INF. This does not make sense. Abort...
\end{verbatim}
    If you encounter this problem, try to scale the observatios,
    \verb|time = time / max(time)| or similar, before running
    \verb|inla()|.
\end{itemize}


\end{document}


% LocalWords:  np Hyperparameters Ntrials

%%% Local Variables: 
%%% TeX-master: t
%%% End: 
