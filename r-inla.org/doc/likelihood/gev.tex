\documentclass[a4paper,11pt]{article}
\usepackage[scale={0.8,0.9},centering,includeheadfoot]{geometry}
\usepackage{amstext}
\usepackage{amsmath}
\usepackage{verbatim}

\begin{document}
\section*{Generalised Extreme Value (GEV) distribution}

\subsection*{Parametrisation}

The GEV distribution is defined through the cummulative distribution
function
\begin{displaymath}
    F(y; \eta, \tau, \xi) =
    \exp\left(
      - \left[ 1 + \xi \sqrt{\tau s} (y-\eta)\right]^{-1/\xi}
    \right)
\end{displaymath}
for
\begin{displaymath}
    1 + \xi \sqrt{\tau s} (y-\eta) > 0
\end{displaymath}
and for a continuously response $y$ where
\begin{description}
\item[$\eta$:] is the linear predictor
\item[$\tau$:] is the ``precision''
\item[$s$:] is a fixed scaling, $s>0$.    
\end{description}

\subsection*{Link-function}

The linear predictor is given in the parameterisation of the GEV
distribution.

\subsection*{Hyperparameters}

The GEV-models has two hyperparameters.
The ``precision'' is represented as
\begin{displaymath}
    \theta_{1} = \log \tau
\end{displaymath}
and the prior is defined on $\theta_{1}$.  The shape parameter $\xi$
is represented as
\begin{displaymath}
    \xi = \xi_{s}\theta_{2}
\end{displaymath}
where $\xi_{s} > 0$ is a \emph{chosen fixed scaling}, and the prior is
defined on $\theta_{2}$\footnote{The $\xi_{s}$ parameter is there for
    numerical reasons only, as the natural ``scale'' of $\xi$ is
    small, and the scaling makes the natural scale of $\theta_{2}$
    similar to other $\theta$'s. The output from INLA reports the
    parameter $\xi$.}

\subsection*{Specification}

\begin{itemize}
\item $\text{family}=\texttt{gev}$
\item Required arguments: $y$ and $s$ (keyword \texttt{scale})
\item The scaling $\xi_{s}$ is given by the argument
    \texttt{gev.xi.scale} and is default set to $0.01$.
\end{itemize}
The weights has default value 1.

\subsubsection*{Hyperparameter spesification and default values}
%% DO NOT EDIT!
%% This file is generated automatically from models.R
\begin{description}
	\item[hyper]\ 
	 \begin{description}
	 	\item[theta1]\ 
	 	 \begin{description}
	 	 	\item[name] log precision
	 	 	\item[short.name] prec
	 	 	\item[initial] 4
	 	 	\item[fixed] FALSE
	 	 	\item[prior] loggamma
	 	 	\item[param] 1 5e-05
	 	 	\item[to.theta] \verb|function(x) log(x)|
	 	 	\item[from.theta] \verb|function(x) exp(x)|
	 	 \end{description}
	 	\item[theta2]\ 
	 	 \begin{description}
	 	 	\item[name] gev parameter
	 	 	\item[short.name] gev
	 	 	\item[initial] 0
	 	 	\item[fixed] FALSE
	 	 	\item[prior] gaussian
	 	 	\item[param] 0 25
	 	 	\item[to.theta] \verb|function(x) x|
	 	 	\item[from.theta] \verb|function(x) x|
	 	 \end{description}
	 \end{description}
	\item[survival] FALSE
	\item[discrete] FALSE
	\item[link] default identity
	\item[status] experimental
	\item[pdf] gev
\end{description}


\subsection*{Example}

In the following example, we estimate the parameters of the GEV
distribution on some simulated data.
%%
\verbatiminput{example-gev.R}

\subsection*{Notes}

None.


\end{document}


% LocalWords:  np Hyperparameters Ntrials gaussian

%%% Local Variables: 
%%% TeX-master: t
%%% End: 
