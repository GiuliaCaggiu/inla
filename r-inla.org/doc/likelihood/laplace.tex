\documentclass[a4paper,11pt]{article}
\usepackage[scale={0.8,0.9},centering,includeheadfoot]{geometry}
\usepackage{amstext}
\usepackage{amsmath}
\usepackage{verbatim}

\begin{document}

\section*{Asymmetric Laplace}

\subsection*{Parametrisation}

The asymmetric Laplace distribution is
\begin{displaymath}
    f(y) = \delta\tau(1 - \tau)\exp\{-\delta\rho_\tau(y - \mu)\}
\end{displaymath}
for continuously responses $y$ where $\rho_\tau(u) = \{\tau - I(u<0)\}u$ is the so-called check function in Koenker and Bassett (1978), and
\begin{description}
\item[$\mu$:] is the the location parameter ($-\infty < \mu < \infty$)
\item[$\tau$:] is the fixed skewness parameter ($0<\tau<1$)
\item[$\delta$:] is the inverse scale parameter ($\delta > 0$).
\end{description}

\subsection*{Scale mixtures of normal representation}
The asymmetric Laplace random variable $y$ can be represented as follows:
\begin{displaymath}
 y = \mu + \xi w + \sigma\sqrt{w/\delta} z,
\end{displaymath}
where $\xi = \frac{1 - 2\tau}{\tau(1-\tau)}$ and $\sigma^2 = \frac{2}{\tau(1-\tau)}$ are two scalars depending on $\tau$. The random variables $w > 0$ and $z$ are independent and have exponential distribution with mean $\delta^{-1}$ and standard normal distribution, respectively. As a result, $y$ has the following hierarchical structure:
\begin{displaymath}
y\mid w\sim N\left(\mu + \xi w, \sigma^2 \delta^{-1}w\right)
\quad \mbox{and} \quad
w \sim \mbox{Exp}(\delta).
\end{displaymath}

\subsection*{Approximating check function}
The log likelihood of asymmetric Laplace distribution has zero second-order derivative everywhere. To implement INLA, we approximate the check function as follows:
\begin{displaymath}
\tilde\rho_{\tau,\gamma}(u)=\bigg\{
\begin{array}{ll}
\gamma^{-1}\log(\cosh(\tau\gamma |u|)) & \mbox{if}\,\, u\ge 0\\
\gamma^{-1}\log(\cosh((1-\tau)\gamma |u|)) & \mbox{if}\,\, u< 0,
\end{array}
\end{displaymath}
where the parameter $\gamma>0$ is fixed and precision of the approximation increases as $\gamma\rightarrow\infty$.

\subsection*{Link-function}

The location parameter is linked to the linear predictor by
\begin{displaymath}
    \mu = \eta
\end{displaymath}

\subsection*{Hyperparameters}

The prior is defined on inverse scale $\delta$.

\subsection*{Specification}

\begin{itemize}
\item $\text{family}=\texttt{laplace}$
%\item Required arguments: $y$ and $w$ (keyword \texttt{weights}). The
%    weights has default value 1.
%\item Optional control arguments: \texttt{sn.shape.max}. Default value is
%    $5.0$.
\end{itemize}

\subsubsection*{Hyperparameter spesification and default values}
%\begin{description}
	\item[hyper]\ 
	 \begin{description}
	 	\item[theta1]\ 
	 	 \begin{description}
	 	 	 \item[ name ] log inverse scale 
	 	 	 \item[ short.name ] iscale 
	 	 	 \item[ initial ] 4 
	 	 	 \item[ fixed ] FALSE 
	 	 	 \item[ prior ] loggamma 
	 	 	 \item[ param ] 1 5e-05 
	 	 \end{description}
	 	\item[theta2]\ 
	 	 \begin{description}
	 	 	 \item[ name ] logit skewness 
	 	 	 \item[ short.name ] skew 
	 	 	 \item[ initial ] 4 
	 	 	 \item[ fixed ] FALSE 
	 	 	 \item[ prior ] gaussian 
	 	 	 \item[ param ] 0 10 
	 	 	 \item[ to.theta ] \verb|function(x, shape.max = 1) log((1+x/shape.max)/(1-x/shape.max))| 
	 	 	 \item[ from.theta ] \verb|function(x, shape.max = 1) shape.max*(2*exp(x)/(1+exp(x))-1)| 
	 	 \end{description}
	 \end{description}
	 \item[ survival ] FALSE 
	 \item[ discrete ] FALSE 
	 \item[ link ] default identity 
	 \item[ pdf ] sn 
\end{description}




\subsection*{Example}

%This is a simulated example requiring the package \verb@sn@.
%\verbatiminput{example-sn.R}
%
\subsection*{Notes}

None.


\end{document}


% LocalWords:  np Hyperparameters Ntrials gaussian

%%% Local Variables: 
%%% TeX-master: t
%%% End: 
