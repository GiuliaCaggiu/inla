\documentclass[a4paper,11pt]{article}
\usepackage[scale={0.8,0.9},centering,includeheadfoot]{geometry}
\usepackage{amstext}
\usepackage{amsmath}
\usepackage{verbatim}

\begin{document}
\section*{Negative Binomial}

\subsection*{Parametrisation}

The negative Binomial distribution is
\begin{displaymath}
    \text{Prob}(y) = \frac{\Gamma(y+n)}{\Gamma(n) \Gamma(y+1)} p^n (1-p)^y
\end{displaymath}
for responses $y=0, 1, 2, \ldots$, where
\begin{description}
\item[$n$:] number of successful trials, or dispersion
    parameter. Must be strictly positive, need not be integer.
\item[$p$:] probability of success in each trial.
\end{description}

\subsection*{Link-function}

The mean and variance of $y$ are given as
\begin{displaymath}
    \mu = n \frac{1-p}{p} \qquad\text{and}\qquad \sigma^{2} = \mu(1 + \frac{\mu}{n})
\end{displaymath}
and the mean is linked to the linear predictor by
\begin{displaymath}
    \mu = E \exp(\eta)
\end{displaymath}
where the hyperparameter $n$ (or the \emph{size}) plays the role of an
overdispersion parameter. $E$ represents knows constant and $\log(E)$
is the offset of $\eta$.

\subsection*{Hyperparameters}

The overdispersion parameter $n$ is represented as
\begin{displaymath}
    \theta = \log(n)
\end{displaymath}
and the prior is defined on $\theta$. 

\subsection*{Specification}

\begin{itemize}
\item $\text{family}=\texttt{nbinomial}$
\item Required arguments: $y$ and $E$ (default $E=1$).
\end{itemize}

\subsubsection*{Hyperparameter spesification and default values}
%% DO NOT EDIT!
%% This file is generated automatically from models.R
\begin{description}
	\item[hyper]\ 
	 \begin{description}
	 	\item[theta]\ 
	 	 \begin{description}
	 	 	\item[name] size
	 	 	\item[short.name] size
	 	 	\item[initial] 2.30258509299405
	 	 	\item[fixed] FALSE
	 	 	\item[prior] loggamma
	 	 	\item[param] 1 1
	 	 	\item[to.theta] \verb|function(x) log(x)|
	 	 	\item[from.theta] \verb|function(x) exp(x)|
	 	 \end{description}
	 \end{description}
	\item[survival] FALSE
	\item[discrete] TRUE
	\item[link] default log
	\item[pdf] nbinomial
\end{description}



\subsection*{Example}

In the following example we estimate the parameters in a simulated
example with negative binomial responses and assign the hyperparameter
$\theta$ a Gaussian prior with mean $0$ and precision $0.01$
\verbatiminput{example-nbinomial.R}

\subsection*{Notes}

As $n\rightarrow\infty$, the negative Binomial converges to the
Poisson distribution. For numerical reasons, if $n$ is too large:
\begin{displaymath}
    \frac{\mu}{n} < 10^{-4},
\end{displaymath}
then the Poisson limit is used.


\end{document}


% LocalWords:  hyperparameter overdispersion Hyperparameters nbinomial

%%% Local Variables: 
%%% TeX-master: t
%%% End: 
