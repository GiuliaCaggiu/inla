\documentclass[a4paper,11pt]{article}
\usepackage[scale={0.8,0.9},centering,includeheadfoot]{geometry}
\usepackage{amstext}
\usepackage{amsmath}
\usepackage{verbatim}

\begin{document}
\section*{Simplex}

\subsection*{Parametrisation}

The Simplex distribution has the following density
\begin{displaymath}
\pi(y)=\frac{\sqrt{(s\tau)}}{\sqrt{2\pi[y(1-y)]^3}}\exp\left\{\frac{-(s\tau)
      (y-\mu)^2}{2y(1-y)\mu^2(1-\mu)^2} \right\}
\end{displaymath}
has has a continuously responses $ 0<y<1$ where
\begin{itemize}
\item[$\mu:$] is the mean,
\item[$\tau:$] is a precision parameter, and
\item[$s$:] is a fixed scaling, $s>0$.    
\end{itemize}
For the simplex distribution we have 
\begin{displaymath}
E(y) = \mu 
\end{displaymath}

\subsection*{Link-function}
The linear predictor $\eta$ is linked to the mean $\mu$ using a
default logit-link,
\begin{displaymath}
    \mu = \frac{\exp{(\eta)}}{1+\exp{(\eta)}}.
\end{displaymath}

\subsection*{Hyperparameter}

The hyperparameter is the precision parameter $\tau$, which is
represented as
\begin{displaymath}
    \tau = \exp(\theta)
\end{displaymath}
and the prior is defined on $\theta$. 

\subsection*{Specification}

\begin{itemize}
\item $\text{family}=\texttt{simplex}$
\item Required arguments: $y$.
\end{itemize}

\subsubsection*{Hyperparameter spesification and default values}
%% DO NOT EDIT!
%% This file is generated automatically from models.R
\begin{description}
	\item[hyper]\ 
	 \begin{description}
	 	\item[theta]\ 
	 	 \begin{description}
	 	 	\item[name] log precision
	 	 	\item[short.name] prec
	 	 	\item[initial] 4
	 	 	\item[fixed] FALSE
	 	 	\item[prior] loggamma
	 	 	\item[param] 1 5e-05
	 	 	\item[to.theta] \verb!function(x) log(x)!
	 	 	\item[from.theta] \verb!function(x) exp(x)!
	 	 \end{description}
	 \end{description}
	\item[survival] FALSE
	\item[discrete] FALSE
	\item[link] default logit probit cloglog
	\item[pdf] simplex
\end{description}

 
 
\subsection*{Example}

In the following example we estimate the parameters in a simulated
example.
\verbatiminput{example-simplex.R}
 
\subsection*{Notes}
 
 None.

\end{document}


% LocalWords:  hyperparameter overdispersion Hyperparameters nbinomial

%%% Local Variables: 
%%% TeX-master: t
%%% End: 
