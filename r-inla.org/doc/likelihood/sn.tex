\documentclass[a4paper,11pt]{article}
\usepackage[compat2]{geometry}
\usepackage{amstext}
\usepackage{amsmath}
\usepackage{verbatim}

\begin{document}
\section*{Skew-Normal}

\subsection*{Parametrisation}

The Skew-Normal distribution is
\begin{displaymath}
    f(y) = 2\frac{\sqrt{w\tau}}{\sqrt{2\pi}} \exp\left( -\frac{1}{2}
      w\tau \left(y-\mu\right)^{2}\right) \; \Phi(a\; a_{\text{max}} \left[w\tau \left(y-\mu\right)\right])
\end{displaymath}
for continuously responses $y$ where $\Phi(\cdot)$ is the cummulative
distribution function for a standard Normal, and
\begin{description}
\item[$\mu$:] is the the location parameter
\item[$\tau$:] is the inverse scale
\item[$w$:] is a fixed weight, $w>0$,
\item[$a$:] is the shape parameter
\item[$a_{\text{max}}$:] is the (fixed) maximum value of the shape
    paramter (added for stability reasons). Default value is $5$.
\end{description}

\subsection*{Link-function}

The location parameter is linked to the linear predictor by
\begin{displaymath}
    \mu = \eta
\end{displaymath}

\subsection*{Hyperparameters}

The inverse scale is represented as
\begin{displaymath}
    \theta_{1} = \log \tau
\end{displaymath}
and the prior is defined on $\theta_{1}$. 

The shape parameter is 
\begin{displaymath}
    a = 2 \frac{\exp(\theta_{2})}{1+\exp(\theta_{2})}-1
\end{displaymath}
and the prior is defined on $\theta_{2}$. 

\subsection*{Specification}

\begin{itemize}
\item $\text{family}=\texttt{sn}$
\item Required arguments: $y$ and $w$ (keyword \texttt{weights}). The
    weights has default value 1.
\item Optional control arguments: \texttt{sn.shape.max}. Default value is
    $5.0$.
\end{itemize}


\subsection*{Example}

This is a simulated example requiring the package \verb@sn@.
\verbatiminput{example-sn.R}

\subsection*{Notes}

None.


\end{document}


% LocalWords:  np Hyperparameters Ntrials gaussian

%%% Local Variables: 
%%% TeX-master: t
%%% End: 
