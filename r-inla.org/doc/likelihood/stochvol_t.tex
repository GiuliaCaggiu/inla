\documentclass[a4paper,11pt]{article}
\usepackage[compat2]{geometry}
\usepackage{amstext}
\usepackage{listings}
\begin{document}

\section*{Student-$t$ model for Stochastic volatility}

\subsection*{Parametrization}

The Student-$t$ likelihood for stochastic volatility models is defined as:
\[
\pi(y |\eta )=\sigma \epsilon 
\]
where
\[
\epsilon \sim T_{\nu}
\]
and $T_{\nu}$ is a Student-$t$ distribution with $\nu$ degrees of freedom {\it standardised} to that is has mean $0$ and variance $1$ for any value of $\nu$.

\subsection*{Link-function}

The scale parameter $\sigma $ is linked to the linear predictor $\eta $ as:
\[
\sigma =\exp(\eta /2)
\]

\subsection*{Hyperparameters}

The degrees of freedom $\nu$ is represented as
\[
\theta=\log(\nu-2)
\]
and the prior is defined on $\theta$

\subsection*{Specification}

\begin{itemize}
\item $\text{family}=\texttt{stochvol.t}$
\item Required argument: $y$.
\end{itemize}

\subsection*{Example}
In the following example we specify the likelihood for the stochastic volatility model to be Student-$t$ 
 
\begin{verbatim}
#simulated data
n=1000
phi=0.53
eta=rep(0.1,n)
for(i in 2:n)
  eta[i]=0.1+phi*(eta[i-1]-0.1)+rnorm(1,0,0.6)
y=exp(eta/2)*rt(n,df=4)
time=1:n
data=list(ret=y,time=time)

#fit the model
formula=ret~f(time,model="ar1",param=c(1,0.001,0,0.4))
result=inla(formula,family="stochvol.t",data=data)
hyper=inla.hyperpar(result)

\end{verbatim}

\subsection*{Notes}

None

\end{document}


% LocalWords: 

%%% Local Variables: 
%%% TeX-master: t
%%% End: 
