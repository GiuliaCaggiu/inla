\documentclass[a4paper,11pt]{article}
\usepackage[scale={0.8,0.9},centering,includeheadfoot]{geometry}
\usepackage{amstext}
\usepackage{listings}
\begin{document}

\section*{Student-$t$ model for Stochastic volatility}

\subsection*{Parametrization}

The Student-$t$ likelihood for stochastic volatility models is defined as:
\[
\pi(y |\eta )=\sigma \epsilon 
\]
where
\[
\epsilon \sim T_{\nu}
\]
and $T_{\nu}$ is a Student-$t$ distribution with $\nu$ degrees of freedom {\it standardised} to that is has mean $0$ and variance $1$ for any value of $\nu$.

\subsection*{Link-function}

The scale parameter $\sigma $ is linked to the linear predictor $\eta $ as:
\[
\sigma =\exp(\eta /2)
\]

\subsection*{Hyperparameters}

The degrees of freedom $\nu$ is represented as
\[
\theta=\log(\nu-2)
\]
and the prior is defined on $\theta$

\subsection*{Specification}

\begin{itemize}
\item $\text{family}=\texttt{stochvol.t}$
\item Required argument: $y$.
\end{itemize}

\subsubsection*{Hyperparameter spesification and default values}
\begin{description}
	\item[hyper]\ 
	 \begin{description}
	 	\item[theta]\ 
	 	 \begin{description}
	 	 	 \item[ name ] degrees of freedom 
	 	 	 \item[ short.name ] dof 
	 	 	 \item[ initial ] 4 
	 	 	 \item[ fixed ] FALSE 
	 	 	 \item[ prior ] loggamma 
	 	 	 \item[ param ] c(1, 0.5) 
	 	 \end{description}
	 \end{description}
	 \item[ survival ] FALSE 
	 \item[ discrete ] FALSE 
\end{description}


\subsection*{Example}
In the following example we specify the likelihood for the stochastic
volatility model to be Student-$t$
 
\begin{verbatim}
n=1000
x = 0.1 * arima.sim(n = n, model = list(ar = 0.9))
y=exp(x/2)*rt(n,df=6)
time=1:n
data=data.frame(y,time)

formula=y~f(time, model="ar1")+1
result=inla(formula,family="stochvol.t",data=data)
## sometimes we need to add
## control.inla = list(cmin = 1e-2)
## to make it converge
hyper=inla.hyperpar(result)
\end{verbatim}

\subsection*{Notes}

None

\end{document}


% LocalWords: 

%%% Local Variables: 
%%% TeX-master: t
%%% End: 
