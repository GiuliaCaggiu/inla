\documentclass[a4paper,11pt]{article}
\usepackage[scale={0.8,0.9},centering,includeheadfoot]{geometry}
\usepackage{amstext}
\usepackage{listings}
\begin{document}

\section*{Student-$t$}

\subsection*{Parametrization}

The Student-$t$ likelihood is defined so that 
\[
\sqrt{s\ \tau}(y - \eta)\sim T_{\nu}
\]
for continuous response $y$ where
\begin{itemize}
\item[$\tau$]: is the precision parameter
\item[$s$]: is a fixed scaling $s>0$
\item[$\eta$]: is the linear predictor
\item[$T_{\nu}$]: is a standardized Student-$t$ with $\nu$ degrees of
    freedom such that its variace is $1$ for any value of $\nu$.
\end{itemize}
\subsection*{Link-function}

Identity

\subsection*{Hyperparameters}

This likelihood has to hyperparameters
\begin{eqnarray*}
    \theta_1 &=& \log(\tau)\\
    \theta_2&=&\log(\nu-2)
\end{eqnarray*}
and the prior is defined on $\theta=(\theta_1,\theta_2)$. 

\subsection*{Specification}

\begin{itemize}
\item $\text{family}=\texttt{T}$
\item Required argument: $y$ and $s$ (keyword {\tt scale}, default
    to 1).
\end{itemize}

\subsubsection*{Hyperparameter spesification and default values}
\begin{description}
	\item[hyper]\ 
	 \begin{description}
	 	\item[theta1]\ 
	 	 \begin{description}
	 	 	 \item[ name ] log precision 
	 	 	 \item[ short.name ] prec 
	 	 	 \item[ initial ] 3 
	 	 	 \item[ fixed ] FALSE 
	 	 	 \item[ prior ] loggamma 
	 	 	 \item[ param ] 1 5e-05 
	 	 	 \item[ to.theta ] \verb|function(x) log(x)| 
	 	 	 \item[ from.theta ] \verb|function(x) exp(x) | 
	 	 \end{description}
	 	\item[theta2]\ 
	 	 \begin{description}
	 	 	 \item[ name ] log degrees of freedom 
	 	 	 \item[ short.name ] dof 
	 	 	 \item[ initial ] 0 
	 	 	 \item[ fixed ] FALSE 
	 	 	 \item[ prior ] loggamma 
	 	 	 \item[ param ] 1 0.5 
	 	 	 \item[ to.theta ] \verb|function(x) log(x-2)| 
	 	 	 \item[ from.theta ] \verb|function(x) 2+exp(x)| 
	 	 \end{description}
	 \end{description}
	 \item[ survival ] FALSE 
	 \item[ discrete ] FALSE 
\end{description}


\subsection*{Example}
\begin{verbatim}
#simulate data
n=100
phi=0.85
mu=0.5
eta=rep(0,n)
for(i in 2:n)
eta[i]=mu+phi*(eta[i-1]-mu)+rnorm(1)
nu=3
t=rt(n,df=nu)
y=eta+t/(sqrt(nu/(nu-2)))
data=list(y=y,z=seq(1:n))
#define the model and fit
formula=y~f(z,model="ar1")
result=inla(formula,family="T",data=data)
\end{verbatim}

\subsection*{Notes}
None

\end{document}


% LocalWords: 

%%% Local Variables: 
%%% TeX-master: t
%%% End: 
