\documentclass[a4paper,11pt]{article}
\usepackage[scale={0.8,0.9},centering,includeheadfoot]{geometry}
\usepackage{amstext}
\usepackage{amsmath}
\usepackage{verbatim}
\newcommand{\vect}[1]{\boldsymbol{#1}}
\begin{document}
\section*{Weibull}

\subsection*{Parametrisation}

The Weibull distribution is
\begin{displaymath}
    f(y) = \alpha y^{\alpha-1}
    \lambda\exp( - \lambda  y^{\alpha}),
    \qquad \alpha>0, \qquad \lambda>0
\end{displaymath}
where
\begin{description}
\item[$\alpha$:] shape parameter.
\end{description}
In survival analysis, models are generally specified through the
hazard function. For Weibull model the hazard function is:
\begin{displaymath}
    h(y)  = \alpha y^{\alpha-1} \lambda
\end{displaymath}

\subsection*{Link-function}

The parameter $\lambda$ is linked to the linear predictor as:
\[
\lambda = \exp(\eta)
\]
\subsection*{Hyperparameters}

The $\alpha$ parameter is represented as
\[
\theta = \log\alpha
\]
and the prior is defined on $\theta$.

\subsection*{Specification}

\begin{itemize}
\item $\text{family}=\texttt{weibull}$
\item Required arguments: $y$ (to be given in a format by using
    $\texttt{inla.surv()}$ function )
\end{itemize}

\subsubsection*{Hyperparameter spesification and default values}
\documentclass[a4paper,11pt]{article}
\usepackage[scale={0.8,0.9},centering,includeheadfoot]{geometry}
\usepackage{amstext}
\usepackage{amsmath}
\usepackage{verbatim}
\newcommand{\vect}[1]{\boldsymbol{#1}}
\begin{document}
\section*{Weibull}

\subsection*{Parametrisation}

The Weibull distribution is
\begin{displaymath}
    \text{Prob}(y) = \alpha y^{\alpha-1} \lambda\exp( - \lambda  y^{\alpha}), \qquad \alpha>0, \qquad \lambda>0
\end{displaymath}
where
\begin{description}
\item[$\alpha$:] shape parameter.
\end{description}
In survival analysis, models are generally specified through the
hazard function. For Weibull model the hazard function is:
\begin{displaymath}
    h(y)  = \alpha y^{\alpha-1} \lambda
\end{displaymath}

\subsection*{Link-function}

The parameter $\lambda$ is linked to the linear predictor as:
\[
\lambda = \exp(\eta)
\]
\subsection*{Hyperparameters}

The $\alpha$ parameter is represented as
\[
\theta = \log\alpha
\]
and the prior is defined on $\theta$.

\subsection*{Specification}

\begin{itemize}
\item $\text{family}=\texttt{weibull}$
\item Required arguments: $y$ (to be given in a format by using
    $\texttt{inla.surv()}$ function )
\end{itemize}

\subsection*{Example}

In the following example we estimate the parameters in a simulated
case \verbatiminput{example-weibull.R}

\subsection*{Notes}

\begin{itemize}
\item Weibull model can be used for right censored, left censored,
    interval censored data.
\item A general frame work to represent time is given by
    $\texttt{inla.surv}$
\item If the observed times $y$ are large/huge, then this can cause
    numerical overflow in the likelihood routines giving error
    messages like
\begin{verbatim}
        file: smtp-taucs.c  hgid: 891deb69ae0c  date: Tue Nov 09 22:34:28 2010 +0100
        Function: GMRFLib_build_sparse_matrix_TAUCS(), Line: 611, Thread: 0
        Variable evaluates to NAN/INF. This does not make sense. Abort...
\end{verbatim}
    If you encounter this problem, try to scale the observatios,
    \verb|time = time / max(time)| or similar, before running
    \verb|inla()|. 
\end{itemize}


\end{document}


% LocalWords:  np Hyperparameters Ntrials

%%% Local Variables: 
%%% TeX-master: t
%%% End: 



\subsection*{Example}

In the following example we estimate the parameters in a simulated
case \verbatiminput{example-weibull.R}

\subsection*{Notes}

\begin{itemize}
\item Weibull model can be used for right censored, left censored,
    interval censored data.
\item A general frame work to represent time is given by
    $\texttt{inla.surv}$
\item If the observed times $y$ are large/huge, then this can cause
    numerical overflow in the likelihood routines giving error
    messages like
\begin{verbatim}
        file: smtp-taucs.c  hgid: 891deb69ae0c  date: Tue Nov 09 22:34:28 2010 +0100
        Function: GMRFLib_build_sparse_matrix_TAUCS(), Line: 611, Thread: 0
        Variable evaluates to NAN/INF. This does not make sense. Abort...
\end{verbatim}
    If you encounter this problem, try to scale the observatios,
    \verb|time = time / max(time)| or similar, before running
    \verb|inla()|. 
\end{itemize}


\end{document}


% LocalWords:  np Hyperparameters Ntrials

%%% Local Variables: 
%%% TeX-master: t
%%% End: 
