\documentclass[a4paper,11pt]{article}
\usepackage[scale={0.8,0.9},centering,includeheadfoot]{geometry}
\usepackage{amstext}
\usepackage{amsmath}
\usepackage{verbatim}

\begin{document}
\section*{Zero-inflated models: Poisson and Binomial}

\subsection*{Parametrisation}

There is support two types of zero-inflated models, which we name type
$0$ and type $1$. These are defined for both the Binomial, the Poisson
and the negative Binomial likelihood. For simplicity we will describe
only the Poisson as the other two cases are similar.

\subsubsection*{Type 0}

The (type 0) likelihood is defined as
\begin{displaymath}
    \text{Prob}(y \mid \ldots ) = p \times 1_{[y=0]} +
    (1-p)\times \text{Poisson}(y \mid y > 0)
\end{displaymath}
where $p$ is a hyperparameter where
\begin{displaymath}
    p = \frac{\exp(\theta)}{1+\exp(\theta)}
\end{displaymath}
and $\theta$ is the internal representation of $p$; meaning that the
initial value and prior is given for $\theta$. This is model is called
\texttt{zeroinflatedpoisson0} (and \texttt{zeroinflatedbinomial0}).

\subsection*{Type 1}

The (type 1) likelihood is defined as
\begin{displaymath}
    \text{Prob}(y \mid \ldots ) = p \times 1_{[y=0]} +
    (1-p)\times \text{Poisson}(y)
\end{displaymath}
where $p$ is a hyperparameter where
\begin{displaymath}
    p = \frac{\exp(\theta)}{1+\exp(\theta)}
\end{displaymath}
and $\theta$ is the internal representation of $p$; meaning that the
initial value and prior is given for $\theta$. This is model is called
\texttt{zeroinflatedpoisson1} (and \texttt{zeroinflatedbinomial1}).

\subsection*{Link-function}

As for the Poisson, the Binomial and the negative Binomial.

\subsection*{Hyperparameters}

For Poisson and the Binomial, there is one hyperparameter; where
\begin{displaymath}
    p = \frac{\exp(\theta)}{1+\exp(\theta)}
\end{displaymath}
and the prior and initial value is is given for $\theta$.

For the negative Binomial, there are two hyperparameters.  The
overdispersion parameter $n$ is represented as
\begin{displaymath}
    \theta_{1} = \log(n)
\end{displaymath}
and the prior is defined on $\theta_{1}$. The zero-inflation parameter
$p$, is represented as
\begin{displaymath}
    p = \frac{\exp(\theta_{2})}{1+\exp(\theta_{2})}
\end{displaymath}
and the prior and initial value is is given for $\theta_{2}$.

\subsection*{Specification}

\begin{itemize}
\item $\text{family}=\texttt{zeroinflatedbinomial0}$
\item $\text{family}=\texttt{zeroinflatedbinomial1}$
\item $\text{family}=\texttt{zeroinflatednbinomial0}$
\item $\text{family}=\texttt{zeroinflatednbinomial1}$
\item $\text{family}=\texttt{zeroinflatedpoisson0}$
\item $\text{family}=\texttt{zeroinflatedpoisson1}$
\item Required arguments: As for the Binomial, the negative Binomial
    and Poisson likelihood.
\end{itemize}

\subsubsection*{Hyperparameter spesification and default values}
\paragraph{Zeroinflated Binomial Type 0}
\begin{description}
	\item[hyper]\ 
	 \begin{description}
	 	\item[theta]\ 
	 	 \begin{description}
	 	 	 \item[ name ] logit probability 
	 	 	 \item[ short.name ] prob 
	 	 	 \item[ initial ] -1 
	 	 	 \item[ fixed ] FALSE 
	 	 	 \item[ prior ] gaussian 
	 	 	 \item[ param ] -1 0.2 
	 	 	 \item[ to.theta ] \verb|function(x) log(x/(1-x))| 
	 	 	 \item[ from.theta ] \verb|function(x) exp(x)/(1+exp(x))| 
	 	 \end{description}
	 \end{description}
	 \item[ survival ] FALSE 
	 \item[ discrete ] FALSE 
	 \item[ link ] default logit probit cloglog 
\end{description}


\paragraph{Zeroinflated Binomial Type 1}
\begin{description}
	\item[hyper]\ 
	 \begin{description}
	 	\item[theta]\ 
	 	 \begin{description}
	 	 	 \item[ name ] logit probability 
	 	 	 \item[ short.name ] prob 
	 	 	 \item[ initial ] -1 
	 	 	 \item[ fixed ] FALSE 
	 	 	 \item[ prior ] gaussian 
	 	 	 \item[ param ] -1 0.2 
	 	 	 \item[ to.theta ] \verb|function(x) log(x/(1-x))| 
	 	 	 \item[ from.theta ] \verb|function(x) exp(x)/(1+exp(x))| 
	 	 \end{description}
	 \end{description}
	 \item[ survival ] FALSE 
	 \item[ discrete ] FALSE 
	 \item[ link ] default logit probit cloglog 
\end{description}


\paragraph{Zeroinflated NegBinomial Type 0}
\begin{description}
	\item[hyper]\ 
	 \begin{description}
	 	\item[theta1]\ 
	 	 \begin{description}
	 	 	 \item[ name ] log size 
	 	 	 \item[ short.name ] size 
	 	 	 \item[ initial ] 2.30258509299405 
	 	 	 \item[ fixed ] FALSE 
	 	 	 \item[ prior ] loggamma 
	 	 	 \item[ param ] 1 1 
	 	 	 \item[ to.theta ] \verb|function(x) log(x)| 
	 	 	 \item[ from.theta ] \verb|function(x) exp(x)| 
	 	 \end{description}
	 	\item[theta2]\ 
	 	 \begin{description}
	 	 	 \item[ name ] logit probability 
	 	 	 \item[ short.name ] prob 
	 	 	 \item[ initial ] -1 
	 	 	 \item[ fixed ] FALSE 
	 	 	 \item[ prior ] gaussian 
	 	 	 \item[ param ] -1 0.2 
	 	 	 \item[ to.theta ] \verb|function(x) log(x/(1-x))| 
	 	 	 \item[ from.theta ] \verb|function(x) exp(x)/(1+exp(x))| 
	 	 \end{description}
	 \end{description}
	 \item[ survival ] FALSE 
	 \item[ discrete ] FALSE 
\end{description}


\paragraph{Zeroinflated NegBinomial Type 1}
\begin{description}
	\item[hyper]\ 
	 \begin{description}
	 	\item[theta1]\ 
	 	 \begin{description}
	 	 	 \item[ name ] log size 
	 	 	 \item[ short.name ] size 
	 	 	 \item[ initial ] 2.30258509299405 
	 	 	 \item[ fixed ] FALSE 
	 	 	 \item[ prior ] loggamma 
	 	 	 \item[ param ] 1 1 
	 	 	 \item[ to.theta ] \verb|function(x) log(x)| 
	 	 	 \item[ from.theta ] \verb|function(x) exp(x)| 
	 	 \end{description}
	 	\item[theta2]\ 
	 	 \begin{description}
	 	 	 \item[ name ] logit probability 
	 	 	 \item[ short.name ] prob 
	 	 	 \item[ initial ] -1 
	 	 	 \item[ fixed ] FALSE 
	 	 	 \item[ prior ] gaussian 
	 	 	 \item[ param ] -1 0.2 
	 	 	 \item[ to.theta ] \verb|function(x) log(x/(1-x))| 
	 	 	 \item[ from.theta ] \verb|function(x) exp(x)/(1+exp(x))| 
	 	 \end{description}
	 \end{description}
	 \item[ survival ] FALSE 
	 \item[ discrete ] FALSE 
\end{description}


\paragraph{Zeroinflated Poisson Type 0}
%% DO NOT EDIT!
%% This file is generated automatically from models.R
\begin{description}
	\item[hyper]\ 
	 \begin{description}
	 	\item[theta]\ 
	 	 \begin{description}
	 	 	\item[name] logit probability
	 	 	\item[short.name] prob
	 	 	\item[initial] -1
	 	 	\item[fixed] FALSE
	 	 	\item[prior] gaussian
	 	 	\item[param] -1 0.2
	 	 	\item[to.theta] \verb|function(x) log(x/(1-x))|
	 	 	\item[from.theta] \verb|function(x) exp(x)/(1+exp(x))|
	 	 \end{description}
	 \end{description}
	\item[survival] FALSE
	\item[discrete] FALSE
	\item[link] default log
	\item[pdf] zeroinflated
\end{description}


\paragraph{Zeroinflated Poisson Type 1}
%% DO NOT EDIT!
%% This file is generated automatically from models.R
\begin{description}
	\item[hyper]\ 
	 \begin{description}
	 	\item[theta]\ 
	 	 \begin{description}
	 	 	\item[name] logit probability
	 	 	\item[short.name] prob
	 	 	\item[initial] -1
	 	 	\item[fixed] FALSE
	 	 	\item[prior] gaussian
	 	 	\item[param] -1 0.2
	 	 	\item[to.theta] \verb|function(x) log(x/(1-x))|
	 	 	\item[from.theta] \verb|function(x) exp(x)/(1+exp(x))|
	 	 \end{description}
	 \end{description}
	\item[survival] FALSE
	\item[discrete] FALSE
	\item[link] default log
	\item[pdf] zeroinflated
\end{description}




\subsection*{Example}

In the following example we estimate the parameters in a simulated
example for both type 0 and type 1.
\subsubsection*{Poisson}

\verbatiminput{example-zero-inflated-poisson.R}

\subsubsection*{Binomial}

\verbatiminput{example-zero-inflated-binomial.R}

\subsection*{Notes}

None.

\subsection*{Extentions}

There are some extentions available which currently is only
implemented for the cases where its needed/requested.
\begin{description}
\item[Type 2] Is like Type 1 but where (for the Poisson)
    \begin{displaymath}
        p = 1-\left( \frac{E\exp(x)}{1 + E\exp(x)}\right)^{\alpha}
    \end{displaymath}
    where $\alpha > 0$ is the hyperparameter instead of $p$ (and
    $E\exp(x)$ is the mean).  Available for Poisson as
    \texttt{zeroinflatedpoisson2}, for binomial as
    \texttt{zeroinflatedbinomial2} and for the negative binomial as
    \texttt{zeroinflatednbinomial2}.

    The internal representation is $\theta = \log(\alpha)$ and prior
    is defined on $\log(\alpha)$.
\end{description}

\paragraph{Zeroinflated Poisson Type 2}
%% DO NOT EDIT!
%% This file is generated automatically from models.R
\begin{description}
	\item[hyper]\ 
	 \begin{description}
	 	\item[theta]\ 
	 	 \begin{description}
	 	 	\item[name] log alpha
	 	 	\item[short.name] a
	 	 	\item[initial] 0.693147180559945
	 	 	\item[fixed] FALSE
	 	 	\item[prior] gaussian
	 	 	\item[param] 0.693147180559945 1
	 	 	\item[to.theta] \verb!function(x) log(x)!
	 	 	\item[from.theta] \verb!function(x) exp(x)!
	 	 \end{description}
	 \end{description}
	\item[survival] FALSE
	\item[discrete] FALSE
	\item[link] default log
	\item[pdf] zeroinflated
\end{description}


\paragraph{Zeroinflated Binomial Type 2}
\begin{description}
	\item[hyper]\ 
	 \begin{description}
	 	\item[theta]\ 
	 	 \begin{description}
	 	 	 \item[ name ] logit probability 
	 	 	 \item[ short.name ] prob 
	 	 	 \item[ initial ] -1 
	 	 	 \item[ fixed ] FALSE 
	 	 	 \item[ prior ] gaussian 
	 	 	 \item[ param ] -1 0.2 
	 	 	 \item[ to.theta ] \verb|function(x) log(x/(1-x))| 
	 	 	 \item[ from.theta ] \verb|function(x) exp(x)/(1+exp(x))| 
	 	 \end{description}
	 \end{description}
	 \item[ survival ] FALSE 
	 \item[ discrete ] FALSE 
	 \item[ link ] default logit probit cloglog 
\end{description}


\paragraph{Zeroinflated Negative Binomial Type 2}
\begin{description}
	\item[hyper]\ 
	 \begin{description}
	 	\item[theta1]\ 
	 	 \begin{description}
	 	 	 \item[ name ] log size 
	 	 	 \item[ short.name ] size 
	 	 	 \item[ initial ] 2.30258509299405 
	 	 	 \item[ fixed ] FALSE 
	 	 	 \item[ prior ] loggamma 
	 	 	 \item[ param ] 1 1 
	 	 	 \item[ to.theta ] \verb|function(x) log(x)| 
	 	 	 \item[ from.theta ] \verb|function(x) exp(x)| 
	 	 \end{description}
	 	\item[theta2]\ 
	 	 \begin{description}
	 	 	 \item[ name ] log alpha 
	 	 	 \item[ short.name ] a 
	 	 	 \item[ initial ] 0.693147180559945 
	 	 	 \item[ fixed ] FALSE 
	 	 	 \item[ prior ] gaussian 
	 	 	 \item[ param ] 2 1 
	 	 	 \item[ to.theta ] \verb|function(x) log(x)| 
	 	 	 \item[ from.theta ] \verb|function(x) exp(x)| 
	 	 \end{description}
	 \end{description}
	 \item[ survival ] FALSE 
	 \item[ discrete ] FALSE 
\end{description}


\end{document}


% LocalWords:  np Hyperparameters Ntrials gaussian hyperparameter

%%% Local Variables: 
%%% TeX-master: t
%%% End: 
% LocalWords:  zeroinflatedpoisson zeroinflatedbinomial
