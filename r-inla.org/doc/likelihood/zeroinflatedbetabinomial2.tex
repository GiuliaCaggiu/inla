\documentclass[a4paper,11pt]{article}
\usepackage[scale={0.8,0.9},centering,includeheadfoot]{geometry}
\usepackage{amstext}
\usepackage{amsmath}
%\usepackage{verbatim}

\begin{document}
\section*{Zero-inflated models: Beta-Binomial}

\subsection*{Parameterisation}

There is support for a further zero-inflated model of type $2$ (see
zero-inflated.pdf), the zero-inflated beta-binomial.  It is only
defined for type 2.

\subsubsection*{Type 2}

The likelihood is defined as

\begin{displaymath}
    \text{Prob}(y \mid \ldots ) = p \times 1_{[y=0]} +
    (1-p)\times \text{Beta-binomial}(y)
\end{displaymath}

where:
\begin{displaymath}
    p = 1-\left( \frac{\exp(x)}{1 + \exp(x)}\right)^{\alpha}
\end{displaymath}

\subsection*{Link-function}

As for the Binomial (see Zero-inflated.pdf).

\subsection*{Hyperparameters}

The Beta-binomial distribution has two arguments ($\beta_1$ \&
$\beta_2$) which we assume are a (specific) function of an underlying
hyperparameter ($\delta$) \& $x$. There is a further hyperparameter,
$\alpha$, governing zero-inflation where:

\vspace{5mm}

\noindent
The parameter controlling the degree of overdispersion, $\delta$, is
represented as
\begin{displaymath}
    \theta_{1} = \log(\delta)
\end{displaymath}
and the prior is defined on $\theta_{1}$.

\vspace{5mm}

\noindent
The zero-inflation parameter $\alpha$, is represented as

\begin{displaymath}
    \theta_2 = \log(\alpha)
\end{displaymath}
and the prior and initial value is is given for $\theta_{2}$.

\subsection*{Specification}

\begin{itemize}
\item $\text{family}=\texttt{zeroinflatedbetabinomial2}$
\item Required arguments: As for the zero-inflated-nbinomial2
    likelihood.
\end{itemize}

\subsubsection*{Hyperparameter spesification and default values}
\begin{description}
	\item[hyper]\ 
	 \begin{description}
	 	\item[theta1]\ 
	 	 \begin{description}
	 	 	 \item[ name ] log alpha 
	 	 	 \item[ short.name ] a 
	 	 	 \item[ initial ] 0.693147180559945 
	 	 	 \item[ fixed ] FALSE 
	 	 	 \item[ prior ] gaussian 
	 	 	 \item[ param ] 0.693147180559945 1 
	 	 	 \item[ to.theta ] \verb|| 
	 	 	 \item[ from.theta ] \verb|| 
	 	 \end{description}
	 	\item[theta2]\ 
	 	 \begin{description}
	 	 	 \item[ name ] beta 
	 	 	 \item[ short.name ] b 
	 	 	 \item[ initial ] 0 
	 	 	 \item[ fixed ] FALSE 
	 	 	 \item[ prior ] gaussian 
	 	 	 \item[ param ] 0 1 
	 	 	 \item[ to.theta ] \verb|| 
	 	 	 \item[ from.theta ] \verb|| 
	 	 \end{description}
	 \end{description}
	 \item[ survival ] FALSE 
	 \item[ discrete ] FALSE 
	 \item[ link ] default logit probit cloglog 
	 \item[ pdf ] zeroinflated 
\end{description}


\subsection*{Example}

In the following we estimate the parameters in a simulated example.

\begin{verbatim}
Example-zero-inflated-beta-binomial2.R

nx = 1000                 # number of x's to consider
n.trial = 20              # size of each binomial trial
x = rnorm(nx)             # generating x

delta = 10                   #hyperparameter 1
p = exp(1+x)/(1+exp(1+x))    #hyperparameter 2
alpha = 2                      #ZI parameter
q = p^alpha                    #prob presence

beta_1=delta*p                    #beta-bin parameter 1
beta_2=delta*(1-p)                #beta-bin parameter 2                                 
rb = rbeta(nx, beta_1, beta_2, ncp = 0)

        
y = rep(0,nx)                       #generating data                    
abs.pres = rbinom(nx,1,q)
y[abs.pres==1] = rbinom( sum(abs.pres>0), n.trial, rb[abs.pres==1])

formula = y ~ x +1
r = inla(formula, data = data.frame(x,y), family = "zeroinflatedbetabinomial2",
        control.data = list(hyper=list(a = list(prior = "flat", param=numeric(0)),
                                       b = list(prior = "flat", param=numeric(0)))),
        Ntrials = rep(n.trial, nx),
        verbose=TRUE)
\end{verbatim}
\end{document}
