\documentclass[a4paper,11pt]{article}
\usepackage[scale={0.8,0.9},centering,includeheadfoot]{geometry}
\usepackage{amstext}
\usepackage{listings}
\begin{document}

\section*{The PC prior for the correlation $\rho$ with $\rho=1$ as the
    base-model}

\subsection*{Parametrization}
This prior is the PC prior for the correlation $\rho$ where $\rho=1$
is the base-model. The density for $\rho$ is
\begin{displaymath}
    \pi(\rho) = \frac{\lambda \exp(-\lambda \mu(\rho))}{%%
        1-\exp(-\sqrt{2}\lambda)} \; J(\rho)
\end{displaymath}
where
\begin{displaymath}
    \mu(\rho) = \sqrt{1-\rho}
\end{displaymath}
and
\begin{displaymath}
    J(\rho) = \frac{1}{2\mu(\rho)}
\end{displaymath}
The parameter $\lambda$ is defined through
\begin{displaymath}
    \text{Prob}( \rho > u) = \alpha, \qquad -1 < u < 1, \quad
    \sqrt{\frac{1-u}{2}} <\alpha<1
\end{displaymath}
where $(u, \alpha{})$ are the parameters to this prior. The solution
is implicite
\begin{displaymath}
    \frac{\exp(-\lambda \sqrt{1-u})}{1-\exp(-\sqrt{2}\lambda)} = \alpha
\end{displaymath}
which explains why we have have
\begin{displaymath}
    \alpha > \mu(u)/\sqrt{2} = \sqrt{\frac{1-u}{2}}
\end{displaymath}
for a solution to exists with $\lambda > 0$. 
So for $u=1/2$ then $\alpha > 1/2$.

\subsection*{Specification}
This prior for the hyperparameters is specified inside the
\texttt{hyper}-spesification, as
\begin{center}
    \texttt{hyper = list(<theta> = list(prior="pc.rho1", param=c(<u>,<alpha>))))}
\end{center}

\subsection*{Example}

\subsection*{Notes}

See also functions \texttt{inla.pc.\{d,p,q,r\}rho1}

\end{document}


% LocalWords: 

%%% Local Variables: 
%%% TeX-master: t
%%% End: 
