\documentclass[a4paper,11pt]{article}
\usepackage[scale={0.8,0.9},centering,includeheadfoot]{geometry}
\usepackage{amstext}
\usepackage{listings}
\begin{document}

\section*{ Flat prior}

\subsection*{Parametrization}
The flat prior has density
\begin{equation}
\pi(\theta)\propto 1
\end{equation}
for continuous $\theta$.
 
\subsection*{Specification}
The flat prior for the hyperparameters is specified inside the {\tt
    f()} function as the following using the old style:
\begin{center}
    {\tt f(<whatever>,prior="flat")}
\end{center}
or better, the new style
\begin{center}
    {\tt f(<whatever>, hyper = list( <theta> = list(prior="flat",
        param = numeric())))}
\end{center}
In the case where there is one hyperparameter for that particular
f-model. In the case where we want to specify the prior for the
hyperparameter of an observation model, for example the negative Bino-
mial, the the prior spesification will appear inside the
control.family()-argument; see the following example for illustration.

\subsection*{Example}

In the following example we estimate the parameters in a simulated
example with gaussian responses and assign for the log-precision
$\log\tau$, a flat prior; see the Notes.
\begin{verbatim}
n=100
z=rnorm(n)
y=rnorm(n,z,1)
data=list(y=y,z=z)
formula=y~1+z
result=inla(formula,family="gaussian",data=data,control.family=list(prior="flat"))
\end{verbatim}

\subsection*{Notes}

The {\tt inla} program uses $\pi(\theta)=1$ for computations.  Note
that for precision $\tau$, a flat prior for $\log\tau$ corresponds to
the prior $1/\tau$ for $\tau$.

\end{document}


% LocalWords: 

%%% Local Variables: 
%%% TeX-master: t
%%% End: 
