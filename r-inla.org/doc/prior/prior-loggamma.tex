\documentclass[a4paper,11pt]{article}
\usepackage[compat2]{geometry}
\usepackage{amstext}
\usepackage{listings}
\begin{document}

\section*{ Log-gamma prior}

\subsection*{Parametrization}
The Gamma distribution has density
\begin{equation}
\pi(\tau)=\frac{b^a}{\Gamma(a)}\tau^{a-1}\exp(-b\ \tau)
\end{equation}
for positeve $\tau$ where:
\begin{description}
\item[$a>0$] is the shape parameter
\item[$b>0$] is the inverse-scale parameter
\end{description}
The mean of $\tau$ is $a/b$ and the variance is $a/b^2$.

The variable $\theta$ has a {\it log-Gamma} distribution if $\tau=\exp\theta$ has a Gamma distribution.

\subsection*{Specification}
The Log-Gamma prior for the hyperparameters is specified inside the {\tt f()} function as following:
\begin{center}
{\tt f(<whatever>,prior=loggamma,param=c(<a>,<b>))}
\end{center}
 In the case where there is one hyperparameter for that particular f-model. In the case where we want
to specify the prior for the hyperparameter of an observation model, for example the negative Gaussian, the the prior spesification will appear inside the control.data()-argument; see the following
example for illustration.

\subsection*{Example}

In the following example we estimate the parameters in a simulated example with gaussian
responses and assign the hyperparameter θ (the precision parameter), a log-Gamma prior with parameters $a=0.1$ and $b=0.1$


\begin{verbatim}
n=100
z=rnorm(n)
y=rnorm(n,z,1)

data=list(y=y,z=z)
formula=y~1+z
result=inla(formula,family="gaussian",data=data,
       control.data=list(prior="loggamma",param=c(0.1,0.1)))
\end{verbatim}

\subsection*{Notes}
None



\end{document}


% LocalWords: 

%%% Local Variables: 
%%% TeX-master: t
%%% End: 
