\documentclass[a4paper,11pt]{article}
\usepackage[scale={0.8,0.9},centering,includeheadfoot]{geometry}
\usepackage{amstext}
\usepackage{listings}
\begin{document}

\section*{Truncated Gaussian prior}

\subsection*{Parametrization}
This is a prior for a precision $\tau$ and defined as follows.  The
standard deviation $\sigma = 1/\sqrt{\tau}$ is Gaussian distributed
with mean $\mu$ and precision $\kappa$ but truncated to be positive.

\subsection*{Specification}
This prior for the hyperparameters is specified inside the {\tt
    f()} function as the following using the old style
\begin{center}
    {\tt f(<whatever>, prior="logtnormal", param=c(<mean $\mu$>, <precision $\kappa$>))}
\end{center}
or, better, the new style
\begin{center}
    {\tt f(<whatever>, hyper = list( <theta> = list(prior="logtnormal", param=c(<mean $\mu$>, <precision $\kappa$>))))}
\end{center}
Similar with "logtgaussian".

\subsection*{Example}

In the following example we estimate the parameters in a simulated
example with gaussian responses and assign for the precision
$\tau$, the above prior.
\begin{verbatim}
n=100
z=rnorm(n)
y=rnorm(n,z,1)
data=list(y=y,z=z)
formula=y~1+z
result=inla(formula, family="gaussian", data=data,
            control.data=list(hyper = list(prec = list(prior="logtnormal", param = c(0, 0.01)))))
\end{verbatim}

\end{document}


% LocalWords: 

%%% Local Variables: 
%%% TeX-master: t
%%% End: 
