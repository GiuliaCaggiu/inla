\documentclass[a4paper,11pt]{article}
\usepackage[scale={0.8,0.9},centering,includeheadfoot]{geometry}
\usepackage{amstext}
\usepackage{verbatim}
\begin{document}

\section*{``Table'': a tabulated prior}

This prior allow the user to submit a prior for $\theta$ in a
tabulated form, which is then interpolated to evaluate
$\log\pi(\theta)$ as a continous function of the corresponding
$\theta$.  Let
\begin{displaymath}
    \theta_{1}, \theta_{2}, \ldots, \theta_{m}
\end{displaymath}
be $m$ values for $\theta$ with corresponding log-prior density
\begin{displaymath}
    \log\pi(\theta_{1}),
    \log\pi(\theta_{2}), \ldots,
    \log\pi(\theta_{m}).
\end{displaymath}
To define this as a prior in \verb|R-INLA|, define \emph{one} object
of type \verb|character|, with content
\begin{displaymath}
    \text{table:}\;
    \theta_{1} \; \theta_{2}\;  \ldots \; \theta_{m} \;
    \log\pi(\theta_{1}) \;
    \log\pi(\theta_{2}) \; \ldots \;
    \log\pi(\theta_{m})
\end{displaymath}
and use this as the name for the prior. 

\subsection*{Example}
This example define a loggamma-prior as the prior for the
log-precision in three different ways.

\verbatiminput{example-table.R}

\subsection*{Notes}
\begin{itemize}
\item If the internal optimiser in \verb|R-INLA| needs to evaluate the
    (log-)prior outside the domain given, it will stop and give an error.
\end{itemize}
\end{document}


% LocalWords:  hyperparameters param gaussian hyperparameter univariate lgamma

%%% Local Variables: 
%%% TeX-master: t
%%% End: 
% LocalWords:  muparser
