\documentclass[11pt,a4paper]{article}
\usepackage{ifpdf}

\ifpdf
    \pdfadjustspacing=1
    \usepackage[compat2,pdftex]{geometry}
    %%To get a better output for pdf files substitute the fonts with
    %% \usepackage{lmodern}
    %%or, if you do not have the lm fonts installed:
    \usepackage{times}
    %%\usepackage{mathptmx}
    %%\usepackage[scaled=0.92]{helvet} % or [scaled=0.92], if you like
    %%\usepackage{courier}

    %% faar ikke colorlinks av....
    %%\usepackage[colorlinks=false,pdftex,
    %%    pdftitle={},%%
    %%    pdfauthor={},%%
    %%    baseurl={http://www.math.ntnu.no/~hrue}]{hyperref}
\else
    \usepackage[compat2,dvips]{geometry}
    \usepackage{times}
    %%\usepackage[colorlinks=false,dvips]{hyperref}
\fi


%%\usepackage[dvips]{graphicx}
\usepackage{pgf}
\usepackage{subfigure}




\usepackage{amsmath,amssymb}
\usepackage[active]{srcltx}


%%\usepackage{block}
%%\usepackage{my-showkeys}

\usepackage[colon,longnamesfirst]{natbib}

%%% columns centered by the `.' see the latex-companion page 129.
\usepackage{dcolumn}\newcolumntype{d}[1]{D{.}{.}{#1}}

\newtheorem{lemma}{Lemma}
\newtheorem{theorem}{Theorem}
\newtheorem{corrolary}{Corrolary}
\newtheorem{example}{Example}
\newcommand{\vect}[1]{\boldsymbol{#1}}

\def\qedbox{\ifvmode\else\unskip\fi~\penalty10000%%
    \hfill{\large$\blacksquare$}}

\def\Fig#1{Figure~\ref{#1}}
\def\Tab#1{Table~\ref{#1}}
\def\Thm#1{Theorem~\ref{#1}}
\def\Cor#1{Corrolary~\ref{#1}}
\def\Sec#1{Section~\ref{#1}}
\def\eref#1{(\ref{#1})}
\def\Eref#1{Eq.~(\ref{#1})}
\def\miscinfo#1#2{{\footnotesize\indent\textsc{#1: }\ignorespaces #2}}   
\def\R{\mathbb{R}}
\def\mm#1{\ensuremath{\boldsymbol{#1}}} % version: amsmath
\def\Var{\text{Var}}
\def\E{\text{E}}
\def\Prec{\text{Prec}}

\linespread{1.2}

\begin{document}
\bibliographystyle{apalike}
{\Large Exponential  and Weibull Models}\\
\vspace{0.5cm}\\
We consider the  data of times to infection of kidney dialysis patients.
In a study \cite{art504}, (given in the book by \cite{book501})  designed to assess 
the time to first exitsite
infection (in months) in patients with renal insufficiency, 43 patients
utilized a surgically placed catheter (Group 1), and 76 patients utilized
a percutaneous placement of their catheter (Group 2), a total of 119 patients.\\
The variables represented in the data set are $\tt{time}$ to infection in months/$10$
denoted by $t$, infection indicator or $\tt{event}$ ($0$=no, $1$=yes) denoted by  $\delta$ and
 catheter $\tt{placement}$ (1=surgically, 2=percutaneously) denoted by  $trt$. We analyse the data set using exponential model and Weibull model.\\
The \textbf{exponential model} for this example can be specified as:
\begin{displaymath}
t_{i}\sim \mathit{E}(\lambda_{i})
\end{displaymath}
Where each survival time follows an exponential distribution with parameter $\lambda_{i}$  and  $i$ is from $1$ to $119$. 
For this example we have only one covariate, catheter placement ($trt$) 
and therefore $ \vect{\beta} = (\beta_{0},\beta_{1})^{'} $, where $\beta_{0}$ denotes the
intercept term and $\beta_{1}$ denotes the coefficient for the placement covariate ($trt$). Here, the latent field is
\begin{displaymath}
\lambda_{i}= \exp(\eta_{i}) 
\end{displaymath}
with
\begin{displaymath} 
\eta_{i}= \beta_{0}+ trt_{i}\beta_{1} 
\end{displaymath}
where both $\beta_{0}$ and $\beta_{1}$ are assign the following priors distributions
\begin{displaymath}
\beta_{0}\sim N(0,0.001)
\end{displaymath}
\begin{displaymath}
\beta_{1}\sim N(0,0.001)
\end{displaymath} 
There is no hyperparameter used in this model.\\
\\
The \textbf{Weibull model} for this example can be specified as:
\begin{displaymath}
t_{i}\sim \text{Weibull}(\alpha,\lambda_{i})
\end{displaymath}
Here also , the latent field is
\begin{displaymath}
\lambda_{i}= \exp(\eta_{i}) 
\end{displaymath}
with
\begin{displaymath} 
\eta_{i}= \beta_{0}+ trt_{i}\beta_{1} 
\end{displaymath}
where $\beta_{0}$ and $\beta_{1}$ are assign the following priors distributions
\begin{displaymath}
\beta_{0}\sim N(0,0.001)
\end{displaymath}
\begin{displaymath}
\beta_{1}\sim N(0,0.001)
\end{displaymath} 
The model has one hyperparameter, $\alpha$ , we assign the following  prior distribution
\begin{displaymath}
\alpha \sim \text{Gamma}(1, 0.001)
\end{displaymath} 
 
 
 
\bibliography{rupalibib} 
\end{document}