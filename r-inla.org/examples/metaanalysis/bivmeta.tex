\documentclass[a4paper,11pt]{article}
\usepackage[compat2]{geometry}
\usepackage{amstext}
\usepackage{listings}
\usepackage{amsmath,amssymb}

\def\mm#1{\ensuremath{\boldsymbol{#1}}} % version: amsmath


\begin{document}
\bibliographystyle{apalike}

\section*{Bivariate meta-analysis of sensitivity and specificity}
{\small{[This example was kindly
        provided by Andrea Riebler and Michaela Paul, from the
        University of Zurich. Thanks!]}}

The bivariate model is a model for meta-analysing diagnostic studies
reporting pairs of sensitivity and specificity (\cite{bivariate2}).
Preserving the bivariate structure of the data, pairs of sensitivity
(Se) and specificity (Sp) are jointly analysed. Any existing
correlation between these two measures is taken into account via
random effects. Covariates can be added to the bivariate model and
have a separate effect on sensitivity and specificity.

 Data are taken from a meta-analysis conducted by \cite{bivariate3}
    to compare the utility of three types of diagnostic imaging -
    lymphangiography (LAG), computed tomography (CT) and magnetic
    resonance (MR) - to detect lymph node metastases in patients with
    cervical cancer. The dataset consists of a total of $46$ studies:
    the first $17$ for LAG, the following $19$ for CT and the last
    $10$ for $MR$. We analyse this data set using a generalised linear
    mixed model approach (\cite{bivariate1}).
    \begin{align}
        \text{TN}{^i}|\mu_i &\sim \text{Bin}(\text{TN}{^i} +
        \text{FP}{^i}, \text{Sp}{^i}),&
        \text{logit}(\text{Sp}{^i}) &= \mm{X}_i \mm{\alpha} + \mu_i,\\
        \text{TP}{^i}|\nu_i &\sim \text{Bin}(\text{TP}{^i} +
        \text{FN}{^i}, \text{Se}{^i}),&
        \text{logit}(\text{Se}{^i}) &= \mm{Z}_i \mm{\beta} + \nu_i,\\[0.3cm]
        {\mu_i \choose \nu_i} &\sim \mathcal{N} \left[ {0 \choose 0},
          \begin{pmatrix} 1/\tau_\mu & \rho/\sqrt{\tau_\mu \tau_\nu} \\
              \rho/\sqrt{\tau_\mu \tau_\nu} & 1/\tau_\nu
          \end{pmatrix}
        \right], \label{eq:biv}
    \end{align}
    where TN, FP, TP and FN represent the number of true negatives,
    false positives, true positives, and false negatives, respectively
    and $\mm{X}_i, \mm{Z}_i$ are (possibly overlapping) vectors of
    covariates related to $\text{Sp} = \tfrac{\text{TN}}{\text{TN} +
        \text{FP}}$ and $\text{Se} = \tfrac{\text{TP}}{\text{TP} +
        \text{FN}}$.  The index $i$ represents study $i$ in the
    meta-analysis.  Here, $\mm{X}_i \mm{\alpha} =
    \alpha_{\text{LAG}}\cdot \text{LAG}_i + \alpha_{\text{CT}} \cdot
    \text{CT}_i + \alpha_{\text{MR}} \cdot \text{MR}_i$ and $\mm{Z}_i
    \mm{\beta} = \beta_{\text{LAG}} \cdot \text{LAG}_i +
    \beta_{\text{CT}} \cdot \text{CT}_i + \beta_{\text{MR}} \cdot
    \text{MR}_i$ whereby
    \begin{align*}
        \text{LAG}_i &=
        \begin{cases}
            1 \quad \text{if} \quad i = 0, \ldots, 16\\
            0 \quad \text{else}
        \end{cases}
        \text{CT}_i &=
        \begin{cases}
            1 \quad \text{if} \quad i = 17, \ldots, 35\\
            0 \quad \text{else}
        \end{cases}
        \text{MR}_i &=
        \begin{cases}
            1 \quad \text{if} \quad i = 36, \ldots, 45\\
            0 \quad \text{else}
        \end{cases}
    \end{align*}
    The model has three hyperparameters $\mm{\theta} = (\log
    \tau_\mu,\log \tau_\nu, \rho)$. The correlation parameter is
    constrained to $[-1,1]$.  We reparameterise the correlation
    parameter $\rho$ using Fisher's z-transformation as
    \begin{equation*}
        \rho^\star = \text{logit}\left(\frac{\rho+1}{2} \right)
    \end{equation*}
    which assumes values over the whole real line and assign the
    following prior distribution to $\rho^\star$
    \begin{align*}
        \rho^\star &\sim \mathcal{N}(0,1/0.4) \intertext{The prior
            precision of $0.4$ corresponds, roughly, to a uniform
            prior on $[-1,1]$ for $\rho$.  For the other
            hyperparameters we assign the following prior
            distributions}
        \log\tau_\mu &\sim \text{LogGamma}(0.25, 0.025)\\
        \log\tau_\nu &\sim \text{LogGamma}(0.25, 0.025)\\
    \end{align*}

\small\bibliography{../mybib} \newpage


\end{document}


% LocalWords: 

%%% Local Variables: 
%%% TeX-master: t
%%% End: 
