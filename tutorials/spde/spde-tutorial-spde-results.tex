\chapter{The SPDE approach}\label{ch:spde} 

In the literature there is some ideas to 
fit GF by an approximation of the GF to any GMRF. 
The very good alternative found is shuch one that 
found a explicit link between the an 
stochastic partial differential equation (SPDE) 
and the Mat\'ern Gaussian random fields, 
\cite{lindgrenRL:2011}. 
The solution of the that SPDE 
thought the Finite Element Method (FEM) 
provides a explicit link between GF to GMRF. 

\section{The \cite{lindgrenRL:2011} results} 

The SPDE approach is based on two main results. 
The first one extends the result obtained by \cite{besag:1981}. 
This result is to approximate a GF with generalized 
covariance function, obtained when $\nu \rightarrow 0$ 
in the Mat\'ern correlation function. 
This approximation, considering a regular two-dimensional 
lattice with number of sites tending to infinite, 
is that the full conditional have 
\begin{equation}
E(x_{ij}|x_{-ij}) = \frac{1}{a}(x_{i-1,j}+x_{i+1,j}+x_{i,j-1}+x_{i,j+1})
\end{equation}
and $Var(x_{ij}|x_{-ij}) = 1/a$ for $|a|>4$. 
In the representation using precision matrix, 
we have, for one single site, just the upper right 
quadrant and with $a$ as the central element, that 
\begin{equation}\label{eq:q0}
\begin{matrix}
-1  & \\
a & -1
\end{matrix}
\end{equation}

Considering a GF $x(\bbu)$ with the Mat\'ern 
covariance is a solution to the linear fractional SPDE 
\begin{equation}
(\kappa^2 - \Delta )^{\alpha/2}x(\bbu) = \bW (\bbu ),\;\;\;
\bbu \in \Re^d,\;\;\;\alpha=\nu+d/2,\;\;\kappa>0,\;\;\nu>0,
\end{equation}
\cite{lindgrenRL:2011} show that for $\nu=1$ and $\nu=2$ 
the GMRF representations are convolutions of (\ref{eq:q0}). 
So, for $\nu=1$, in that representation we have:
\begin{equation}
\begin{matrix}
1  & & \\
-2a & 2 & \\
4+a^2 & -2a & 1
\end{matrix}
\end{equation}\label{eq:q1}
and, for $\nu=2$:
\begin{equation}
\begin{matrix}
-1  & & &\\
3a & -3 & &\\
-3(a^2+3) & 6a & 3 & \\
a(a^2+12) & -3(a^2+3) & 3a & -1
\end{matrix}
\end{equation}\label{eq:q2}
This is an intuitive result, because if we have 
larger $\nu$ on the Mat\'ern correlation function of the GF, 
we need more non zero neighbours sites in the GMRF representation. 
Remember that it is the smoothness parameter, so if the 
process is more smooth, we need a more larger neighbourhood 
on the GMRF representation. 

If the spatial locations are on a irregular grid, 
it is necessary the use of the second result on
\cite{lindgrenRL:2011}. 
To extend the first result, a suggestion is the use of 
the finite element method (FEM) for a interpolation of 
the locations of observations to the nearest grid point. 
To do it, suppose that the $\Re^2$ is subdivided into 
a set of non-intersecting triangles, where any two
triangles meet in at most a common edge or corner. 
The three corners of a triangle are named \textit{vertices}.
The suggestion is to start with the location of the 
observed points and add some triangles (heuristically) 
with restriction to maximize the allowed edge length 
and minimize the allowed angles. 
The the approximation is 
\[x(\bbu) = \sum_{k=1}^{n}\psi_k(\bbu)w_k\]
for some chosen basis functions {$\psi_k$}, 
Gaussian distributed weights $w_k$ and $n$ the 
number of vertices on the triangulation.
If the functions $\psi_k$ are piecewise linear in each 
triangle, $\psi_k$ is 1 at vertices $k$ and 0 at all 
other vertices. 

The second result is obtained using the $n\times n$ matrices 
$\bC$, $\bG$ and $\bK$ with entries 
\begin{equation}
C_{i,j} = \langle \psi_i, \psi_j\rangle, \;\;\;\;\;
G_{i,j} = \langle \nabla \psi_i, \nabla \psi_j \rangle, \;\;\;\;\;
(\bK_{\kappa^2})_{i,j} = \kappa^2 C_{i,j} + G_{i,j}
\end{equation}
to get the precision matrix $\bQ_{\alpha,\kappa}$ 
as a function of $\kappa^2$ and $\alpha$: 
\begin{equation}\label{eq:Qalpha}\begin{array}{c}
\bQ_{1,\kappa^2} = \bK_{\kappa^2}, \\
\bQ_{2,\kappa^2} = \bK_{\kappa^2}\bC^{-1}\bK_{\kappa^2}, \\
\bQ_{\alpha,\kappa^2} = \bK_{\kappa^2}\bC^{-1}Q_{\alpha-2,\kappa^2}\bC^{-1}\bK_{\kappa^2}, 
\;\;\;\mbox{for} \;\;\alpha = 3,4,...\;.
\end{array}\end{equation}
Here we have too the notion that if $\nu$ increases, 
we need a more dense precision matrix. 

The $\bQ$ precision matrix is generalized for a fractional values 
of $\alpha$ (or $\nu$) using a Taylor approximation, 
see the author's discussion response in \cite{lindgrenRL:2011}. 
From this approximation, we have the polynomial of 
order $p=\lceil \alpha \rceil$ for the precision matrix 
\begin{equation}\label{eq:Qfrac}
\bQ = \sum_{i=0}^p b_i \bC(\bC^{-1}\bG)^i.
\end{equation}
For $\alpha=1$ and $\alpha=2$ we have the (\ref{eq:Qalpha}). 
Because, for $\alpha=1$, we have $b_0=\kappa^2$ and $b_1=1$, 
and for $\alpha=2$, we have $b_0=\kappa^4$, 
$b_1=\alpha\kappa^4$ and $b_2=1$. 
For fractional $\alpha=1/2$ 
$b_0=3\kappa/4$ and $b_1=\kappa^{-1}3/8$. 
And, for $\alpha=3/2$, ($\nu=0.5$, the exponential case), 
$b_0=15\kappa^3/16$, $b_1=15\kappa/8$, 
$b_2=15\kappa^{-1}/128$. 
Using these results combined with recursive construction, 
for $\alpha>2$, we have GMRF approximations for all positive 
integers and half-integers. 

